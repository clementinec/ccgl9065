% Options for packages loaded elsewhere
\PassOptionsToPackage{unicode}{hyperref}
\PassOptionsToPackage{hyphens}{url}
%
\documentclass[
  ignorenonframetext,
]{beamer}
\usepackage{pgfpages}
\setbeamertemplate{caption}[numbered]
\setbeamertemplate{caption label separator}{: }
\setbeamercolor{caption name}{fg=normal text.fg}
\beamertemplatenavigationsymbolsempty
% Prevent slide breaks in the middle of a paragraph
\widowpenalties 1 10000
\raggedbottom
\setbeamertemplate{part page}{
  \centering
  \begin{beamercolorbox}[sep=16pt,center]{part title}
    \usebeamerfont{part title}\insertpart\par
  \end{beamercolorbox}
}
\setbeamertemplate{section page}{
  \centering
  \begin{beamercolorbox}[sep=12pt,center]{part title}
    \usebeamerfont{section title}\insertsection\par
  \end{beamercolorbox}
}
\setbeamertemplate{subsection page}{
  \centering
  \begin{beamercolorbox}[sep=8pt,center]{part title}
    \usebeamerfont{subsection title}\insertsubsection\par
  \end{beamercolorbox}
}
\AtBeginPart{
  \frame{\partpage}
}
\AtBeginSection{
  \ifbibliography
  \else
    \frame{\sectionpage}
  \fi
}
\AtBeginSubsection{
  \frame{\subsectionpage}
}

\usepackage{amsmath,amssymb}
\usepackage{iftex}
\ifPDFTeX
  \usepackage[T1]{fontenc}
  \usepackage[utf8]{inputenc}
  \usepackage{textcomp} % provide euro and other symbols
\else % if luatex or xetex
  \usepackage{unicode-math}
  \defaultfontfeatures{Scale=MatchLowercase}
  \defaultfontfeatures[\rmfamily]{Ligatures=TeX,Scale=1}
\fi
\usepackage{lmodern}
\ifPDFTeX\else  
    % xetex/luatex font selection
\fi
% Use upquote if available, for straight quotes in verbatim environments
\IfFileExists{upquote.sty}{\usepackage{upquote}}{}
\IfFileExists{microtype.sty}{% use microtype if available
  \usepackage[]{microtype}
  \UseMicrotypeSet[protrusion]{basicmath} % disable protrusion for tt fonts
}{}
\makeatletter
\@ifundefined{KOMAClassName}{% if non-KOMA class
  \IfFileExists{parskip.sty}{%
    \usepackage{parskip}
  }{% else
    \setlength{\parindent}{0pt}
    \setlength{\parskip}{6pt plus 2pt minus 1pt}}
}{% if KOMA class
  \KOMAoptions{parskip=half}}
\makeatother
\usepackage{xcolor}
\newif\ifbibliography
\setlength{\emergencystretch}{3em} % prevent overfull lines
\setcounter{secnumdepth}{-\maxdimen} % remove section numbering


\providecommand{\tightlist}{%
  \setlength{\itemsep}{0pt}\setlength{\parskip}{0pt}}\usepackage{longtable,booktabs,array}
\usepackage{calc} % for calculating minipage widths
\usepackage{caption}
% Make caption package work with longtable
\makeatletter
\def\fnum@table{\tablename~\thetable}
\makeatother
\usepackage{graphicx}
\makeatletter
\def\maxwidth{\ifdim\Gin@nat@width>\linewidth\linewidth\else\Gin@nat@width\fi}
\def\maxheight{\ifdim\Gin@nat@height>\textheight\textheight\else\Gin@nat@height\fi}
\makeatother
% Scale images if necessary, so that they will not overflow the page
% margins by default, and it is still possible to overwrite the defaults
% using explicit options in \includegraphics[width, height, ...]{}
\setkeys{Gin}{width=\maxwidth,height=\maxheight,keepaspectratio}
% Set default figure placement to htbp
\makeatletter
\def\fps@figure{htbp}
\makeatother

\makeatletter
\@ifpackageloaded{caption}{}{\usepackage{caption}}
\AtBeginDocument{%
\ifdefined\contentsname
  \renewcommand*\contentsname{Table of contents}
\else
  \newcommand\contentsname{Table of contents}
\fi
\ifdefined\listfigurename
  \renewcommand*\listfigurename{List of Figures}
\else
  \newcommand\listfigurename{List of Figures}
\fi
\ifdefined\listtablename
  \renewcommand*\listtablename{List of Tables}
\else
  \newcommand\listtablename{List of Tables}
\fi
\ifdefined\figurename
  \renewcommand*\figurename{Figure}
\else
  \newcommand\figurename{Figure}
\fi
\ifdefined\tablename
  \renewcommand*\tablename{Table}
\else
  \newcommand\tablename{Table}
\fi
}
\@ifpackageloaded{float}{}{\usepackage{float}}
\floatstyle{ruled}
\@ifundefined{c@chapter}{\newfloat{codelisting}{h}{lop}}{\newfloat{codelisting}{h}{lop}[chapter]}
\floatname{codelisting}{Listing}
\newcommand*\listoflistings{\listof{codelisting}{List of Listings}}
\makeatother
\makeatletter
\makeatother
\makeatletter
\@ifpackageloaded{caption}{}{\usepackage{caption}}
\@ifpackageloaded{subcaption}{}{\usepackage{subcaption}}
\makeatother
\ifLuaTeX
  \usepackage{selnolig}  % disable illegal ligatures
\fi
\usepackage{bookmark}

\IfFileExists{xurl.sty}{\usepackage{xurl}}{} % add URL line breaks if available
\urlstyle{same} % disable monospaced font for URLs
\hypersetup{
  pdftitle={CCGL9065: Our Response to Climate Change: HK2100},
  pdfauthor={Dr.~Hongshan Guo and Class},
  hidelinks,
  pdfcreator={LaTeX via pandoc}}

\title{CCGL9065: Our Response to Climate Change: HK2100}
\subtitle{Futurologist, Fake News and Histories}
\author{Dr.~Hongshan Guo and Class}
\date{}

\begin{document}
\frame{\titlepage}

\section{This Week's Battlefield}\label{this-weeks-battlefield}

\begin{frame}{Two Sides. Two Views on Truth.}
\phantomsection\label{two-sides.-two-views-on-truth.}
PRO-CLIMATE

= Scientific Consensus Matters

= ``Denialism is dangerous''

PRO-DEVELOPMENT

= Question Everything

= ``Skepticism is healthy''
\end{frame}

\begin{frame}{The Core Tension}
\phantomsection\label{the-core-tension}
\begin{longtable}[]{@{}ll@{}}
\toprule\noalign{}
PRO-CLIMATE & PRO-DEVELOPMENT \\
\midrule\noalign{}
\endhead
Trust the science & Question the models \\
Expert consensus & Healthy skepticism \\
Urgency requires action & Uncertainty requires caution \\
Denialism is funded & Alarmism is funded too \\
Facts over feelings & Predictions often fail \\
\bottomrule\noalign{}
\end{longtable}

\textbf{This tension defines debates about truth, expertise, and
action.}
\end{frame}

\begin{frame}{1: Introduction to Futurology}
\phantomsection\label{introduction-to-futurology}
\begin{itemize}
\tightlist
\item
  \textbf{Futurology Defined}: Systematic study of future possibilities.
\item
  \textbf{Diverse Practitioners}: Wide range of contributors.

  \begin{itemize}
  \tightlist
  \item
    \emph{Insurance agents}: Assess future risks.
  \item
    \emph{Artists}: Imagine futuristic concepts.
  \end{itemize}
\item
  Sci-Fi to life: Arthur C. Clarke's satellite prediction

  \begin{itemize}
  \tightlist
  \item
    Predicted geostationary (on-orbit) satellites in 1945.
  \item
    Basis for modern communications.
  \item
    Launch of Syncom (1963)
  \end{itemize}
\end{itemize}
\end{frame}

\begin{frame}{2: Learning from the Past}
\phantomsection\label{learning-from-the-past}
\begin{itemize}
\tightlist
\item
  \textbf{Historical Patterns}: Guide to forecasting.
\item
  \textbf{Einstein's Insight}: ``New thinking for new problems.''

  \begin{itemize}
  \tightlist
  \item
    Recognizing the need for innovative solutions.
  \end{itemize}
\end{itemize}
\end{frame}

\begin{frame}{3: Multiple Histories}
\phantomsection\label{multiple-histories}
\begin{itemize}
\tightlist
\item
  \textbf{Varied Perspectives}: History's multifaceted nature.
\item
  \textbf{Impact on Worldviews}: Shapes perception of the future.

  \begin{itemize}
  \tightlist
  \item
    \emph{Western vs.~Eastern narratives}: Different focuses and
    lessons.
  \end{itemize}
\end{itemize}
\end{frame}

\begin{frame}{4: Alternative Histories}
\phantomsection\label{alternative-histories}
\begin{itemize}
\tightlist
\item
  \textbf{`What Ifs' Exploration}: Examining different historical
  outcomes.

  \begin{itemize}
  \tightlist
  \item
    \emph{Scenario}: If the Library of Alexandria survived.

    \begin{itemize}
    \tightlist
    \item
      Potential acceleration of scientific progress.
    \end{itemize}
  \end{itemize}
\item
  \textbf{Influence on Future Thinking}: Alternative pasts lead to
  diverse futures.
\end{itemize}
\end{frame}

\begin{frame}{5: The Power of Stories}
\phantomsection\label{the-power-of-stories}
\begin{itemize}
\tightlist
\item
  \textbf{Narratives Influence Expectations}: Stories shape our view of
  the future.
\item
  \textbf{Choosing Narratives}: Selecting which stories to carry
  forward.

  \begin{itemize}
  \tightlist
  \item
    \emph{Moon Landing}: Symbolizes human achievement and potential.
  \end{itemize}
\end{itemize}
\end{frame}

\begin{frame}{Moon Landing}
\phantomsection\label{moon-landing}
\begin{figure}[H]

{\centering \includegraphics{data/moonlandingcons.png}

}

\caption{Wiki Page on Moon Landing Conspiracy}

\end{figure}%
\end{frame}

\begin{frame}{5.1: Moon Landing Controversy - USA's Perspective}
\phantomsection\label{moon-landing-controversy---usas-perspective}
\begin{itemize}
\tightlist
\item
  \textbf{Historic Achievement}:

  \begin{itemize}
  \tightlist
  \item
    Apollo 11 moon landing in 1969 celebrated as a monumental success.
  \item
    Symbolized technological and exploratory supremacy.
  \end{itemize}
\item
  \textbf{USSR's Skepticism}:

  \begin{itemize}
  \tightlist
  \item
    Claims of inconsistencies in moon landing footage and photos.
  \item
    Allegations of the moon landing being a Cold War propaganda tool.
  \end{itemize}
\item
  \textbf{Talking Points}:

  \begin{itemize}
  \tightlist
  \item
    USA emphasized transparency, broadcasting the landing live.
  \item
    Highlighted extensive documentation and astronaut testimonials.
  \end{itemize}
\item
  \textbf{Contradictions \& Counterclaims}:

  \begin{itemize}
  \tightlist
  \item
    Counteracted skepticism with scientific explanations and physical
    moon rock samples.
  \item
    Addressed conspiracy theories directly in public discourse.
  \end{itemize}
\end{itemize}
\end{frame}

\begin{frame}{5.2: Moon Landing Controversy - USSR's Perspective}
\phantomsection\label{moon-landing-controversy---ussrs-perspective}
\begin{itemize}
\tightlist
\item
  \textbf{Space Race Competition}:

  \begin{itemize}
  \tightlist
  \item
    Intense rivalry to achieve significant milestones first.
  \item
    Propaganda used to highlight Soviet space achievements.
  \end{itemize}
\item
  \textbf{Casting Doubt}:

  \begin{itemize}
  \tightlist
  \item
    Some Soviet officials and media suggested the moon landing could be
    fabricated.
  \item
    Raised questions about the technological feasibility and safety.
  \end{itemize}
\item
  \textbf{Talking Points}:

  \begin{itemize}
  \tightlist
  \item
    Emphasized Soviet space firsts, like launching the first satellite,
    Sputnik, and first man in space, Yuri Gagarin.
  \item
    Questioned the authenticity of NASA's moon landing evidence.
  \end{itemize}
\item
  \textbf{Contradictions \& Counterclaims}:

  \begin{itemize}
  \tightlist
  \item
    Despite public skepticism, some Soviet scientists acknowledged the
    moon landing's authenticity.
  \item
    Over time, official stance softened, recognizing the achievement.
  \end{itemize}
\end{itemize}
\end{frame}

\begin{frame}{6: Questioning Historical Accuracy}
\phantomsection\label{questioning-historical-accuracy}
\begin{itemize}
\tightlist
\item
  \textbf{Reliability Issues}: Recognizing history's subjectivity.

  \begin{itemize}
  \tightlist
  \item
    \emph{Revisionist History}: How textbooks can skew perceptions.
  \end{itemize}
\item
  \textbf{Diverse Future Scenarios}: Result from questioning historical
  accuracy.
\end{itemize}
\end{frame}

\begin{frame}{7: The Influence of Fake Histories}
\phantomsection\label{the-influence-of-fake-histories}
\begin{itemize}
\tightlist
\item
  \textbf{Fiction's Impact}: How fabricated stories can shape collective
  memory.

  \begin{itemize}
  \tightlist
  \item
    \emph{Orson Welles' ``War of the Worlds''}:

    \begin{itemize}
    \tightlist
    \item
      1938 radio drama caused public panic.
    \item
      Demonstrates power of narrative to influence reality.
    \end{itemize}
  \end{itemize}
\item
  \textbf{Critical Thinking}: Importance of discerning fact from
  fiction.
\end{itemize}
\end{frame}

\begin{frame}{8: Historical Revisionism}
\phantomsection\label{historical-revisionism}
\begin{itemize}
\tightlist
\item
  \textbf{Changing Histories}: Continuous reinterpretation of past
  events.

  \begin{itemize}
  \tightlist
  \item
    \emph{Reassessing Legacies}: How views on historical figures evolve.

    \begin{itemize}
    \tightlist
    \item
      Shifts in perception about figures like Columbus or Churchill.
    \end{itemize}
  \end{itemize}
\item
  \textbf{Adapting Predictions}: Updating future forecasts with new
  historical insights.
\end{itemize}
\end{frame}

\begin{frame}{9: Crafting Future Narratives}
\phantomsection\label{crafting-future-narratives}
\begin{itemize}
\tightlist
\item
  \textbf{Imagining Future Stories}: Visioning what tales we'll tell.

  \begin{itemize}
  \tightlist
  \item
    \emph{Elon Musk's Mars Plan}: Envisions human settlement on Mars.

    \begin{itemize}
    \tightlist
    \item
      Represents ambition, exploration, and potential human resilience.
    \end{itemize}
  \end{itemize}
\item
  \textbf{Narratives' Power}: Today's stories shape tomorrow's
  realities.
\end{itemize}
\end{frame}

\begin{frame}{9.1: Western Media Perspective on the Ukrainian Invasion}
\phantomsection\label{western-media-perspective-on-the-ukrainian-invasion}
\begin{itemize}
\tightlist
\item
  \textbf{Aggression Framing}:

  \begin{itemize}
  \tightlist
  \item
    Described as an unprovoked act of aggression by Russia against
    Ukraine's sovereignty.
  \item
    Emphasis on international law violations.
  \end{itemize}
\item
  \textbf{Global Response}:

  \begin{itemize}
  \tightlist
  \item
    Coverage of widespread international condemnation and sanctions
    against Russia.
  \item
    Support for Ukraine highlighted, including aid and refugee
    assistance.
  \end{itemize}
\item
  \textbf{Humanitarian Focus}:

  \begin{itemize}
  \tightlist
  \item
    Reports on the humanitarian crisis, including civilian casualties
    and displacement.
  \item
    Stories of Ukrainian resilience and resistance.
  \end{itemize}
\item
  \textbf{Critiques of Russian Narrative}:

  \begin{itemize}
  \tightlist
  \item
    Questioning of Russian motives and justifications for the invasion.
  \item
    Examination of the impact on global stability and European security.
  \end{itemize}
\end{itemize}
\end{frame}

\begin{frame}{9.2: Russian-Friendly Media Perspective on the Ukrainian
Invasion}
\phantomsection\label{russian-friendly-media-perspective-on-the-ukrainian-invasion}
\begin{itemize}
\tightlist
\item
  \textbf{Security Concerns}:

  \begin{itemize}
  \tightlist
  \item
    Framing the action as a response to security threats and NATO's
    eastward expansion.
  \item
    Emphasis on protecting Russian-speaking populations in Ukraine.
  \end{itemize}
\item
  \textbf{Historical Context}:

  \begin{itemize}
  \tightlist
  \item
    References to historical ties between Russia and Ukraine to justify
    intervention.
  \item
    Portrayal of the action as reclamation or unification, not invasion.
  \end{itemize}
\item
  \textbf{Western Influence}:

  \begin{itemize}
  \tightlist
  \item
    Accusations of Western meddling in Ukrainian affairs and provoking
    conflict.
  \item
    Critique of Western sanctions as unjust and harmful to global
    relations.
  \end{itemize}
\item
  \textbf{Narrative Control}:

  \begin{itemize}
  \tightlist
  \item
    Attempts to control the narrative through state media and censorship
    of dissenting views.
  \item
    Dismissal of Western reports as biased or fake news.
  \end{itemize}
\end{itemize}
\end{frame}

\begin{frame}{10: Embracing Complexity in Futurology}
\phantomsection\label{embracing-complexity-in-futurology}
\begin{itemize}
\tightlist
\item
  \textbf{Forecasting Complexity}: The multifaceted nature of predicting
  the future.
\item
  \textbf{Informed Approaches}: Leveraging a nuanced understanding of
  history.

  \begin{itemize}
  \tightlist
  \item
    \emph{Butterfly Effect}: Small changes can lead to significant
    consequences.

    \begin{itemize}
    \tightlist
    \item
      Understanding chaos theory's implications for prediction.
    \end{itemize}
  \end{itemize}
\end{itemize}
\end{frame}

\begin{frame}{11: Known Historical Discrepancies of Climate-Change
Related Stories}
\phantomsection\label{known-historical-discrepancies-of-climate-change-related-stories}
\end{frame}

\begin{frame}{Mideval Warm Period}
\phantomsection\label{mideval-warm-period}
\begin{figure}[H]

{\centering \includegraphics{data/MWP.png}

}

\caption{Mideval Warm Period (Wikipedia)}

\end{figure}%
\end{frame}

\begin{frame}{11.1: \textbf{Medieval Warm Period (MWP) Debate}:}
\phantomsection\label{medieval-warm-period-mwp-debate}
\begin{itemize}
\tightlist
\item
  \textbf{Controversy}: The MWP refers to a time from about 950 to 1250
  AD when temperatures were thought to be \textbf{unusually warm} in
  some regions of the North Atlantic. The debate centers around the
  extent, timing, and global impact of the MWP.
\item
  \textbf{Differing Interpretations}:

  \begin{itemize}
  \tightlist
  \item
    Some climate change skeptics have used the MWP to argue that
    \emph{current global warming is part of a natural climate
    variability.}
  \item
    In contrast, the majority of climate scientists contend that current
    warming is \textbf{unprecedented and largely anthropogenic}.
  \end{itemize}
\item
  \textbf{Impact on Climate Change Discourse}: The controversy has
  fueled discussions about the reliability of climate models and
  historical climate data reconstructions.
\end{itemize}
\end{frame}

\begin{frame}
\end{frame}

\section{Building Your Truth
Spectacle}\label{building-your-truth-spectacle}

\begin{frame}{The Formula (Reminder)}
\phantomsection\label{the-formula-reminder}
\textbf{Fact} + \textbf{Human Story} + \textbf{Stakes} =
\textbf{Spectacle}

Weak

``Misinformation is a problem''

Better

``Oil companies funded climate denial for 40 years''

Spectacle

``ExxonMobil's own scientists predicted climate change in 1982. Then the
company spent millions telling you it wasn't real. They knew. They lied.
You paid.''
\end{frame}

\begin{frame}{PRO-CLIMATE: Make It Personal}
\phantomsection\label{pro-climate-make-it-personal}
\textbf{Don't say:} ``Climate denial is funded by fossil fuel
companies.''

\textbf{Say:} ``The same playbook. The same PR firms. Tobacco companies
denied cancer for decades. Oil companies denied warming for decades. You
were the mark both times.''

\textbf{Don't say:} ``Trust the scientific consensus.''

\textbf{Say:} ``97\% of climate scientists agree. That's the same
consensus level as `smoking causes cancer.' You wouldn't bet your life
on the 3\%. Why bet your grandchildren's?''
\end{frame}

\begin{frame}{PRO-DEVELOPMENT: Paint the Picture}
\phantomsection\label{pro-development-paint-the-picture}
\textbf{Don't say:} ``Predictions have been wrong before.''

\textbf{Say:} ``In 1970, scientists predicted an ice age. In 1989, they
said the Maldives would be underwater by 2018. The Maldives just opened
8 new luxury resorts. Excuse us for being skeptical.''

\textbf{Don't say:} ``Healthy skepticism is scientific.''

\textbf{Say:} ``They called Galileo a denier too. Science advances by
questioning consensus, not by silencing dissent. Who's the real
anti-science side?''
\end{frame}

\begin{frame}{Climate-Change as Fake, News from Trump's Claim (Case 1)}
\phantomsection\label{climate-change-as-fake-news-from-trumps-claim-case-1}
\begin{quote}
``The concept of global warming was created by and for the Chinese in
order to make U.S. manufacturing non-competitive.'' \emph{(Trump, Tweet,
2012)}
\end{quote}

\note{\textbf{Critical Assessment (Speaker Notes):}\\
Encourage students to verify statements through reputable, science-based
historical timelines and scholarly publications, demonstrating that
conspiracy claims typically lack credible sources.}
\end{frame}

\begin{frame}
\begin{block}{Explanation \& Fact-check}
\phantomsection\label{explanation-fact-check}
\textbf{Explanation:}\\
This is a conspiracy theory claiming climate change was intentionally
fabricated by China to damage U.S. economic interests, without
scientific or historical basis.

\textbf{Fact-check:}\\
Climate science dates back to the late 19th century, long before
contemporary Chinese economic policies. It's supported by NASA, NOAA,
and the IPCC.
\end{block}
\end{frame}

\begin{frame}{Climate-Change as Fake, News from Trump's Claim, Case Two}
\phantomsection\label{climate-change-as-fake-news-from-trumps-claim-case-two}
\begin{quote}
``It used to not be climate change, it used to be global warming\ldots{}
That wasn't working too well because it was getting too cold all over
the place.''\\
\emph{(Trump, Interview with Piers Morgan, 2018)}
\end{quote}

\note{\textbf{Critical Assessment (Speaker Notes):}\\
Educate students on the difference between short-term weather (variable)
and climate (long-term averages and trends) to prevent conflating
temporary phenomena with overall climate patterns.}
\end{frame}

\begin{frame}
\begin{block}{Explanation \& Fact-check}
\phantomsection\label{explanation-fact-check-1}
\textbf{Explanation:}\\
Trump confuses short-term weather variations with the long-term trend of
climate change.

\textbf{Fact-check:}\\
Global temperatures have consistently risen over decades (NASA/IPCC).
Terminology shifted to ``climate change'' reflecting broader impacts.
\end{block}
\end{frame}

\begin{frame}{Hocky Stick Graph}
\phantomsection\label{hocky-stick-graph}
\begin{figure}[H]

{\centering \includegraphics{data/hockey.jpg}

}

\caption{Hocky Stick Graph (Wikipedia)}

\end{figure}%
\end{frame}

\begin{frame}{11.2: \textbf{Hockey Stick Graph Controversy}:}
\phantomsection\label{hockey-stick-graph-controversy}
\begin{itemize}
\tightlist
\item
  \textbf{Controversy}:

  \begin{itemize}
  \tightlist
  \item
    The ``hockey stick graph,'' first published by Michael Mann and
    colleagues in the late 1990s, showed a sharp rise in global
    temperatures in the 20th century after a long period of relative
    stability, resembling a hockey stick.
  \item
    Critics questioned the data sources, methodologies, and statistical
    techniques used to create the graph.
  \end{itemize}
\item
  \textbf{Differing Interpretations}:

  \begin{itemize}
  \tightlist
  \item
    Skeptics used the controversy to challenge the consensus on
    anthropogenic global warming,\\
  \item
    numerous scientific bodies and researchers have since reaffirmed the
    graph's general conclusion about significant recent warming.
  \end{itemize}
\item
  \textbf{Impact on Climate Change Discourse}: The controversy
  highlighted the challenges of paleoclimate reconstruction and the
  politicization of climate science.
\end{itemize}
\end{frame}

\begin{frame}{Climategate}
\phantomsection\label{climategate}
\begin{figure}[H]

{\centering \includegraphics{CCGL9065_W7_files/mediabag/maxresdefault.jpg}

}

\caption{Climate Gate Snippet}

\end{figure}%
\end{frame}

\begin{frame}{11.3: \textbf{Climategate Email Controversy}:}
\phantomsection\label{climategate-email-controversy}
\begin{itemize}
\tightlist
\item
  \textbf{Controversy}:

  \begin{itemize}
  \tightlist
  \item
    In 2009, a significant number of emails and documents were leaked
    from the University of East Anglia's Climatic Research Unit (CRU).
  \item
    Critics alleged that the emails showed scientists discussing
    \emph{ways to manipulate data and suppress dissenting views}.
  \end{itemize}
\item
  \textbf{Differing Interpretations}:

  \begin{itemize}
  \tightlist
  \item
    Climate change skeptics claimed the emails proved scientific
    misconduct and conspiracy among climate scientists.
  \item
    However, several independent investigations cleared the scientists
    of wrongdoing, concluding that the emails were taken out of context.
  \end{itemize}
\item
  \textbf{Impact on Climate Change Discourse}: ``Climategate'' fueled
  public skepticism about climate science and the integrity of climate
  researchers, despite the lack of evidence for scientific fraud.
\end{itemize}
\end{frame}

\begin{frame}{Climate-Change as Fake, News from Trump's Claim, Case
Three}
\phantomsection\label{climate-change-as-fake-news-from-trumps-claim-case-three}
\begin{quote}
``The ocean is going to rise by 1/100th of an inch over 400 years.
That's not our problem.''\\
\emph{(Trump, Rally, December 2015)}
\end{quote}

\note{\textbf{Critical Assessment (Speaker Notes):}\\
Guide students in consulting current scientific models and official
reports, emphasizing reliance on measured data rather than anecdotal
claims.}
\end{frame}

\begin{frame}
\begin{block}{Explanation \& Fact-check}
\phantomsection\label{explanation-fact-check-2}
\textbf{Explanation:}\\
Trump significantly understates sea-level rise projections and the
associated risks.

\textbf{Fact-check:}\\
IPCC projects global sea levels could rise approximately 0.3--1.1 meters
by 2100, causing severe impacts on coastal communities.
\end{block}
\end{frame}

\begin{frame}{12.1 Enhancing Perspectives}
\phantomsection\label{enhancing-perspectives}
\begin{block}{What We Are Doing:}
\phantomsection\label{what-we-are-doing}
\begin{itemize}
\tightlist
\item
  Engaging in a role-play activity that simulates a debate around the
  theme of Decolonizing Ocean Science within the Hong Kong context.
\item
  You will assume roles advocating for Pro-Climate policies and
  Pro-Development viewpoints, as well as various stakeholder positions.
\end{itemize}
\end{block}

\begin{block}{Why We Are Doing It:}
\phantomsection\label{why-we-are-doing-it}
\begin{itemize}
\tightlist
\item
  To explore diverse perspectives on complex issues that intersects
  futurologists, fake news, and power of narratives in face of
  historical accounts.
\item
  To understand the implications of policy decisions and scientific
  practices on different communities and the environment.
\item
  \textbf{To foster empathy and critical thinking} by stepping into the
  shoes of various stakeholders affected by these issues.
\end{itemize}
\end{block}
\end{frame}

\begin{frame}{12.2 Enhancing Perspectives (Continued)}
\phantomsection\label{enhancing-perspectives-continued}
\begin{block}{How does these activities help:}
\phantomsection\label{how-does-these-activities-help}
\begin{itemize}
\tightlist
\item
  They highlights the nuances in the climate change debate on various
  issues - angles of historical accounts that can/may be contradictory
  for this week.
\item
  Encouraging informed discussions on how traditional and scientific
  knowledge can complement each other in addressing pressing
  environmental challenges.
\end{itemize}
\end{block}

\begin{block}{Our Classroom Environment:}
\phantomsection\label{our-classroom-environment}
\begin{itemize}
\tightlist
\item
  This is a \textbf{safe space} for exploration and discussion.
\item
  There are no wrong answers here, only opportunities to learn and
  understand different viewpoints.
\item
  Respect and open-mindedness are our guiding principles. Every opinion
  shared is valued and contributes to our collective learning.
\item
  Feedback and reflection are encouraged. This is a chance to voice
  thoughts, ask questions, and grow from the experience.
\end{itemize}
\end{block}
\end{frame}

\begin{frame}{\href{countdown.qmd}{Let's Assign Groups Again}}
\phantomsection\label{lets-assign-groups-again}
\end{frame}

\begin{frame}{Group Discussion Reform: Climate Change Persona Discussion
(10 min)}
\phantomsection\label{group-discussion-reform-climate-change-persona-discussion-10-min}
\begin{enumerate}
\tightlist
\item
  \textbf{Introductions}

  \begin{itemize}
  \tightlist
  \item
    Briefly introduce your persona:

    \begin{itemize}
    \tightlist
    \item
      Name, occupation, or role in society.
    \item
      One sentence on how climate change directly impacts your persona.
    \end{itemize}
  \end{itemize}
\item
  \textbf{Round-table: Share Key Insights}

  \begin{itemize}
  \tightlist
  \item
    Identify one major climate issue your persona is most concerned
    about and explain it to others
  \item
    Explain briefly how this issue directly affects your group or
    personal priorities
  \end{itemize}
\item
  \textbf{Converse more}

  \begin{itemize}
  \tightlist
  \item
    Engage with others by asking at least one question to another
    persona about their views or challenges related to climate change.
  \item
    Respond openly, staying true to your persona's motivations, beliefs,
    and professional interests.
  \item
    Any common grounds you could identify? Any persona from the group
    has a better point?
  \item
    Acknowledge the different perspectives in a respectable manner and
    decide on your champions!
  \end{itemize}
\end{enumerate}
\end{frame}

\begin{frame}{Extra: Perspective Cheatsheets}
\phantomsection\label{extra-perspective-cheatsheets}
\end{frame}

\begin{frame}{Ex.1: Pro-Climate Policy Suggestions}
\phantomsection\label{ex.1-pro-climate-policy-suggestions}
\begin{itemize}
\tightlist
\item
  \textbf{Aggressive Emission Reductions}:

  \begin{itemize}
  \tightlist
  \item
    Implement strict carbon caps.
  \item
    Incentivize renewable energy adoption.
  \end{itemize}
\item
  \textbf{Sustainable Infrastructure}:

  \begin{itemize}
  \tightlist
  \item
    Invest in green public transport.
  \item
    Retrofit buildings for energy efficiency.
  \end{itemize}
\item
  \textbf{Conservation Efforts}:

  \begin{itemize}
  \tightlist
  \item
    Expand protected natural areas.
  \item
    Promote biodiversity restoration projects.
  \end{itemize}
\item
  \textbf{Green Innovation}:

  \begin{itemize}
  \tightlist
  \item
    Fund research in sustainable technologies.
  \item
    Support startups with green solutions.
  \end{itemize}
\item
  \textbf{Global Cooperation}:

  \begin{itemize}
  \tightlist
  \item
    Strengthen international climate agreements.
  \item
    Provide aid for vulnerable countries' climate resilience.
  \end{itemize}
\end{itemize}
\end{frame}

\begin{frame}{Ex.2: Pro-Development Policy Stances}
\phantomsection\label{ex.2-pro-development-policy-stances}
\begin{itemize}
\tightlist
\item
  \textbf{Economic Growth Focus}:

  \begin{itemize}
  \tightlist
  \item
    Prioritize policies that ensure economic stability.
  \item
    Balance environmental regulations with business interests.
  \end{itemize}
\item
  \textbf{Energy Independence}:

  \begin{itemize}
  \tightlist
  \item
    Support a diverse energy portfolio, including fossil fuels.
  \item
    Invest in clean coal and natural gas technologies.
  \end{itemize}
\item
  \textbf{Market-Driven Solutions}:

  \begin{itemize}
  \tightlist
  \item
    Encourage voluntary corporate sustainability initiatives.
  \item
    Leverage market forces to drive environmental innovation.
  \end{itemize}
\item
  \textbf{Adaptation Strategies}:

  \begin{itemize}
  \tightlist
  \item
    Focus on adapting infrastructure to withstand climate impacts.
  \item
    Invest in flood defenses and drought-resistant agriculture.
  \end{itemize}
\item
  \textbf{Regulatory Caution}:

  \begin{itemize}
  \tightlist
  \item
    Avoid over-regulation that could hinder industrial competitiveness.
  \item
    Implement flexible policies that allow for business innovation.
  \end{itemize}
\end{itemize}
\end{frame}

\begin{frame}{Ex.3 General Public Engagement and Response to Climate
Challenges and Misinformation}
\phantomsection\label{ex.3-general-public-engagement-and-response-to-climate-challenges-and-misinformation}
\begin{itemize}
\tightlist
\item
  \textbf{Logistics Professionals}:

  \begin{itemize}
  \tightlist
  \item
    \textbf{Sustainable Practices}: Adopt green logistics and
    transportation methods.
  \item
    \textbf{Fact-Checking}: Verify sources when addressing
    climate-related logistics issues.
  \end{itemize}
\item
  \textbf{Policymakers/Bureaucrats}:

  \begin{itemize}
  \tightlist
  \item
    \textbf{Informed Policy Making}: Base policies on scientific
    evidence and consensus.
  \item
    \textbf{Public Education}: Lead initiatives to educate the public on
    climate facts vs.~misinformation.
  \end{itemize}
\item
  \textbf{Light Workers (Community and Social Workers)}:

  \begin{itemize}
  \tightlist
  \item
    \textbf{Community Resilience}: Support community-led climate
    resilience and adaptation projects.
  \item
    \textbf{Misinformation Awareness}: Organize workshops to improve
    media literacy on climate topics.
  \end{itemize}
\item
  \textbf{Essential Workers}:

  \begin{itemize}
  \tightlist
  \item
    \textbf{Workplace Sustainability}: Advocate for sustainable
    practices in essential services sectors.
  \item
    \textbf{Critical Engagement}: Question and verify climate
    information related to their fields.
  \end{itemize}
\end{itemize}
\end{frame}

\begin{frame}{Ex.3 (Continued)}
\phantomsection\label{ex.3-continued}
\begin{itemize}
\tightlist
\item
  \textbf{Farmers}:

  \begin{itemize}
  \tightlist
  \item
    \textbf{Climate-Smart Agriculture}: Implement and share practices
    that increase resilience to climate change.
  \item
    \textbf{Local Knowledge Sharing}: Counter misinformation by sharing
    local, evidence-based agricultural successes.
  \end{itemize}
\end{itemize}

\begin{block}{General Strategies for All:}
\phantomsection\label{general-strategies-for-all}
\begin{itemize}
\tightlist
\item
  \textbf{Critical Consumption of Information}: Practice critical
  thinking and verify information through reputable sources.
\item
  \textbf{Community Dialogue}: Engage in open discussions to address
  climate change misinformation.
\item
  \textbf{Advocacy and Activism}: Support and participate in campaigns
  that advocate for truthful, science-based climate communication.
\end{itemize}
\end{block}
\end{frame}



\end{document}
