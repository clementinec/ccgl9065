% Options for packages loaded elsewhere
\PassOptionsToPackage{unicode}{hyperref}
\PassOptionsToPackage{hyphens}{url}
%
\documentclass[
  ignorenonframetext,
]{beamer}
\usepackage{pgfpages}
\setbeamertemplate{caption}[numbered]
\setbeamertemplate{caption label separator}{: }
\setbeamercolor{caption name}{fg=normal text.fg}
\beamertemplatenavigationsymbolsempty
% Prevent slide breaks in the middle of a paragraph
\widowpenalties 1 10000
\raggedbottom
\setbeamertemplate{part page}{
  \centering
  \begin{beamercolorbox}[sep=16pt,center]{part title}
    \usebeamerfont{part title}\insertpart\par
  \end{beamercolorbox}
}
\setbeamertemplate{section page}{
  \centering
  \begin{beamercolorbox}[sep=12pt,center]{part title}
    \usebeamerfont{section title}\insertsection\par
  \end{beamercolorbox}
}
\setbeamertemplate{subsection page}{
  \centering
  \begin{beamercolorbox}[sep=8pt,center]{part title}
    \usebeamerfont{subsection title}\insertsubsection\par
  \end{beamercolorbox}
}
\AtBeginPart{
  \frame{\partpage}
}
\AtBeginSection{
  \ifbibliography
  \else
    \frame{\sectionpage}
  \fi
}
\AtBeginSubsection{
  \frame{\subsectionpage}
}

\usepackage{amsmath,amssymb}
\usepackage{iftex}
\ifPDFTeX
  \usepackage[T1]{fontenc}
  \usepackage[utf8]{inputenc}
  \usepackage{textcomp} % provide euro and other symbols
\else % if luatex or xetex
  \usepackage{unicode-math}
  \defaultfontfeatures{Scale=MatchLowercase}
  \defaultfontfeatures[\rmfamily]{Ligatures=TeX,Scale=1}
\fi
\usepackage{lmodern}
\ifPDFTeX\else  
    % xetex/luatex font selection
\fi
% Use upquote if available, for straight quotes in verbatim environments
\IfFileExists{upquote.sty}{\usepackage{upquote}}{}
\IfFileExists{microtype.sty}{% use microtype if available
  \usepackage[]{microtype}
  \UseMicrotypeSet[protrusion]{basicmath} % disable protrusion for tt fonts
}{}
\makeatletter
\@ifundefined{KOMAClassName}{% if non-KOMA class
  \IfFileExists{parskip.sty}{%
    \usepackage{parskip}
  }{% else
    \setlength{\parindent}{0pt}
    \setlength{\parskip}{6pt plus 2pt minus 1pt}}
}{% if KOMA class
  \KOMAoptions{parskip=half}}
\makeatother
\usepackage{xcolor}
\newif\ifbibliography
\setlength{\emergencystretch}{3em} % prevent overfull lines
\setcounter{secnumdepth}{-\maxdimen} % remove section numbering


\providecommand{\tightlist}{%
  \setlength{\itemsep}{0pt}\setlength{\parskip}{0pt}}\usepackage{longtable,booktabs,array}
\usepackage{calc} % for calculating minipage widths
\usepackage{caption}
% Make caption package work with longtable
\makeatletter
\def\fnum@table{\tablename~\thetable}
\makeatother
\usepackage{graphicx}
\makeatletter
\def\maxwidth{\ifdim\Gin@nat@width>\linewidth\linewidth\else\Gin@nat@width\fi}
\def\maxheight{\ifdim\Gin@nat@height>\textheight\textheight\else\Gin@nat@height\fi}
\makeatother
% Scale images if necessary, so that they will not overflow the page
% margins by default, and it is still possible to overwrite the defaults
% using explicit options in \includegraphics[width, height, ...]{}
\setkeys{Gin}{width=\maxwidth,height=\maxheight,keepaspectratio}
% Set default figure placement to htbp
\makeatletter
\def\fps@figure{htbp}
\makeatother

\makeatletter
\@ifpackageloaded{caption}{}{\usepackage{caption}}
\AtBeginDocument{%
\ifdefined\contentsname
  \renewcommand*\contentsname{Table of contents}
\else
  \newcommand\contentsname{Table of contents}
\fi
\ifdefined\listfigurename
  \renewcommand*\listfigurename{List of Figures}
\else
  \newcommand\listfigurename{List of Figures}
\fi
\ifdefined\listtablename
  \renewcommand*\listtablename{List of Tables}
\else
  \newcommand\listtablename{List of Tables}
\fi
\ifdefined\figurename
  \renewcommand*\figurename{Figure}
\else
  \newcommand\figurename{Figure}
\fi
\ifdefined\tablename
  \renewcommand*\tablename{Table}
\else
  \newcommand\tablename{Table}
\fi
}
\@ifpackageloaded{float}{}{\usepackage{float}}
\floatstyle{ruled}
\@ifundefined{c@chapter}{\newfloat{codelisting}{h}{lop}}{\newfloat{codelisting}{h}{lop}[chapter]}
\floatname{codelisting}{Listing}
\newcommand*\listoflistings{\listof{codelisting}{List of Listings}}
\makeatother
\makeatletter
\makeatother
\makeatletter
\@ifpackageloaded{caption}{}{\usepackage{caption}}
\@ifpackageloaded{subcaption}{}{\usepackage{subcaption}}
\makeatother
\ifLuaTeX
  \usepackage{selnolig}  % disable illegal ligatures
\fi
\usepackage{bookmark}

\IfFileExists{xurl.sty}{\usepackage{xurl}}{} % add URL line breaks if available
\urlstyle{same} % disable monospaced font for URLs
\hypersetup{
  pdftitle={CCGL9065: Our Response to Climate Change: HK2100},
  pdfauthor={Dr.~Hongshan Guo and Class},
  hidelinks,
  pdfcreator={LaTeX via pandoc}}

\title{CCGL9065: Our Response to Climate Change: HK2100}
\subtitle{Democracy and Capitalism}
\author{Dr.~Hongshan Guo and Class}
\date{}

\begin{document}
\frame{\titlepage}

\section{This Week's Battlefield}\label{this-weeks-battlefield}

\begin{frame}{Two Sides. Two Economic Worldviews.}
\phantomsection\label{two-sides.-two-economic-worldviews.}
PRO-CLIMATE

= System Change

= ``Capitalism caused this crisis''

PRO-DEVELOPMENT

= Market Solutions

= ``Growth lifts all boats''
\end{frame}

\begin{frame}{The Core Tension}
\phantomsection\label{the-core-tension}
\begin{longtable}[]{@{}ll@{}}
\toprule\noalign{}
PRO-CLIMATE & PRO-DEVELOPMENT \\
\midrule\noalign{}
\endhead
System change needed & Reform within system \\
Degrowth / post-growth & Green growth \\
Collective ownership & Private innovation \\
Regulation \& mandates & Market incentives \\
Present suffering & Future prosperity \\
\bottomrule\noalign{}
\end{longtable}

\textbf{This tension appears in every economic climate debate.}
\end{frame}

\begin{frame}{From last: Energy as the source of GHG contributor}
\phantomsection\label{from-last-energy-as-the-source-of-ghg-contributor}
Why do we care so much about understanding the role - and the lack
thereof - of energy within the scope of understanding energy?

Pro/Cons

\begin{itemize}
\tightlist
\item
  Why is energy so important?
\item
  Why can we not make do with lesser available energy?
\item
  What is causing the transition to pause?
\item
  Why have we came to this point where most of the energy sources are
  known to have issues?
\end{itemize}
\end{frame}

\section{When was it the earliest of all times that humans started to
harvest
energy?}\label{when-was-it-the-earliest-of-all-times-that-humans-started-to-harvest-energy}

\section{CAMPFIRE}\label{campfire}

\begin{frame}{But what about other energy sources and how they came to
being?}
\phantomsection\label{but-what-about-other-energy-sources-and-how-they-came-to-being}
\begin{itemize}
\tightlist
\item
  Industrial processes
\item
  1st and 2nd Industrial Revolution
\item
  Energy \& \textbf{SURPLUS}:

  \begin{itemize}
  \tightlist
  \item
    Source
  \item
    Transportation
  \item
    End-User
  \item
    Importer (Much more complicated)
  \end{itemize}
\item
  Energy issues arise when put out at scale
\end{itemize}

{Surplus exchange fostered market and currency, an aggregated version of
which ultimately becomes capital.}
\end{frame}

\begin{frame}{{What is capital?}}
\phantomsection\label{what-is-capital}
The accumulated wealth of an individual, company or community, used as a
fund for carrying on fresh production; wealth in any form used to help
in producing more wealth.
\end{frame}

\begin{frame}{{What is Capitalism?}}
\phantomsection\label{what-is-capitalism}
``Capitalism and''Capitalist'' are 19th (middle 1800s) Centry
Pejoratives.

\begin{itemize}
\item
  enormous productive capacity.
\item
  always on the edge of being o.ut of control
\item
  attack the existing the economic system
\item
  not a highly productive system but an {\textbf{unjust stystem}}
\end{itemize}
\end{frame}

\begin{frame}{Who was this person whose photo was taken by Steichen?}
\phantomsection\label{who-was-this-person-whose-photo-was-taken-by-steichen}
\begin{itemize}
\tightlist
\item
  Does he look warm and fuzzy?
\item
  What's he holding?
\end{itemize}

\begin{figure}[H]

{\centering \includegraphics[width=4.16667in,height=\textheight]{data/john-pierpont-morgan.jpg}

}

\caption{Do we know who this is?}

\end{figure}%
\end{frame}

\begin{frame}{Capitalism is\ldots{} (per Karl Marx)}
\phantomsection\label{capitalism-is-per-karl-marx}
\begin{itemize}
\tightlist
\item
  Disease for which scientific socialism is the cure;
\item
  A method by which to steal the labor of the exploited masses;
\item
  \emph{A system that undermines every traditionally established way of
  making a living}; (known trademark of capitalism)
\item
  {A system that inevitably undermines its own foundations.}
\end{itemize}
\end{frame}

\begin{frame}{Capitalism is therefore per current definitions:}
\phantomsection\label{capitalism-is-therefore-per-current-definitions}
\begin{itemize}
\tightlist
\item
  a name given by its enemies
\item
  a system relies heaviliy on the private use of capital combined with
  profit motive founded on self/family interests
\item
  has hundreds of variations in terms of capitalism system, but the
  essentials remains unchanged:

  \begin{itemize}
  \tightlist
  \item
    private deployment of capital
  \item
    open acknowledgement of profit as a motive
  \item
    sympathetic understanding of self or household interests
  \item
    \emph{tolerance for innovation}: characteristic fact of historical
    society is resistance to it as a kind of \textbf{human nature} to
    resist fresh innovations.
  \end{itemize}
\end{itemize}
\end{frame}

\begin{frame}{Some side readings that I do recommend to take this class
(and many more in the near future.)}
\phantomsection\label{some-side-readings-that-i-do-recommend-to-take-this-class-and-many-more-in-the-near-future.}
\begin{enumerate}
\tightlist
\item
  Wealth of Nations (Smith, Bantum Paperback)
\item
  Manifesto (Marx)
\item
  The Consititution of Liberty (Hayek, market conservative and
  conservative movement)
\item
  Farewell to Alms (Gregory Clarke, capitalism more and better,
  polemical work written by economist)
\item
  The Mystery of Capital (Hernando Desoto, why less developed countries
  exist in the world: formal and enforceable property rights are
  precondition to capital devleopment)
\item
  A Failure of Capitalism (Posner, Judge, Conservative judge teaching at
  Chicago law school, no obvious ideological distortion)
\item
  The Bottom Billion (Paul Collier, world bank economist, diagnosis and
  a strategy for using capital development to alleviate most of the
  world's poverty)
\item
  The White Tiger (Fiction, describing the realworld penetrating novel
  written from the point of view of a very poor, smart and flawed young
  man in south india. Man booker prize. 2008.)
\end{enumerate}
\end{frame}

\begin{frame}{Quick Overview of Relevant Concepts Known to be tied to
climate change}
\phantomsection\label{quick-overview-of-relevant-concepts-known-to-be-tied-to-climate-change}
\begin{itemize}
\tightlist
\item
  \textbf{Capitalist Realism} (Frederic Jameson):

  \begin{itemize}
  \tightlist
  \item
    Capitalism is the only viable economic system and that
  \item
    alternatives are unimaginable.
  \end{itemize}
\item
  \textbf{Disaster Capitalism} (Naomi Klein): exploitation of natural
  disasters and crises for profit.
\item
  \textbf{Surveillance Capitalism} (Shoshana Zuboff): using technology
  to monitor and control individuals in the name of profit.
\item
  \textbf{Zero Marginal Cost Society} (Jeremy Rifkin):

  \begin{itemize}
  \tightlist
  \item
    hypothetical society
  \item
    technology and automation make goods and services virtually free,
  \item
    potentially leading to a post-scarcity economy.
  \end{itemize}
\item
  \textbf{Green New Deal}:

  \begin{itemize}
  \tightlist
  \item
    proposed economic plan
  \item
    aims to transition the economy to become carbon-neutral and
    environmentally sustainable while
  \item
    promoting social justice and economic growth.
  \end{itemize}
\end{itemize}
\end{frame}

\begin{frame}{Breaking it down a bit further with possible angles to
propose solutions/interventions:}
\phantomsection\label{breaking-it-down-a-bit-further-with-possible-angles-to-propose-solutionsinterventions}
\begin{block}{The tragedy of the commons:}
\phantomsection\label{the-tragedy-of-the-commons}
\begin{itemize}
\tightlist
\item
  Prioritize own interests over common goods
\item
  Individuals and Corporations
\item
  New economic system may help, but how?
\end{itemize}
\end{block}

\begin{block}{The limitations of carbon pricing: price GHGs
emissions(aka pay to pollute)}
\phantomsection\label{the-limitations-of-carbon-pricing-price-ghgs-emissionsaka-pay-to-pollute}
\begin{itemize}
\tightlist
\item
  usual solution to climate change
\item
  TLDR explanation: pricing pollution
\item
  might include: carbon credit (right), carbon tax (all), carbon offset
  (+/-)
\item
  government selling right to pollute?
\item
  moving price target with carbon worsening/improving?
\item
  carbon-for-trade or carbon-for-profit?
\end{itemize}
\end{block}
\end{frame}

\begin{frame}{Continued: The need for a post-growth economy:}
\phantomsection\label{continued-the-need-for-a-post-growth-economy}
\begin{itemize}
\tightlist
\item
  Continued growth mindset is unsustainable
\item
  New economic system that prioritize environmental sustainability with
  a focus on well-being, equity and sustainability over GDP growth
\end{itemize}

\begin{block}{Bhutan's Gross National Happiness}
\phantomsection\label{bhutans-gross-national-happiness}
\begin{itemize}
\tightlist
\item
  Measuring progress through Gross National Happiness Index
\item
  Emphasize well-being, environmental conervation and cultural
  preservation
\end{itemize}
\end{block}
\end{frame}

\begin{frame}{Collective ownership:}
\phantomsection\label{collective-ownership}
\begin{itemize}
\tightlist
\item
  Shift from individual ownership to collective models
\item
  Prioritize community and environmental needs
\end{itemize}

\begin{block}{Mondragon Corporation in Spain}
\phantomsection\label{mondragon-corporation-in-spain}
\begin{itemize}
\tightlist
\item
  Corporation and federation of worker cooperative based in Basque
  region of spain
\item
  One of the largest cooperatives, owned and operated by workers
\item
  Emphasize social and community benefits over individual profit
\end{itemize}
\end{block}

\begin{block}{Community Land Trusts}
\phantomsection\label{community-land-trusts}
\begin{itemize}
\tightlist
\item
  Nonprofit, community-based orgnization
\item
  Aim to propvide affordable housing by owning and managing land
  collectively (e.g.~Boston, London)
\end{itemize}
\end{block}
\end{frame}

\begin{frame}{Paradigm Shift in Economics}
\phantomsection\label{paradigm-shift-in-economics}
\begin{itemize}
\tightlist
\item
  Exploration of steady-state and circular economies
\item
  Long-term sustainability and meeting societal needs
\item
  Reevaluating values and assumptions in capitalism
\end{itemize}

\begin{block}{New Zealand's Wellbeing Budget:}
\phantomsection\label{new-zealands-wellbeing-budget}
\begin{itemize}
\tightlist
\item
  prioritize national well-being over traditional economic measures
\item
  spending focused on mental health, child poverty, and environmental
  preservation.
\end{itemize}
\end{block}

\begin{block}{Nordic Model:}
\phantomsection\label{nordic-model}
\begin{itemize}
\tightlist
\item
  Scandinavian countries (Sweden, Norway, Denmark, Finland, and Iceland)
\item
  free market capitalism + comprehensive welfare state and collective
  bargaining at national level
\item
  focusing on social equality and well-being.
\end{itemize}
\end{block}
\end{frame}

\begin{frame}{Mixed Economy Approach}
\phantomsection\label{mixed-economy-approach}
\begin{itemize}
\tightlist
\item
  Combining capitalism and socialism elements
\item
  Balancing economic growth with environmental sustainability
\end{itemize}

\begin{block}{Germany's Social Market Economy:}
\phantomsection\label{germanys-social-market-economy}
\begin{itemize}
\tightlist
\item
  A capitalist economic system + a framework of social policies
\item
  balancing free enterprise with social welfare measures and
  environmental protection.
\end{itemize}
\end{block}

\begin{block}{Singapore's Blend of Capitalism and Social Policies:}
\phantomsection\label{singapores-blend-of-capitalism-and-social-policies}
\begin{itemize}
\tightlist
\item
  competitive and open market policies + strong state intervention
\item
  seen effect in housing, healthcare, and public transport to ensure
  social welfare.
\end{itemize}
\end{block}
\end{frame}

\begin{frame}{General Public's Obvious Response}
\phantomsection\label{general-publics-obvious-response}
\begin{block}{Understanding Role in a Sustainable Future}
\phantomsection\label{understanding-role-in-a-sustainable-future}
\begin{itemize}
\tightlist
\item
  Every action counts\\
\item
  Lead by example\\
\item
  Inspire change in your community
\end{itemize}
\end{block}

\begin{block}{Possible Actions to take}
\phantomsection\label{possible-actions-to-take}
\begin{itemize}
\tightlist
\item
  \textbf{Community Participation}:

  \begin{itemize}
  \tightlist
  \item
    Join or start local sustainability initiatives\\
  \item
    Engage in community-based conservation projects
  \end{itemize}
\item
  \textbf{Educate and Spread Awareness}:

  \begin{itemize}
  \tightlist
  \item
    Share knowledge about sustainability practices\\
  \item
    Encourage friends and family to adopt eco-friendly habits
  \end{itemize}
\item
  \textbf{Practice Responsible Consumption}:

  \begin{itemize}
  \tightlist
  \item
    Support ethical and sustainable brands\\
  \item
    Reduce, reuse, recycle in daily life
  \end{itemize}
\item
  \textbf{Personal Investment in Sustainability}:

  \begin{itemize}
  \tightlist
  \item
    Use public transportation, bike, or walk when possible\\
  \item
    Grow your own food or support local, sustainable agriculture
  \item
    Invest in renewable energy and electric vehicles/efficient
    appliances
  \end{itemize}
\end{itemize}
\end{block}
\end{frame}

\section{Building Your Economic
Spectacle}\label{building-your-economic-spectacle}

\begin{frame}{The Formula (Reminder)}
\phantomsection\label{the-formula-reminder}
\textbf{Fact} + \textbf{Human Story} + \textbf{Stakes} =
\textbf{Spectacle}

Weak

``Capitalism causes emissions''

Better

``100 companies produce 71\% of global emissions''

Spectacle

``While you recycle, Shell knew about climate change in 1988 and spent
millions denying it''
\end{frame}

\begin{frame}{PRO-CLIMATE: Make It Personal}
\phantomsection\label{pro-climate-make-it-personal}
\textbf{Don't say:} ``Carbon pricing has limitations.''

\textbf{Say:} ``They want you to pay more for petrol while ExxonMobil
gets tax breaks. You're being charged for their mess.''

\textbf{Don't say:} ``We need systemic change.''

\textbf{Say:} ``Your grandfather could afford a house on one salary. You
can't afford rent on two. That's not laziness --- that's a system
extracting everything from you.''
\end{frame}

\begin{frame}{PRO-DEVELOPMENT: Paint the Picture}
\phantomsection\label{pro-development-paint-the-picture}
\textbf{Don't say:} ``Markets drive innovation.''

\textbf{Say:} ``In 2010, solar cost \$378/MWh. Today: \$36. That's not
government mandates --- that's competition. Capitalism did that.''

\textbf{Don't say:} ``We need economic growth.''

\textbf{Say:} ``My grandmother grew up without electricity in rural
China. Capitalism gave her grandchildren air conditioning, smartphones,
and choices. Don't take that away in the name of the planet.''
\end{frame}

\begin{frame}
\end{frame}

\section{\texorpdfstring{A broader debate between pro-climate change
advocates who support {\emph{dynamic market solutions}} and anti-climate
change voices who {\emph{caution against departures from traditional
economic
practices}}.}{A broader debate between pro-climate change advocates who support dynamic market solutions and anti-climate change voices who caution against departures from traditional economic practices.}}\label{a-broader-debate-between-pro-climate-change-advocates-who-support-dynamic-market-solutions-and-anti-climate-change-voices-who-caution-against-departures-from-traditional-economic-practices.}

\section{\texorpdfstring{Think Tesla ({if you have
to.})}{Think Tesla (if you have to.)}}\label{think-tesla-if-you-have-to.}

\begin{frame}{What are the possible negative impact if as a society we
switch to a new economy model that turns out to be wrong?}
\phantomsection\label{what-are-the-possible-negative-impact-if-as-a-society-we-switch-to-a-new-economy-model-that-turns-out-to-be-wrong}
\begin{itemize}
\tightlist
\item
  Economic Slowdown
\item
  Investment Displacement
\item
  Innovation Misdirection: innovation encouraged in the \emph{wrong}
  sector
\item
  Increased Cost of Living: carbon tax reflected in day-to-day expenses
\item
  Energy Insecurity
\item
  Unintended Environmental Consequences: e.g.~promoting biofuels too
  much leading to deforestation
\item
  Social and Economic Inequality: Unfair to those without means of
  transition, exacerbating inequalities
\item
  Resistance and Backlash
\end{itemize}
\end{frame}

\begin{frame}{PRO-CLIMATE Arguments for a Resilient HK Economy}
\phantomsection\label{pro-climate-arguments-for-a-resilient-hk-economy}
\begin{itemize}
\tightlist
\item
  \textbf{Innovative Market Solutions}:

  \begin{itemize}
  \tightlist
  \item
    Embrace green tech startups and innovations.
  \item
    Foster a competitive market for renewable energy.
  \end{itemize}
\item
  \textbf{Flexible Economic Models}:

  \begin{itemize}
  \tightlist
  \item
    Test sustainable business practices and circular economy models.
  \item
    Encourage investment in green infrastructure and public transport
    upgrades.
  \end{itemize}
\item
  \textbf{Public-Private Partnerships}:

  \begin{itemize}
  \tightlist
  \item
    Collaborate on sustainability projects, like smart city initiatives.
  \item
    Leverage private capital for public good, with government
    incentives.
  \end{itemize}
\item
  \textbf{Policy Reforms for Sustainability}:

  \begin{itemize}
  \tightlist
  \item
    Implement progressive environmental regulations that encourage
    innovation.
  \item
    Introduce carbon pricing to incentivize low-carbon operations.
  \end{itemize}
\item
  \textbf{Community Engagement and Education}:

  \begin{itemize}
  \tightlist
  \item
    Increase public awareness campaigns on environmental issues.
  \item
    Support community-led sustainability initiatives for broader impact.
  \end{itemize}
\end{itemize}
\end{frame}

\begin{frame}{PRO-DEVELOPMENT Arguments for HK's Economy}
\phantomsection\label{pro-development-arguments-for-hks-economy}
\begin{itemize}
\tightlist
\item
  \textbf{Economic Stability Focus}:

  \begin{itemize}
  \tightlist
  \item
    Prioritize economic growth and stability over experimental green
    policies.
  \item
    Maintain a business-friendly environment to ensure job security.
  \end{itemize}
\item
  \textbf{Proven Strategies Only}:

  \begin{itemize}
  \tightlist
  \item
    Rely on established, efficient technologies and practices.
  \item
    Avoid untested economic models that risk market disruption.
  \end{itemize}
\item
  \textbf{Minimal Regulatory Changes}:

  \begin{itemize}
  \tightlist
  \item
    Caution against over-regulation that could hinder business
    operations.
  \item
    Advocate for voluntary corporate sustainability measures.
  \end{itemize}
\item
  \textbf{Cost-Benefit Analysis}:

  \begin{itemize}
  \tightlist
  \item
    Evaluate environmental initiatives based on ROI and economic impact.
  \item
    Ensure environmental policies do not overly burden SMEs.
  \end{itemize}
\item
  \textbf{Incremental Improvements}:

  \begin{itemize}
  \tightlist
  \item
    Focus on gradual enhancements in energy efficiency and waste
    management.
  \item
    Support small-scale, low-risk green initiatives with clear benefits.
  \end{itemize}
\end{itemize}
\end{frame}

\begin{frame}
\end{frame}

\section{Activity: Economic Futures
Debate}\label{activity-economic-futures-debate}

\begin{frame}{Create Your Persona}
\phantomsection\label{create-your-persona}
\textbf{PRO-CLIMATE personas:}

\begin{itemize}
\tightlist
\item
  Labor union organizer fighting for just transition
\item
  Economist advocating degrowth
\item
  Community activist in polluted industrial zone
\item
  Youth climate striker
\end{itemize}

\textbf{PRO-DEVELOPMENT personas:}

\begin{itemize}
\tightlist
\item
  Hong Kong business owner worried about regulations
\item
  Developing-nation finance minister
\item
  Tech entrepreneur building ``green'' startups
\item
  Traditional economist focused on GDP growth
\end{itemize}

Who are you? What's your story? What do you fear losing?
\end{frame}

\begin{frame}{Remember: Fact-Check Your Stories}
\phantomsection\label{remember-fact-check-your-stories}
Every story must be \textbf{fact-checkable}.

OK to Say

\begin{itemize}
\tightlist
\item
  ``Shell knew about climate change in 1988'' \emph{(documented)}
\item
  ``Solar costs dropped 89\% since 2010'' \emph{(IEA data)}
\item
  ``Bhutan measures Gross National Happiness'' \emph{(policy fact)}
\end{itemize}

NOT OK

\begin{itemize}
\tightlist
\item
  ``Capitalism has killed millions'' \emph{(vague, unverifiable)}
\item
  ``Green policies destroy all jobs'' \emph{(exaggeration)}
\item
  ``Degrowth will solve everything'' \emph{(unfounded claim)}
\end{itemize}
\end{frame}

\begin{frame}{Human Story: The Shenzhen Factory Worker}
\phantomsection\label{human-story-the-shenzhen-factory-worker}
\textbf{Mei Li} worked 12-hour shifts making electronics for
\$400/month. Her factory was shut down for ``environmental violations.''

\textbf{PRO-CLIMATE says:} ``Finally! That factory was poisoning the
river. Workers like Mei deserve clean air.''

\textbf{PRO-DEVELOPMENT says:} ``Mei lost her job. Her family went
hungry. Now she begs on the street. Was the clean river worth it?''

\textbf{The real question:} How do we transition without leaving Mei
behind?

\emph{Both narratives are emotionally powerful. Both are incomplete.
Your job: Find the fuller story.}
\end{frame}



\end{document}
