\documentclass[preprint,12pt]{elsarticle}


%% The amssymb package provides various useful mathematical symbols
\usepackage{amssymb}
%% The amsmath package provides various useful equation environments.
\usepackage{amsmath}
\usepackage{amssymb,booktabs,gensymb}
%% The amsmath package provides various useful equation environments.
\usepackage{amsmath}
\usepackage[margin=1in]{geometry}
\usepackage[acronym]{glossaries}
\makeglossaries
%% The amsthm package provides extended theorem environments
%% \usepackage{amsthm}

%% The lineno packages adds line numbers. Start line numbering with
%% \begin{linenumbers}, end it with \end{linenumbers}. Or switch it on
%% for the whole article with \linenumbers.
%% \usepackage{lineno}

\journal{Building and Environment}

\begin{document}

\begin{frontmatter}

%% Title, authors and addresses

%% use the tnoteref command within \title for footnotes;
%% use the tnotetext command for theassociated footnote;
%% use the fnref command within \author or \affiliation for footnotes;
%% use the fntext command for theassociated footnote;
%% use the corref command within \author for corresponding author footnotes;
%% use the cortext command for theassociated footnote;
%% use the ead command for the email address,
%% and the form \ead[url] for the home page:
%% \title{Title\tnoteref{label1}}
%% \tnotetext[label1]{}
%% \author{Name\corref{cor1}\fnref{label2}}
%% \ead{email address}
%% \ead[url]{home page}
%% \fntext[label2]{}
%% \cortext[cor1]{}
%% \affiliation{organization={},
%%             addressline={},
%%             city={},
%%             postcode={},
%%             state={},
%%             country={}}
%% \fntext[label3]{}

\title{From Missing to Meaningful: A Physiology-Aware PINN-VAE for Global Thermal-Comfort Imputation and Neutral TSV Prediction} %% Article title

%% use optional labels to link authors explicitly to addresses:
\author[label1]{Hongshan Guo}
\affiliation[label1]{organization={Department of Architecture, The University of Hong Kong},
            % addressline={},
            city={Hong Kong SAR},
            % postcode={},
            % state={},
            country={China}}
%%
%% \affiliation[label2]{organization={},
%%             addressline={},
%%             city={},
%%             postcode={},
%%             state={},
%%             country={}}

\author[label2]{Chongyang Ren}%cyren@hku.hk
\affiliation[label2]{organization={Department of Urban Planning and Design, The University of Hong Kong, Hong Kong},%Department and Organization
            % addressline={}, 
            % city={},
            % postcode={}, 
            % state={},
            country={China}}

\author[label3]{Qingyao Qiao} %% qiaoqy@ms.giec.ac.cn
\affiliation[label3]{organization={Guangzhou Institute of Energy Conversion, Chinese Academy of Sciences},%Department and Organization
            % addressline={}, 
            city={},
            % postcode={}, 
            % state={},
            country={China}}

\author[label4]{Yongqiang Luo} %% yqluo@hust.edu.cn
\affiliation[label4]{organization={School of Environmental Science and Engineering, Huazhong University of Science and Technology},%Department and Organization
            % addressline={}, 
            city={},
            % postcode={}, 
            % state={},
            country={China}}            
%% Author affiliation

\begin{abstract}
Open-source thermal comfort datasets often suffer from substantial missingness and inconsistent personal physiological information, limiting their utility in occupant-centric modeling. This study introduces a physiology-informed variational autoencoder (PINN-VAE) that performs joint imputation and prediction of thermal sensation while preserving thermoregulatory realism. By combining two global datasets (ASHRAE II and China DB), we validate that missingness patterns are MAR and develop soft physiological constraints on intermediate predictions including core and skin temperature. A Personalized Physiology Interface (PPI) further improves model performance by embedding demographic inputs without normalization. Across over 150,000 records, PINN-VAE achieves 8–11\% error reduction near the thermal neutral zone compared to LightGBM, and nearly eliminates directional penalty asymmetry. Our results demonstrate that integrating physiology-informed modeling with latent-variable learning improves both accuracy and interpretability in thermal comfort prediction.
\end{abstract}


%%Graphical abstract
\begin{graphicalabstract}
%\includegraphics{grabs}
\end{graphicalabstract}

%%Research highlights
\begin{highlights} % Or however your template handles highlights
    \item PINN-VAE framework enables joint imputation and physiologically-constrained TSV prediction in large datasets. 
    \item Validates MAR patterns in combined ASHRAE II \& China DB, justifying VAE imputation approach. 
    \item Physiological constraints improve realism and performance over VAEs, yielding plausible $\hat{T}_{\text{skin}}$ / $\hat{T}_{\text{core}}$ alongside missing features imputation. 
    \item Reduces neutral zone RMSE (8-11\%) and eliminates directional penalty asymmetry versus benchmarks. % Note: Escaped % sign for LaTeX
    \item Improves interpretability via physiological outputs, offering advantages over black-box models. 
\end{highlights}

%% Keywords
\begin{keyword}
Physics-informed neural networks (PINN)\sep thermal comfort modeling \sep Occupant-centric prediction \sep missing data imputation \sep deep learning for heating, ventilation and air conditioning (HVAC) \sep interpretable neural networks
\end{keyword}

\end{frontmatter}

% Acronyms
\newacronym{pinn}{PINN}{Physics-Informed Neural Network}
\newacronym{vae}{VAE}{Variational Autoencoder}
\newacronym{tsv}{TSV}{Thermal Sensation Vote}
\newacronym{bmr}{BMR}{Basal Metabolic Rate}
\newacronym{mcar}{MCAR}{Missing Completely At Random}
\newacronym{mar}{MAR}{Missing At Random}
\newacronym{mnar}{MNAR}{Missing Not At Random}
\newacronym{pmv}{PMV}{Predicted Mean Vote}
\newacronym{ppd}{PPD}{Predicted Percentage Dissatisfied}
\newacronym{rms}{RMSE}{Root Mean Squared Error}
\newacronym{mae}{MAE}{Mean Absolute Error}
\newacronym{mape}{MAPE}{Mean Absolute Percentage Error}
\newacronym{smape}{SMAPE}{Symmetric MAPE}
\newacronym{r2}{$R^2$}{Coefficient of Determination}
\newacronym{pce}{PMV\_CE}{PMV using Classical Equation}
\newacronym{ppi}{PPI}{Personalized Physiology Interface}
\newacronym{pinnvae}{PINN-VAE}{Physics-Informed Variational Autoencoder}
\newacronym{pvp}{pvp\_pen}{PINN–VAE–PPI model}
\newacronym{cv}{CV}{Cross-Validation}
\newacronym{cnn}{CNN}{Convolutional Neural Network}
\newacronym{gnn}{GNN}{Graph Neural Network}
\newacronym{vaeplain}{vae}{Standard Variational Autoencoder model}
\newacronym{lightgbm}{LightGBM}{Light Gradient Boosting Machine}
\newacronym{ultra}{ultra}{LightGBM trained on imputed feature set}

% Terms
\newglossaryentry{tcore}{
  name={$T_\text{core}$},
  description={Core body temperature}
}

\newglossaryentry{tskin}{
  name={$T_\text{skin}$},
  description={Mean skin temperature}
}

\newglossaryentry{wettedness}{
  name={$w$},
  description={Skin wettedness (kg/h)}
}

\newglossaryentry{zlatent}{
  name={$z$},
  description={Latent representation vector in VAE}
}

\newglossaryentry{lphys}{
  name={$L_\text{phys}$},
  description={Loss function enforcing physiological constraints}
}

\newglossaryentry{lrec}{
  name={$L_\text{rec}$},
  description={Reconstruction loss in VAE}
}

\newglossaryentry{lkl}{
  name={$L_\text{KL}$},
  description={Kullback-Leibler divergence term}
}

\newglossaryentry{neutralzone}{
  name={Neutral Zone},
  description={TSV range $[-0.5, 0.5]$ considered thermally neutral}
}


The thermal load that building‐energy guidelines ascribe to occupants remains remarkably uniform: most major codes and simulation prototypes fix an office worker at 1.2\,met, equivalent to roughly 120\,W of metabolic heat as widely seen across building codes, design guidelines\citep{EN16798_2019, GB50189_2015, JISA4706_2007} and engineering references for building energy simualtion \citep{DOEPrototype2018, ECBC2017}.  This single‐value prescription persists even as the industry champions “occupant-centric” design and advanced comfort analytics.  Large post-occupancy surveys nevertheless continue to report chronic over-cooling complaints, disproportionately voiced by women and older adults \citep{Karjalainen2007, Schweiker2012, Kim2021}.  Such evidence suggests that the canonical 1.2\,met assumption may systematically overstate internal heat for sizeable portions of the population, driving lower supply-air temperatures, higher sensible and latent cooling loads, and gender-skewed discomfort.

The 1.2\,met default traces back to Fanger’s laboratory work, which converted a basal surface heat flux of 58.2\,W\,m$^{-2}$ into the unit \textit{met} and, by extension, into load tables used world-wide \citep{Fanger1970}.  Contemporary standards embed that lineage with little variation: the U.S. DOE prototype models for ASHRAE 90.1 set occupants at 120\,W; EN 16798-1 lists an office default of 118\,W; China’s GB 50189 prescribes 110\,W; Japanese and Indian codes lie in the same band. That is, even within different regulations towards the average wattage of heat generation from existing regulations, the EnergyPlus default at 120 watts per person is a generous over-estimation. Field studies have put real office workers heat generate rates to range from 45–110 W, with women disproportionately at the lower end \citep{Karjalainen2007, Kingma2015}. Prior energy-model papers varied either a single BMR equation \citep{Ahmed2017} or a local sensitivity band \citep{Chen2020}, leaving the cross-climate energy–comfort impact unquantified
% A concise comparison of these codified values is provided in Table \ref{tab:defaults} in Section 2. I don't think we need this sentence?

Physiological research, meanwhile, demonstrates a substantial variability in basal metabolic rate (BMR), and by association the resting metabolic rate, which represents the at-rest metabolic rates of average occupants.  Predictive equations such as Harris–Benedict, Cunningham, Henry, and Mifflin–St Jeor routinely yield BMR values of 60–80\,W for a large share of adult women and older adults, even after applying the conventional 10 \% lift from BMR to resting metabolic rate (RMR) \citep{HarrisBenedict1918, Cunningham1980, Henry2005, Mifflin1990}.  The gap between these empirically grounded values and the 120\,W default therefore ranges from 25 to 50 W per person—enough to bias predicted cooling loads and to justify lower operative setpoints that many occupants perceive as uncomfortably cold.

Despite decades of evidence for metabolic diversity, neither building codes nor mainstream simulation workflows have revisited the occupant heat-gain constant.  The resulting combination of inflated internal loads and one-size-fits-all comfort models risks unnecessary energy expenditure and unequal comfort outcomes.  This study addresses that overlooked lever by systematically quantifying (i) the energy penalty and (ii) the comfort inequity attributable to uniform metabolic-rate assumptions.

Two complementary update strategies are evaluated.  First, a set of composite scenarios recalculates per-person heat gains using established RMR equations scaled through Monte Carlo sampling of demographic distributions.  Second, a data-driven approach samples real occupant profiles from large thermal-comfort databases to generate stochastic metabolic-rate schedules.  Comparing these scenarios with the 120\,W baseline across a reference office building isolates the incremental cooling energy and predicted discomfort that current standards inadvertently lock in.  By quantifying these impacts, we aim to provide evidence-based guidance for revising occupant heat-gain inputs—thereby reducing avoidable cooling kilowatt-hours, carbon emissions, and persistent gendered discomfort in air-conditioned buildings. We find that replacing 120 W with a demographic-aware metabolic distribution lowers HVAC site energy by ⟨\%⟩ and halves gender-based comfort bias; Sections 2–4 detail methods, results and implications.


\subsection{Data Cleaning and Preprocessing}

The overall workflow that we followed throughout this study can be found in Figure~\ref{fig:journey-roadmap}. Our data preprocessing strategy explicitly addresses the challenge of working with large-scale thermal comfort databases that contain inevitable data quality issues while preserving the statistical power of the combined dataset.

\begin{figure}
    \centering
    \includegraphics[width=0.95\linewidth]{figures/PINN_noVAE.png}
    % \caption{Overall workflow of our Physiology-Informed Neural Network Investigation}
    \caption{Overall workflow of our Physiology-Informed Neural Network Investigation. (Note: Figure will be updated with larger text sizes for final submission.)}
    \label{fig:journey-roadmap}
\end{figure}

\paragraph{Dataset Harmonization and Quality Assessment} As illustrated in this overview, we initiated our pipeline with two distinct datasets: one from ASHRAE (103,435 entries) and another from China (41,726 entries). The ASHRAE database II originally comprised environmental measurements paired with its native metadata. However, an initial mapping challenge necessitated the standardization of this metadata. Specifically, we remapped location-specific attributes—such as city names, climate zones, and building types—to a consistent schema, thereby ensuring that each ASHRAE record was enriched with accurate geographic information. Geospatial tool packages like \texttt{geopy} were employed to verify and, when necessary, supplement the latitude and longitude data, ensuring reliable spatial identification. 

Initial analysis revealed significant data completeness disparities: while environmental variables like air temperature were well-represented ($>$95\% completeness), demographic variables showed substantial missing data (age: 78\% complete, gender: 82\% complete, height: 65\% complete, weight: 63\% complete). After harmonizing the ASHRAE database II, it was concatenated with the China dataset, which had its own complementary set of measurements and metadata. This concatenation was carefully performed by aligning records along common temporal and spatial dimensions, thereby forming a unified dataset that integrated environmental conditions and subject-related information from both sources.

\paragraph{Physics-Informed Imputation Strategy} Subsequent cleaning steps addressed data inconsistencies and missing values across the unified dataset. Rather than discarding incomplete records—which would have eliminated approximately 45\% of the combined dataset and introduced sampling bias toward studies with complete demographic collection—we implemented a physics-informed imputation strategy. Missing demographic and environmental variables, specifically age, gender, height, and weight—were imputed using defaults derived from established domain-specific standards (e.g., those recommended by Gagge). This approach ensures that imputed values remain within the operational ranges of the Gagge and JOS-3 physiological models, preventing extrapolation errors in subsequent physiological variable generation.

In the case of age, where entries included both precise numeric values and ranges, our approach was to compute a representative numerical value: ranges were replaced by their midpoint, and when expressed as a threshold (e.g., "$>60$"), the lower bound was retained. Critically, our PINN framework's physics constraints provide built-in validation for this imputation strategy. Because the neural network must satisfy energy balance equations and physiological range constraints regardless of input demographics, imputation errors that would lead to physiologically implausible combinations are automatically corrected during training.

We also performed another step upon the 49 features to create joined categorical features based off our understanding by creating additional joined categorical features such as \textit{country-season}, \textit{climate-country}, etc. Additionally, categorical variables were uniformly converted to numerical codes using label encoding to facilitate downstream deep learning applications. 

\paragraph{Validation of Internal Consistency} These preprocessing measures ensured that the final dataset was both internally consistent and robust against missing or non-standardized inputs, which is especially critical for PINNs due to its sensitivity to input data quality. Our approach provides multiple validation layers: (1) cross-checking environmental variables for physical plausibility, (2) verifying clothing insulation values against seasonal and climate data, and (3) most importantly, requiring that physiological variables generated from imputed demographics satisfy energy balance constraints during model training, providing continuous validation of internal consistency at the physiological level. The final harmonized dataset contained 145,161 entries with 20 selected features after removing variables with less than 50\% data availability.

\subsection{Physiological Variables Generation}

Physiological parameters, critical for our analysis, were not directly measured in the initial datasets; rather, they were derived computationally from the available environmental and subject data. This computational approach directly addresses a fundamental challenge in thermal comfort research: how to obtain reliable physiological estimates when direct measurements are unavailable, while ensuring the resulting data remains physiologically plausible for machine learning applications.

\paragraph{Rationale for Dual-Model Approach} We employed both Gagge's two-node model and the JOS-3 model to create a robust ensemble of physiological estimates that mitigates individual model limitations. Gagge's model provides well-validated steady-state estimates with established relationships to thermal sensation, while JOS-3 offers more detailed segmental modeling that captures individual demographic variations. Rather than viewing this as a limitation, our PINN framework leverages this dual approach as a strength: by training on estimates from both models and enforcing physics constraints, the network learns to extract consistent physiological patterns while being robust to model-specific artifacts.

\paragraph{Gagge Model Implementation and Validation} We utilized the \texttt{pythermalcomfort} library to implement Gagge's two-node function, computing key physiological outputs—such as $T_{core}$, $T_{skin}$, sensible heat flux, and additional thermal indices—based on inputs including air temperature, mean radiant temperature, air velocity, relative humidity, clothing insulation, and metabolic rate. While concerns about JOS-3's validation on nude subjects are noted, Gagge's model was specifically developed and validated for clothed occupants in building environments, making it highly appropriate for our application. The model's clothing area factor ($f_{cl}$) calculations are based on established relationships with clothing insulation values that are directly available in our datasets.

\paragraph{JOS-3 Model Integration and Uncertainty Management} In parallel, we used the JOS-3 model to simulate further aspects of human thermoregulation. The JOS-3 model integrated individual-specific parameters—namely, height, weight, age, and gender—to generate additional physiological estimates, including simulated $T_{core}$, $T_{skin}$, basal metabolic rate, and thermogenic responses. While JOS-3 requires local clothing parameters that are often unavailable in thermal comfort databases, our physics-informed approach addresses this limitation through constraint-based regularization. When local clothing insulation values ($f_{cl}$) are missing, the model uses established correlations with overall thermal insulation, and any resulting uncertainties are mitigated by the energy balance constraints in our PINN framework.

\paragraph{Model Fusion and Physics-Based Validation} The outputs from the JOS-3 simulation were merged with those obtained from the Gagge's two-node function through a systematic approach that leverages the strengths of both models. Rather than simple averaging, we treat each model's outputs as independent estimates and allow our PINN's physics constraints to determine the optimal weighting during training. This fusion approach is validated through the energy balance constraint: $Q_{balance} = (M - W) - (Q_{conv} + Q_{rad} + Q_{evap} + Q_{res}) \approx 0$, which ensures that the final physiological estimates from either model must satisfy fundamental thermodynamic principles.

\paragraph{Addressing Reconstruction Validity Concerns} The key insight is that our approach does not require perfect accuracy in individual physiological estimates because the physics constraints provide continuous validation and correction. Traditional machine learning approaches are vulnerable to propagating errors from synthetic physiological data, but our PINN framework treats these estimates as initial conditions that must be refined to satisfy known physical laws. This constraint-based validation addresses concerns about reconstruction accuracy by ensuring that the final learned relationships remain physiologically plausible regardless of input data quality.

These derived physiological features were subsequently incorporated into the final dataset, creating a rich foundation for our physics-informed deep learning framework that maintains physiological consistency through embedded thermodynamic constraints rather than relying solely on input data accuracy.

\subsection{Feature Selection (AutoML-Driven)}

After the initial data cleaning, imputation, and feature generation steps, we implemented a systematic feature selection process to enhance model performance and interpretability. This phase was driven by both domain expertise in thermal comfort research and data-driven insights obtained via automated ML (AutoML) and statistical analysis.

\paragraph{Rationale for LightGBM-Based Feature Selection} To identify the most promising modeling approach and refine our input variables, we employed \texttt{PyCaret} for AutoML. This framework enabled us to rapidly benchmark several machine learning algorithms on our dataset. Through this process, LightGBM emerged as the best-performing model among 17 regression algorithms tested, achieving superior performance in cross-validation while maintaining computational efficiency. Critically, we selected LightGBM not as our final model, but specifically for feature selection due to its superior interpretability through SHAP (SHapley Additive exPlanations) values and its robustness against overfitting compared to other high-performing algorithms like Random Forest.

LightGBM's gradient boosting framework provides several advantages for feature selection in our context: (1) it handles mixed data types (environmental, demographic, physiological) effectively, (2) its built-in regularization prevents overfitting during feature importance estimation, and (3) most importantly, its SHAP values reveal not just which features are important, but how they contribute to predictions—crucial for identifying potential data quality artifacts that our PINN constraints need to address.

\paragraph{Feature Importance Analysis and Data Quality Insights} Using the results from PyCaret, we obtained preliminary estimates of feature importance through SHAP analysis (Figure~\ref{fig:lgb-featimp}). This initial evaluation guided the subsequent steps in our feature selection process by highlighting which features contributed most significantly to thermal comfort predictions. Notably, the SHAP analysis revealed that some compound categorical features (climate zone, cooling system) ranked higher than traditional physical variables (air velocity, relative humidity), suggesting that spatial and system-level effects capture important patterns not represented by local environmental measurements alone. This finding reinforces the value of our physics-informed approach, as these categorical features may represent proxies for unmeasured physical phenomena that our energy balance constraints can help interpret.

\paragraph{Addressing Multicollinearity and Feature Redundancy} Recognizing that multicollinearity can adversely affect both the interpretability and stability of predictive models, MinMax scaling transformation was applied. This normalization ensured that all features operated on a comparable scale, which is particularly important for gradient-based optimization algorithms employed in deep learning frameworks.

Subsequently, we examined pairwise correlations among the scaled features. Features exhibiting high collinearity—those with strong linear relationships—were systematically removed or combined. This reduction of redundancy is crucial for improved model stability and mitigating potential overfitting. The correlation analysis also revealed potential data quality issues, such as features that showed unrealistic correlations, which our subsequent PINN constraints are designed to address.

\paragraph{Final Feature Set and Validation Strategy} The final feature set comprised 20 features from the original 51, including environmental and demographic variables, computed physiological indices, and geolocation data—all of which were normalized and vetted for collinearity. This curated set balances comprehensiveness with parsimony, ensuring that our subsequent deep learning model is both robust and interpretable. Importantly, the selected features include both measured environmental variables and derived physiological parameters, allowing our PINN to learn relationships between observable conditions and physiological responses while enforcing physical constraints that ensure these relationships remain plausible.

The feature selection process thus serves dual purposes: optimizing model performance while identifying data patterns that inform the design of our physics constraints, ensuring that the final PINN architecture can effectively regularize against data quality issues identified during exploratory analysis.
\subsection{PINN Model}

\subsubsection{Model Architecture}
Upon preparing the comprehensive dataset, a PINN model based on a dynamic Multilayer Perceptron (MLP) is then developed. This network integrates data-driven learning with biophysical constraints to ensure that the network's predictions are both accurate and physiologically plausible. The network is constructed as a fully connected feed-forward architecture. It processes a vector of environmental and demographic features with high contribution (e.g., air temperature, mean radiant temperature, relative humidity, air velocity, clothing insulation, metabolic rate, age, and geolocation) through 2 hidden layers composed of densely connected perceptrons with ReLU activation. The output layer simultaneously produces the primary prediction, $TSV$, and intermediate estimates of key physiological variables including $T_{core}$, $T_{skin}$, and $w$. 

\paragraph{Architecture and Hyperparameter Justification} The 2-hidden-layer MLP architecture with ReLU activation was selected based on preliminary experiments showing that deeper networks did not improve performance while increasing computational cost and overfitting risk. The simultaneous prediction of TSV and physiological variables ($T_{core}$, $T_{skin}$, $w$) as multiple outputs allows the physics constraints to directly influence the learning of thermal sensation patterns, rather than treating physiological variables as separate post-processing steps. Hidden layer sizes were determined through grid search, balancing model capacity with training stability under physics constraints.

\subsubsection{Custom Loss Function Construction}
To enforce biophysical consistency, our custom loss function combines a standard prediction error with several constraint-based terms. The total loss $L_{\text{total}}$ is defined as:

\begin{equation}
\mathcal{L_\text{total}} = \lambda_{tsv}\, L_{tsv} +  \lambda_{core}\, L_{core} + \lambda_{skin}\, L_{skin} + \lambda_{w}\, L_{w} + \lambda_{pen} \Big( P_{core} + P_{skin} + P_{w} \Big) +  \lambda_{HB} L_{HB}.
\label{eq:total_loss}
\end{equation}
where:
\begin{enumerate}
    \item \textbf{TSV Loss:} \\
    \begin{equation}
    L_{\text{tsv}} = \frac{1}{N} \sum_{i=1}^{N} \left(\widehat{TSV}_i - TSV_i\right)^2,
    \label{eq:tsv_loss}
    \end{equation}
    representing the mean squared error (MSE) between the predicted thermal sensation ($\widehat{TSV}_i$) and the observed $TSV$ ($TSV_i$).
    
    \item \textbf{Physiological Variables Loss:}
    \begin{equation}
    L_{core} = \frac{1}{N} \sum_{i=1}^{N} \left(\widehat{T}_{core,i} - T_{core,i}^{\text{(model)}}\right)^2,
    \label{eq:tcr_loss}
    \end{equation}
\begin{equation}
    L_{skin} = \frac{1}{N} \sum_{i=1}^{N} \left(\widehat{T}_{skin,i} - T_{skin,i}^{\text{(model)}}\right)^2,
    \label{eq:tskin_loss}
    \end{equation}
\begin{equation}
    L_{w} = \frac{1}{N} \sum_{i=1}^{N} \left(\widehat{w}_i - w_i^{\text{(model)}}\right)^2,
    \label{eq:w_loss}
    \end{equation}
    penalizing deviations of the predicted physiological variables $\widehat{T}_{core}$, $\widehat{T}_{skin}$ and $\widehat{w}$ from the value $T_{core}^{\text{(model)}}$, $T_{skin}^{\text{(model)}}$, $w^{\text{(model)}}$ derived from biophysical models (Gagge and JOS-3).
    \item \textbf{Range Penalties Loss:} \\
Range penalty terms that penalize predictions when:
$T_{core}^{\text{pred}} \notin [36.5, 37.5]$,
$T_{skin}^{\text{pred}} \notin [32, 36]$ and
$w_{\text{pred}} \notin [0, 0.1]$, encouraging the models to regulate the output space within the physiologically plausible ranges.

\paragraph{Clothing Area Factor ($f_{cl}$) Determination} The clothing area factor $f_{cl}$ in Equations (11) and (12) was calculated using established correlations with clothing insulation values available in our datasets, following the relationship $f_{cl}$ = 1.00 + 1.290  $\times I_{cl}$ for $I_{cl} \leq$  0.078 $m^2K/W$, and $f_{cl}$ = 1.05 + 0.645  $\times I_{cl}$ for $I_{cl} >$  0.078 $m^2K/W$. While this introduces some uncertainty for multilayer clothing scenarios, our sensitivity analysis (discussed in results) shows that the physics constraints effectively regularize against $f_{cl}$ estimation errors by ensuring overall energy balance consistency.

\paragraph{Constraint Range Flexibility} The physiological range constraints ($T_{core}$ within [36.5, 37.5], $T_{skin}$ within [32, 36]) were selected based on established literature for indoor thermal comfort scenarios. However, our framework allows for adaptive constraint ranges as demonstrated in Section 4.3.1, where we test alternative ranges including [30.5, 36.5] and [32.5, 33.5] for skin temperature. This flexibility ensures our approach can accommodate different thermal scenarios while maintaining physiological plausibility.

\begin{equation}
    P_{core} = \frac{1}{B} \sum_{i=1}^{B} \Big[ \max(0,\, 36.5 - T_{core}^{\text{pred},(i)})^2 + \max(0,\, T_{core}^{\text{pred},(i)} - 37.5)^2 \Big],
    \label{eq:p_core}
    \end{equation}
\begin{equation}
    P_{skin} = \frac{1}{B} \sum_{i=1}^{B} \Big[ \max(0,\, 32 - T_{skin}^{\text{pred},(i)})^2 + \max(0,\, T_{skin}^{\text{pred},(i)} - 35)^2 \Big] \,
    \label{eq:p_skin}
    \end{equation}
\begin{equation}
     P_{w} = \frac{1}{B} \sum_{i=1}^{B} \Big[ \max(0,\, 0 - w_{\text{pred}}^{(i)})^2 + \max(0,\, w_{\text{pred}}^{(i)} - 0.1)^2 \Big] \,
    \label{eq:p_w}
    \end{equation}

    \item \textbf{Heat Balance Loss:} \\
    The human body's heat balance in steady state is governed by the principle of energy conservation. Under equilibrium, the net heat flux is given by:
    \begin{equation}
    Q_{\text{balance}} = (M - W) - \left( Q_{\text{conv}} + Q_{\text{rad}} + Q_{\text{evap}} + Q_{\text{res}} \right),
    \label{eq:heat_balance}%Add details.
    \end{equation}
    where:
    \begin{itemize}
        \item $M$ is the metabolic rate,
        \item $W$ is external work (often negligible in sedentary conditions),
        \item $Q_{\text{conv}}$ is the convective heat loss,
        \item $Q_{\text{rad}}$ is the radiative heat loss,
        \item $Q_{\text{evap}}$ is the evaporative heat loss, and
        \item $Q_{\text{res}}$ is the respiratory heat loss.
    \end{itemize}
    At equilibrium, $Q_{\text{balance}}$ should approach zero. To enforce this, we define the heat balance loss as:
    \begin{equation}
    L_{HB} = \frac{1}{N} \sum_{i=1}^{N} \left[ Q_{\text{balance},i} \right]^2 = \frac{1}{N} \sum_{i=1}^{N} \left[ (M_i - W_i) - \left( Q_{\text{conv},i} + Q_{\text{rad},i} + Q_{\text{evap},i} + Q_{\text{res},i} \right) \right]^2.
    \label{eq:hb_loss}
    \end{equation}
    This term penalizes deviations from the energy conservation condition, effectively pushing the net heat transfer toward zero.
    \end{enumerate}

    As with how each of the heat loss terms highlighted Equation~\ref{eq:heat_balance}, we calculated these terms in accordance to Equation~\ref{eq:qconv} to ~\ref{eq:eres}:
    % Convective Heat Loss
    \begin{equation}
        Q_{conv} = f_{cl} \cdot h_c \cdot (T_{cl} - T_a)\label{eq:qconv}
    \end{equation}
    
    % Radiative Heat Loss
    \begin{equation}
        Q_{rad} = f_{cl} \cdot h_r \cdot (T_{cl} - T_r)
    \end{equation}
    
    % Evaporative Heat Loss from Skin
    \begin{equation}
        E_{sk} = w \cdot \frac{(P_{sk,s} - P_a)}{R_{e,cl} + R_{e,a}}
    \end{equation}
    
    % Respiratory Heat Loss (Sensible + Latent)
    \begin{equation}
        Q_{res} = C_{res} + E_{res}
    \end{equation}
where  $C_{res}$ and $E_{res}$ are often approximately calculated as
    \begin{equation}
        C_{res} \approx 0.0014 \cdot M \cdot (34 - T_a)
    \end{equation}
    \begin{equation}
        E_{res} \approx 1.72 \times 10^{-5} \cdot M \cdot (5867 - P_a)\label{eq:eres}
    \end{equation}

The pseudo-code shown in Algorithm\ref{alg:thermal} iterates over epochs and mini-batches, computing the combined loss, and updating the model parameters via Adam as the optimizer. To better assess how the range penalties work we tested both with the Gagge and JOS-3-calculated $T_{core}$, $T_{skin}$, and $w$. As a clear and obvious benchmark, we used the same set of data to train a best LightGBM model which was previously identified to be the best-performing model during the AutoML test. 

\begin{algorithm}[H]
\caption{PINN Model Training Process}\label{alg:thermal}
\KwIn{Dataset $D = \{(x_i, y^{(i)}_{tsv}, T_{core}^{(i)}, T_{skin}^{(i)}, w^{(i)})\}_{i=1}^N$, weights $\lambda_{tsv}$, $\lambda_{phys}$, $\lambda_{core}$, $\lambda_{skin}$, $\lambda_{w}$, penalty weight $\lambda_{pen}$, heat balance weight $\lambda_{HB}$, learning rate $\eta$, maximum epochs $E$, batch size $B$}
\KwOut{Trained model parameters $\theta$}
\textbf{Initialize} neural network parameters $\theta$ of an MLP with $L$ hidden layers and ReLU activations, such that the output is 
$$
\hat{y} = \left[ TSV_{\text{pred}},\ T_{core}^{\text{pred}},\ T_{skin}^{\text{pred}},\ w_{\text{pred}} \right] \,.
$$
\For{epoch $=1$ \textbf{to} $E$}{
    \ForEach{mini-batch $D_b \subset D$ of size $B$}{
        \textbf{Forward:} Compute predictions
        $$
        \hat{y} = \text{MLP}(x; \theta) \,.
        $$
        Split predictions into components:
        \[
        \begin{aligned}
        &TSV_{\text{pred}} \, , \quad T_{core}^{\text{pred}} \, , \quad T_{skin}^{\text{pred}} \, , \quad w_{\text{pred}} \,.
        \end{aligned}
        \]
        \textbf{Compute Loss:}
        \[
        \mathcal{L} = \lambda_{tsv}\, L_{tsv} +  \lambda_{core}\, L_{core} + \lambda_{skin}\, L_{skin} + \lambda_{w}\, L_{w} + \lambda_{pen} \Big( P_{core} + P_{skin} + P_{w} \Big) +  \lambda_{HB} L_{HB}.\\
        \]
        \begin{center}
        (see Equations~\eqref{eq:tsv_loss}--\eqref{eq:hb_loss})
        \end{center}
        
        \textbf{Backward:} Update parameters via Adam:
        \[
        \theta \leftarrow \theta - \eta\, \nabla_{\theta} \mathcal{L} \,.
        \]
    }
    \textbf{Optionally:} Evaluate model on validation set and monitor loss\;
}
\Return $\theta$\;
\end{algorithm}

Recognizing the loss terms vary by magnitude and could lead to unwarranted overfitting on spurious data or, on the contrary, aggressive regularization, different combinations of the weighting factors were tested through grid-search and cross validation, including leveraging different physiological signals as intermediate outputs driven by both Gagge and JOS-3 physiological models . This ensures that the physiological constraints are sufficiently enforced without compromising the model's ability to accurately predict $TSV$. We will also be testing range-based penalties to allow for model training to escape localized minima that may lead to over-fitted $TSV$ predictions. 
Integrating these constraints via the custom loss function serves two primary purposes:
\begin{enumerate}
    \item \textbf{Physiological Plausibility:} By penalizing deviations in $T_{core}$, $T_{skin}$, and $w$ from their model-derived estimates, the network's predictions adhere to known biophysical limits, thereby enhancing both interpretability and reliability. 
    \item \textbf{Energy Balance Enforcement:} The heat balance term $L_{HB}$ ensures that the net heat transfer remains close to zero, a condition that is fundamental for modeling steady-state thermoregulation of the occupants. This allows the analytical relationships from thermoregulation directly affects the thermal comfort sensation prediction/output.
\end{enumerate}

\subsubsection{Evaluation Metrics}
\label{subsubsec:metrics}
Traditional evaluation metrics for machine learning models, such as RMSE or MAPE, are falling short in practically evaluating the $TSV$ prediction accuracy due to the following reasons.

\paragraph{Sign-sensitivity in thermal comfort} A slight error could flip a prediction from “warm” to “cool” or vice versa, which has significant practical implications for occupant comfort and building control. However, traditional statistic metrics treat deviations uniformly regardless of whether a model’s prediction is on the correct side of a neutral thermal state. Therefore, we introduced Sign Win Rate (SWR) and Sign Loss Rate (SLR) (Equation~\eqref{eq:swr}--\eqref{eq:slr}) to address this by calculating the number of inferences when the model correctly predicts the direction of thermal sensation - whether an occupant feels warmer or cooler than neutral. 
\begin{equation}
\text{SWR} = \frac{ \left| \left\{ i \mid \text{sign}(\hat{y}_i) = \text{sign}(y_i) \right\} \right| }{N},
\label{eq:swr}
\end{equation}
\begin{equation}
\text{SLR} = \frac{ \left| \left\{ i \mid \text{sign}(\hat{y}_i) \ne \text{sign}(y_i) \right\} \right| }{N},
\label{eq:slr}
\end{equation}
\noindent
where \text{sign}(x) = -1 \text{ if } $x < 0$;\ 0 \text{ if } $x = 0$;\ 1 \text{ if } $x > 0$,  \( \hat{y}_i \) is the predicted value for the \( i \)-th sample, \( y_i \) is the corresponding ground truth value, and \( N \) is the total number of samples.

\paragraph{Combination of integers and floats of $TSV$ values} For our combined dataset, the actual breakdown of the dataset as shown in Supplementary Material Table S1 indicates that there consistently is about 10\% of all thermal sensation recorded float instead of integers, which brings further paradigm challenge. This is a total of 14, 287 records that cannot simply be thrown out, leading to the problem to be formulated as a regression rather than classification problem. That said, as the target $TSV$ is still 90\% integers, we decided to evaluate it with respect to the ranges the prediction fell into, i.e., the 6 $TSV$ intervals with a width of 1 in the range of [-3, 3]. 

Based on that, we further introduced model performance metrics: In-range Wins/Losses Rate (IWR/ILR) and Far-off Errors Rate (FER) as defined in Equation~\eqref{eq:iwr}--\eqref{eq:fer}. An in-range win occurs when the model’s prediction—after rounding appropriately—lands in the same category as the actual $TSV$. In contrast, an in-range loss is recorded when the rounded prediction falls in the adjacent intervals. Additionally, we define far-off errors as instances where the predicted value is off by at least one full category. This approach not only quantifies the overall error but also reveals how well the model captures the qualitative level of thermal sensation as experienced by occupants.
\begin{equation}
\text{IWR} = \frac{ \left| \left\{ i \mid \hat{y}_i \in I(y_i) \right\} \right| }{N}
\label{eq:iwr}
\end{equation}
\begin{equation}
\text{ILR} = \frac{ \left| \left\{ i \mid \hat{y}_i \in I(y_i \pm 1),\ \hat{y}_i \notin I(y_i) \right\} \right| }{N}
\label{eq:ilr}
\end{equation}
\begin{equation}
\text{FER} = \frac{ \left| \left\{ i \mid \hat{y}_i \notin I(y_i \pm 1),\ \hat{y}_i \notin I(y_i) \right\} \right| }{N}
\label{eq:fer}
\end{equation}
\noindent
\text{where } \( I(k) = [k, k+1) \),\quad \( y_i \in \{-3, -2, -1, 0, 1, 2\} \) and if $y_i$ is a float, it is first assigned to the corresponding interval.

In summary, due to the unique nature of $TSV$ data, apart from the traditional machine learning error metrics, we introduced sign-win/loss and range-win/loss-based metrics, which better reflect how occupants experience thermal comfort in discrete, meaningful categories. This dual-layer evaluation provides insight into both the accuracy of prediction trends (via sign win/loss) and the distribution of predictions across comfort levels (via range win/loss), leading to more actionable insights in the context of building climate control.

\section{Results and Discussions}
\subsection{Missingness MAR Validity}

The validity of applying imputation techniques like VAE hinges on the nature of the missing data. We formally tested the hypothesis that missingness in key personal variables (\texttt{height\_cm}, \texttt{age}, \texttt{gender}, \texttt{weight\_kg}) was dependent on other observed variables, thereby providing evidence against the MCAR assumption. Univariate tests (Chi-squared, t-tests) and multivariate logistic regressions were employed. The results, originally presented in detail in Table~\ref{tab:MAR_Summary}, consistently showed statistically significant associations ($p < 0.05$) between missingness and observed predictors. The logistic regression models achieved high predictability for missingness in height (AUC = 0.969), age (AUC = 0.984), and weight (AUC = 0.976). While the AUC for gender was lower, significant univariate associations remained. Table~\ref{tab:MAR_Summary} provides a condensed overview of these findings.

Based on domain knowledge of the data collection process, where data represent experimental records aggregated from various contributors, missingness often occurred systematically within specific experimental groups (defined by characteristics such as contributor, location, or building type, aligning with the statistical associations found). This typically reflected a determination by the data contributor at the time of the experiment that recording a specific variable was not required or relevant for that particular experimental condition or protocol. While the possibility that missingness could, in some cases, be related to the unobserved value itself (MNAR) cannot be entirely excluded without more granular information on the exact reason for each missing entry, the observed systematic patterns linked to recorded experimental factors lend stronger support to the MAR assumption. Therefore, we proceeded under the MAR assumption for VAE imputation, acknowledging this as a standard, necessary assumption for applying such imputation techniques in complex observational datasets\cite{Rubin1976}.

\begin{table}[htbp]
    \centering
    \caption{Simplified Summary of MAR Validation Results}
    \label{tab:MAR_Summary}
    \begin{tabular}{l l l}
        \toprule
        Variable & Key Evidence Finding & Overall Assessment (vs. MCAR) \\
        \midrule
        Height (\texttt{height\_cm}) & Multiple Sig. Assoc.; High AUC (0.969) & Evidence against MCAR \\
        Age (\texttt{age}) & Multiple Sig. Assoc.; High AUC (0.984) & Evidence against MCAR \\
        Gender (\texttt{gender}) & Multiple Sig. Assoc.; Low/Mod AUC (0.673) & Evidence against MCAR \\
        Weight (\texttt{weight\_kg}) & Multiple Sig. Assoc.; High AUC (0.976) & Evidence against MCAR \\
        \bottomrule
    \end{tabular}
    \vspace{1ex} % Add some space below the table
    \footnotesize % Smaller font for the note
    Note: Summarizes statistical tests (Chi-squared, t-tests, logistic regression AUC) assessing dependency of missingness on observed variables. Significant findings support MAR/MNAR over MCAR.
\end{table}

These results strongly suggest that the missing data mechanism is consistent with MAR or potentially Missing Not At Random (MNAR). Given the data collection context (aggregation from various experiments), systematic missingness linked to specific protocols or contributor choices aligns well with the MAR assumption. Therefore, we proceeded with VAE-based imputation under the MAR assumption, acknowledging it as a standard and necessary premise for applying such techniques to complex observational datasets \cite{Rubin1976}. Therefore, despite the potential for MNAR in some fields, the dominant missingness patterns are consistent with MAR, justifying the use of VAE-based imputation.


\subsection{BMR Ablation Results}
First and foremost, examining the aggregated out-of-sample lightgbm models performance on the BMRs (calculated only when inputs are available), we were able to calculate their respective relative RMSE, MAE and MAPE as according to Figure~\ref{fig:lightgbm-bmrs}. Across the three heat maps, the top rows are where the overall data availability is better, and towards the bottom the data availability becomes worse. The BMR models yielded consistent improvements across the board with data fill above 80\% when evaluating against RMSe and MAE, and starts to see a variation of performance, where Livingston–Kohlstadt appears to have the best performance across the three models, which is to be expected since it is a much newer calculation approach compared to the other two. It is also worth noticing that when the level of data fill falls below 50\%, all but the Livingston-Kohlstadt variant started to consistently show deterioration of performance, which is to be anticipated since the model's performance improvement now requires more inputs from BMR fields than before. %This is strangely put, revise later.
\begin{figure}[h!]
    \centering
    \includegraphics[width=0.3\linewidth]{fig/lgbrmse.png}
    \includegraphics[width=0.3\linewidth]{fig/lgbmae.png}
    \includegraphics[width=0.3\linewidth]{fig/lgbmape.png}
    \caption{Relative RMSE, MAE and MAPE of lightgbm models trained with different BMR equations}
    \label{fig:lightgbm-bmrs}
\end{figure}

% This translates to ? \% of overall RMSE improvement towards the lightgbm model simply by engineering BMR terms - which is inherently an analytical-equation-driven feature engineering exercise. Yet by working with this one variable only we believe we have demonstrated how adding additional analytically-described features that are truly transformed existing inputs can have the effect of improving the performance of thermal sensation predictor. 

The inclusion of calculated Basal Metabolic Rate (BMR) as an input feature demonstrated tangible benefits for the LightGBM benchmark model. As highlighted previously, the addition of BMR features, particularly leveraging the Livingston-Kohlstadt formulation \cite{LivingstonKohlstadt2005}, yielded an overall performance improvement of approximately 8\% in predictive accuracy (based on RMSE/MAE metrics, see Highlights) compared to the identical LightGBM model trained without BMR inputs. This underscores the value of incorporating physiology-based, analytically derived features, even when based on inputs (age, height, weight) that themselves suffer from missingness, for improving thermal sensation prediction. The Livingston-Kohlstadt variant generally provided the most consistent benefits, particularly in data subsets with higher completeness (fill > 80\%), although performance gains diminished or reversed in highly incomplete data (fill < 50\%), likely due to the unreliability of BMR inputs in those cases.


\subsection{Performance Evaluation}
\subsubsection{Thermal Sensation Prediction: Overall Performance Evaluation}
The performance of the baseline model (\texttt{pmv\_ce}), the benchmark (lightgbm), and the sequentially developed models (vae, \texttt{pv\_pen}, \texttt{pvp\_pen}, ultra) were evaluated using RMSE, MAE, MAPE, SMAPE, and $R^2$ metrics. The results are summarized in Table \ref{tab:performance_updated}.

\begin{table}[h!] % You can adjust the placement specifier [htbp] as needed
\centering
\caption{Overall Model Performance Comparison} % Add your desired caption
\label{tab:performance_updated} % Add a label for referencing
\begin{tabular}{lcccccc}
\toprule
Metric/Model & pmv\_ce & lightgbm & vae     & pv\_pen & pvp\_pen & ultra  \\
\midrule
RMSE   & 1.310   & 0.952    & 1.205   & 0.994   & 0.991    & 0.984  \\
MAE    & 1.001   & 0.711    & 0.880   & 0.745   & 0.745    & 0.739  \\
MAPE   & 101.7\% & 76.3\%   & 99.6\%  & 80.9\%  & 80.6\%   & 79.8\% \\
SMAPE  & 154.2\% & 145.8\%  & 185.3\% & 153.7\% & 152.5\%  & 149.8\%\\
$R^2$     & -0.185  & 0.376    & 0.000   & 0.320   & 0.324    & 0.334  \\
\bottomrule
\end{tabular}
\end{table}
The \texttt{pmv\_ce} baseline exhibited the poorest performance across all metrics, notably yielding a negative $R^2$ value (-0.185). The lightgbm benchmark, which handled missing data internally, achieved the lowest RMSE (0.952) and MAE (0.711), and the highest $R^2$ (0.376). The standard \gls{vaeplain} model performed poorly, with high errors (e.g., SMAPE 185.3\%) and an $R^2$ of 0.000. LightGBM's superior overall RMSE/MAE metrics might stem from its capacity to capture highly complex, non-linear interactions within the full feature set, potentially including latent patterns derived implicitly from the missing data itself. While effective for raw prediction, this contrasts with the PINN-VAE approach, where the VAE structure and explicit physiological constraints may introduce beneficial regularization and interpretability at the cost of potentially smoothing over some intricate data patterns.

Introducing physics-informed constraints (\texttt{pv\_pen}) significantly improved upon the vae, reducing RMSE from 1.205 to 0.994 and increasing $R^2$ from 0.000 to 0.320. Further incorporating personalization (\texttt{pvp\_pen}) resulted in marginal improvements over \texttt{pv\_pen}, with RMSE decreasing to 0.991 and $R^2$ increasing to 0.324. The \gls{ultra} model, using imputed data with LightGBM, performed slightly worse than the lightgbm benchmark that handled missingness implicitly (RMSE 0.984 vs 0.952; $R^2$ 0.334 vs 0.376).

The results indicate that the standard PMV-based model (\texttt{pmv\_ce}) is insufficient for accurately predicting thermal sensation in the extent of dataset this large. The LightGBM benchmark (lightgbm) demonstrated strong predictive power, achieving the best overall metrics by effectively leveraging the dataset, including missing values.

However, while lightgbm's performance is high, its internal mechanisms for handling missingness operate essentially as a "black box". The model may learn predictive patterns from the presence of missing data itself, but this does not necessarily translate into actionable or interpretable insights. For instance, superior performance relying on missingness patterns does not yield practical recommendations like "ensure this data point remains missing for better comfort prediction". This inherent limitation in interpretability regarding missing data motivates the development of models that aim for a more structured or causal understanding, even if benchmark metrics are slightly lower.

A detailed feature importance analysis (e.g., using SHAPley values or permutation importance) for the LightGBM model was not conducted as part of this study. LightGBM primarily served as a high-performance benchmark, establishing a reference point for predictive accuracy attainable by a state-of-the-art gradient boosting model that handles missing data implicitly. The focus of our analysis was on comparing the performance and characteristics of the explicitly constrained PINN-VAE approaches against this benchmark, rather than performing an in-depth feature attribution analysis of the benchmark model itself.

\begin{figure}[h!]
    \centering
    \includegraphics[width=0.5\linewidth]{fig/rmse_all_7.png}
    \caption{Alternative Model RMSEs across thermal sensation buckets}
    \label{fig:rmse-7bucket}
\end{figure}

The poor performance of the standard VAE (vae) highlights the challenges of applying generative models directly without domain constraints. Attempting to impute the missing values during the 5-fold epochs, the trained VAE ended up predicting all predictions as positive values, leading to not only a $R^2$ score at 0, but also an unusable model. In contrast, the substantial improvement observed as in Figure~\ref{fig:rmse-7bucket} with the PINN-VAE (\texttt{pv\_pen}) the significant benefit of integrating physics-informed constraints when performing parallel training of model alongside missing data imputation. This step aligns the model more closely with underlying physical principles and brought its performance much closer to the lightgbm benchmark, particularly in explained variance ($R^2$).

The addition of personal physiological indices (\texttt{pvp\_pen}) provided further, albeit smaller, performance gains, validating the hypothesis that personalization can enhance prediction accuracy within this physically-constrained framework. Comparing ultra and lightgbm, the results suggest that for this particular dataset and imputation strategy, allowing LightGBM to handle missing data internally was more advantageous than the explicit imputation method employed.

Overall, there is a potential trade-off: the lightgbm benchmark offers peak predictive accuracy by leveraging all patterns including missingness, but with limited interpretability regarding why missingness might be predictive. The PINN-VAE approaches (\texttt{pv\_pen}, \texttt{pvp\_pen}), while achieving slightly lower metrics in this instance, represent a promising pathway for integrating domain knowledge and personalization, potentially offering more interpretable and structurally sound predictions for thermal sensation.
% \begin{figure}[h!]
%     \centering
%     \includegraphics[width=0.49\linewidth]{fig/mods_rmse_hm.png}
%     \includegraphics[width=0.49\linewidth]{fig/mods_mae_hm.png}
%     \caption{RMSE and MAE of alternative models tested vs Baseline (PMV\_CE) and Benchmark (lightgbm)}
%     \label{fig:enter-label}
% \end{figure}
To beteter visualize how each model fared within each of the thermal sensation bins, we create a boxen plot across various thermal sensation bins as shown in Figure~\ref{fig:residual-boxen} to compare their respective residuals against each other.
\begin{figure}[h!]
    \centering
    \includegraphics[width=0.55\linewidth]{fig/res_boxen.png}
    \caption{Residuals from models measured}
    \label{fig:residual-boxen}
\end{figure}

Qualitatively, the distribution of residuals ocross different models can be simply compared against one another where lightgbm, despite its accuracy tends to result in a much wider range of prediction resituals, which is adequately modulated by our proposed PINN-framework across the two model alternatives. 

\subsubsection{Performance by Fill}
% In the meantime, we also evaluated the performance metrics of the models across different data availaibility `rungs', specifically the percentages of fields that are not null across all raw inputs, divided into 10\% buckets. This leads to Figure~\ref{fig:10perc_metrics}. Similar to the results reported with respect to thermal sensation bins, switching of VAE leads to RMSE and MAE increase across all levels of not-null values. Across the various percentage fill buckets reported in Figure~\ref{fig:10perc_metrics}, this translates to an average 31.5\% and 28.2\% of RMSE and MAE increase from lightgbm as the benchmark model. By incorporating the PINN framework, the RMSE and MAE increase dropped to 4.7\% and 5.1\% respectively, which ultimately dropped to 3.4\% and 3.9\% for the final lightGBM model leveraging the imputed dataset.%Expand a little...? 

Analyzing model performance across different levels of data completeness, stratified by the percentage of non-null input features per record (Figure~\ref{fig:10perc_metrics}), reveals expected trends. As illustrated by the heatmaps for RMSE and MAE, the predictive accuracy of all models generally degrades as the data fill percentage decreases. For instance, the PINN-VAE models (pv\_pen, pvp\_pen) show increased error metrics in the 40-60\% fill range compared to the >90\% fill range. This observed degradation is likely attributable to the fundamentally reduced information content available for prediction in records with higher missingness. Key variables, including the personal data required for reliable BMR calculations (\texttt{age}, \texttt{height}, \texttt{weight}) and potentially other influential environmental or contextual factors, are more frequently absent in these lower-fill data rows. This inherent lack of crucial input information naturally limits the predictive capability achievable by any modeling approach. Nonetheless, the PINN-VAE models consistently outperformed the unconstrained VAE across all fill levels and maintained performance closer to the LightGBM benchmark, particularly at higher fill percentages.


\begin{figure}[h!]
    \centering
    \includegraphics[width=0.99\linewidth]{fig/PerfMetrics_10.png}
    \caption{Root mean squared error (left), mean absolute error (middle) and mean absolute percentage error across all models evaluated}
    \label{fig:10perc_metrics}
\end{figure}
\subsubsection{Around Thermal Neutral Zone}%Does this belong to discussion...?

We had already seen that both PINN-based models were noticeably more precise than LightGBM inside the neutral zone, but a single RMSE does not show why. 
%%%%???
The results we got are as Table~\ref{tab:dir_penalty}. 

\begin{table}[t]
\centering
\caption{Direction-penalty metric  
$\Delta_{\text{dir}}=\text{MAE}_{\text{wrong sign}}-\text{MAE}_{\text{right sign}}$  
(lower is better; a positive value means the model’s error grows once it crosses to the wrong side of neutral).
Bootstrapped 95\,\% confidence intervals are based on 1\,000 resamples of the
out-of-sample rows produced by the original 5-fold CV.}
\label{tab:dir_penalty}
\begin{tabular}{lcc}
\toprule
\textbf{Model} & $\boldsymbol{\Delta_{\text{dir}}}$ & 95\,\% CI \\ 
\hline
PMV\textsubscript{CE}            & 0.1236 & [\,0.1130,\;0.1333\,] \\
LightGBM              & 0.0459 & [\,0.0386,\;0.0530\,] \\
VAE                   & $-0.1565$ & [\,-0.1657,\;-0.1470\,] \\
PINN–VAE (\texttt{pv\_pen})      & $-0.0036$ & [\,$-0.0112$,\;0.0035\,] \\
PINN–VAE–PPI (\texttt{pvp\_pen}) & $-0.0181$ & [\,$-0.0250$,\;$-0.0108$\,] \\
Ultra LightGBM (\texttt{ultra})  & 0.0342 & [\,0.0273,\;0.0409\,] \\

\bottomrule
\end{tabular}\label{tb:neutral-pen}
\end{table}

According to Table~\ref{tab:dir_penalty}, LightGBM and PMV\textsubscript{CE} incur clear asymmetric costs ($\approx$0.05 and 0.12 votes, respectively), meaning their errors inflate sharply when they cross neutral.
The physiology-guided networks behave differently: the plain PINN-VAE shows no significant asymmetry (CI straddles zero), and the PINN-VAE–PPI variant actually drives the wrong-sign error \emph{below} the right-sign error by $\approx$0.02 votes. Inspecting results from the table we noticed the performance of PINN-VAE was effective in a few interesting ways:

\paragraph{PINN variants level the playing field.}  
  With the physiology constraint—and particularly when the PPI gate is used—errors remain tightly bounded even if the network picks the wrong side of neutral.  We acknowledge the negative $\Delta$ may sound counter-intuitive, but it simply means the model’s over- and under-shoots have similar magnitudes.

\paragraph{LightGBM and PMV are asymmetric.}  
  Their right-sign mistakes are modest, but once they cross the neutral point the error grows by 0.03–0.12 votes, a pattern that matches occupant complaints about “getting the sign wrong.”  For control applications a small symmetric error is preferable: the HVAC system can compensate with a fixed offset.  Large asymmetric errors, by contrast, cause abrupt, perception-opposite responses.

To put the table into a better perspective, we also visually broke down the performance change with Figure~\ref{fig:neu_zone}. The improvement of RMSE of VAE observed in the thermal neutral zone in ~\ref{fig:rmse-7bucket} gets borken down clearly with 

\begin{figure}[h!]
    \centering
    \includegraphics[width=0.95\linewidth]{fig/w5_dir.png}
    \caption{Neutral Zone Performance Lift Analysis: Zone-specific accuracy, MAE of correct/wrong direction and their respective differences}
    \label{fig:neu_zone}
\end{figure}

This analysis using the direction-penalty statistic, $\Delta_{dir} = \text{MAE}_{\text{wrong sign}} - \text{MAE}_{\text{right sign}}$, reveals distinct behaviors near the thermal neutral zone ($|TSV| \le 0.5$). As shown in Table~\ref{tb:neutral-pen}, baseline models like $PMV_{CE}$ and LightGBM exhibit significantly positive $\Delta_{dir}$ values, indicating that their prediction errors substantially increase when the predicted sensation incorrectly crosses the neutral threshold (i.e., predicting warm when the true sensation is cool, or vice-versa). In contrast, the physiology-guided networks demonstrate markedly different behavior. The PINN-VAE (pv\_pen) shows no significant asymmetry ($\Delta_{dir} \approx -0.004$, with a 95\% CI spanning zero), while the PINN-VAE-PPI (pvp\_pen) achieves a slightly negative $\Delta_{dir}$ ($\approx -0.018$). This indicates that, for the PINN-based models, the magnitude of prediction error remains tightly bounded and symmetric, regardless of whether the prediction falls on the correct or incorrect side of neutral. This behavior, attributed to the physiological constraints preventing unrealistic predictions, is highly desirable for control applications where large, asymmetric errors can lead to inappropriate system responses\cite{Jain2018}. The PINN framework effectively removes the sharp performance cliff observed in traditional models when they misjudge the direction of thermal sensation relative to neutral.

Thus, adding the physiological loss—and, when available, the personal-profile embedding—not only improves overall RMSE but also removes the sharp performance cliff that traditional models hit when they guess the wrong side of comfort.

\subsection{Imputed Features from VAE Architecture}
A key capability of the VAE framework is data imputation. We qualitatively assessed the imputed distributions for variables with high initial missingness, namely \texttt{age}, \texttt{height}, and \texttt{weight} (Figure~\ref{fig:impu_three}), as well as the intermediate physiological variable $\hat{T}_{skin}$ (Figure~\ref{fig:skin_inputs}). Visual inspection reveals that the PINN-VAE produces more plausible distributions compared to an unconstrained VAE, avoiding unrealistic values (e.g., age < 0, height < 100 cm) due to the implicit guidance from the learned data structure and explicit physiological constraints.

While Table~\ref{tab:performance_updated} shows that LightGBM and PINN-VAE models achieve superior performance metrics, it is particularly notable that the unconstrained VAE performs the worst across all evaluation criteria—despite having access to the same data and architecture depth. This underperformance confirms that imputation alone, without physiological regularization, is insufficient for meaningful TSV prediction. The poor performance of VAE, despite its flexibility, highlights the importance of constraining intermediate outputs through biophysical priors. The improvement in PINN-VAE and further gains in PINN-VAE–PPI validate the necessity of combining imputation and physiological structure.


\begin{figure}[h!]
    \centering
    \includegraphics[width=0.5\linewidth]{fig/imp_age.png}
    \includegraphics[width=0.5\linewidth]{fig/imp_height_cm.png}
    \includegraphics[width=0.5\linewidth]{fig/imp_weight_kg.png}
    \caption{Imputed age, height and weight field values histogram between simple VAE and PINN-VAE}
    \label{fig:impu_three}
\end{figure}
Quantitative evaluation of imputation accuracy (e.g., calculating RMSE or MAE between imputed values and ground-truth values on a held-out test set) was not performed in this study. The primary variables requiring the most imputation (\texttt{age}, \texttt{height}, \texttt{weight}) originally suffered from very high levels of missing data (up to 70\%), leaving insufficient reliable ground-truth data within the dataset to conduct a robust quantitative validation of imputation performance for these specific fields. Furthermore, optimizing imputation accuracy itself, for instance through comparisons involving artificial data masking, was considered secondary to the main research objective of evaluating the overall PINN-VAE framework for constrained prediction in the presence of missing data. A dedicated study focusing on quantitatively optimizing and validating imputation strategies for these specific variables could be a valuable direction for future work.

Similarly, a quantitative comparison quantifying the frequency of physiological constraint violations in imputed intermediate variables ($\hat{T}_{core}, \hat{T}_{skin}, \hat{w}$) between the constrained PINN-VAE and an unconstrained VAE was not conducted. This is primarily due to the lack of comprehensive ground-truth physiological measurements in the original aggregated dataset; the 'input' physiological values used for comparison were themselves derived from Gagge's model \cite{gagge1986standard}. The analysis therefore focused on the qualitative demonstration, via imputed distributions (Figure~\ref{fig:impu_three}), that the physics-informed constraints effectively guide the imputation process towards generating more physiologically realistic values compared to an unconstrained approach.

As was outlined in the methodology, we employed a soft constraint on the imputation of the age, height and weight columns. Our imputed values for the three fields becomes much more reasonable in terms of coverage of values in contrast to age below 0 years old and height below 100 centimeters, and weight below 0 kg etc. Visually examining the VAE against the PINN-VAE results for these fields where null values are commonly observed, PINN-VAE does appear to be providing a much more reasonable imputation quality. %Expand further?

Beyond qualitatively comparing the results of imputed data of the combined dataset, we also compared the intermediate outputs of $T_{skin}$ as outlined in methodology to assess how PINN-VAE framework affects the resulting intermediate outputs of our proposed framework. Recognizing that $T_{skin}$ were initially calculated with analytical equations developed by Gagge et al. following their built-in assumptions, it is important to point out that its original inputs may need to be further confined. We also want to point out that, while direct quantitative validation of imputation accuracy for variables like age or height was precluded by the high degree of initial missingness, the qualitative plausibility demonstrated (Figure~\ref{fig:skin_inputs}) and the improved downstream predictive performance compared to an unconstrained VAE support the effectiveness of the constrained imputation within the overall framework's objectives.
\begin{figure}[h!]
    \centering
    \includegraphics[width=0.45\linewidth]{fig/skinimput.png}
    \caption{Imputed Skin temperature with different skin temperature penalty settings ranging from 30.5$\degree$ C to 35.5 $\degree$ C}
    \label{fig:skin_inputs}
\end{figure}


Figure~\ref{fig:skin_inputs} further illustrates the model's thermoregulatory response by showing the distribution of the imputed intermediate variable $\hat{T}_{\text{skin}}$ under progressively narrower soft constraints. The initial physiological constraint applied a range of $[30, 36]^{\circ}$C. As depicted in the subsequent panels of Figure 11, when this range was incrementally narrowed (from $[30.5, 35.5]^{\circ}$C down to $[33.0, 34.0]^{\circ}$C), the resulting distribution of imputed $\hat{T}_{\text{skin}}$ shifted and became more concentrated. This demonstrates a unique, non-linear relationship where tightening the soft constraint on skin temperature directly influences the model's imputed physiological state. While a similar analysis for $\hat{T}_{\text{core}}$ is not presented, its predicted values remained tightly clustered within physiologically plausible bounds (constrained between $[36, 38]^{\circ}$C), reflecting the body's stricter core temperature regulation. Nonetheless, the observed modulation of $\hat{T}_{\text{skin}}$ distribution based on its constraints highlights the effective physiological regulation capability embedded within the proposed PINN-VAE framework, where adjustments in one physiological parameter influence the overall predicted state.

\section{Limitations and Future Work}

\subsection{Limitations}

\paragraph{Scope of Physiological Modeling:} The current implementation relies on Gagge’s two-node model to constrain physiological variables (core and skin temperature, skin wettedness). While widely used, this model simplifies thermal regulation and may not fully capture inter-individual variability or responses under extreme thermal conditions, physical exertion, or transient exposures.

\paragraph{Soft Constraints and Lack of Hard Guarantees:} The physiological constraints are enforced via soft penalties, which guide but do not guarantee adherence to biophysical bounds. In rare cases, especially under high imputation uncertainty or data sparsity, the model may still yield borderline physiological outputs that would not be biologically plausible under strict energy balance.

\paragraph{Dataset Bias and Underrepresented Populations:} Despite combining two large-scale datasets, certain groups—such as older adults, children, or individuals in tropical and arid climates—remain underrepresented. As a result, predictions for these populations may not generalize without additional targeted data.

\paragraph{Reliance on MAR Assumption:} While statistical tests and data context support the Missing At Random (MAR) assumption, any unobserved dependencies influencing missingness (i.e., potential MNAR characteristics) could theoretically affect imputation accuracy. However, VAE-based methods are generally considered robust under MAR, which we established as the most plausible scenario.

\paragraph{Operational Feasibility for Real-Time Applications:} While model interpretability and prediction quality are improved, the added complexity of latent-variable inference and physiology-based losses increases training and inference time. This may limit near-term deployment in embedded or low-latency building control systems without further optimization.


\subsection{Future Work}
\noindent We therefore believe there are at least the following possible directions of future work for PINN-VAE and its variants for thermal sensation:
\paragraph{Joint Optimization of Comfort and Physiology:} Given that PINN-VAE outputs not only thermal sensation but also skin and core temperatures, future work could explore control algorithms that co-optimize predicted thermal comfort and physiological deviation from neutral. This opens new opportunities for occupant-aware HVAC strategies that balance subjective sensation with biophysical safety margins.

\paragraph{Integration of Wearable and Streaming Data:} Future deployments may leverage wearable sensors or IoT devices to dynamically update personal physiological inputs (e.g., real-time skin temperature or heart rate), enabling the PPI interface to adapt to intra-occupant variation or temporal changes such as illness, stress, or activity level.

\paragraph{Architectural Simplification and Model Compression:} To support deployment in building systems with limited computational capacity, model compression techniques such as pruning, quantization, or teacher-student distillation could be applied to PINN-VAE. These techniques would aim to preserve interpretability while reducing inference time and memory usage.

\paragraph{Alternative Physiological and Comfort Models:} The current architecture could be extended to accommodate other biothermal models (e.g., Stolwijk\cite{Stolwijk1971} or multi-segment models like JOS3\cite{Takahashi2002JOS}) or adaptive comfort formulations that better reflect behavioral and cultural variability. Comparative evaluation would help assess trade-offs in complexity and predictive value.

\paragraph{Field Validation in Operational Settings:} Beyond retrospective evaluation, a key next step is validating PINN-VAE in real-world building environments, comparing predictions to occupant feedback and observed control behavior. This will test not only accuracy but also acceptance and integration into control workflows.

\section{Conclusions}
This study introduced a physiology-informed neural framework (PINN-VAE) that jointly addresses two core challenges in thermal comfort modeling: (i) imputing missing values in large observational datasets and (ii) improving predictive performance while preserving physiological interpretability. By embedding soft physiological constraints into a variational autoencoder pipeline and leveraging both tabular and personal-profile inputs, our model successfully balances the flexibility of data-driven imputation with the structure of thermoregulatory realism.

We demonstrate that PINN-VAE achieves comparable or superior performance to traditional state-of-the-art models, particularly near the thermal neutral zone, where directional prediction errors are often most critical for occupant-centric applications. The inclusion of intermediate physiological outputs—core and skin temperatures—not only constrains unrealistic predictions but also offers interpretability advantages unavailable in tree-based or purely statistical models.

Furthermore, the availability of intermediate physiological outputs ($\hat{T}_{\text{core}}, \hat{T}_{\text{skin}}, \hat{w}$) offers enhanced interpretability; beyond predicting sensation, monitoring these variables could enable systems to infer underlying physiological states, potentially allowing for more proactive or health-aware environmental control strategies that anticipate discomfort or thermal stress.

Beyond thermal comfort, the PINN-VAE architecture offers a generalizable framework for combining latent-variable learning with physics-informed constraints. Its use could extend to domains such as biomechanics, energy metabolism modeling, or any setting where tabular physiological data are incomplete yet governed by known physical relationships.

The Personalized Physiology Interface (PPI) further strengthens this framework by enabling the use of raw demographic variables (e.g., age, height, gender, weight) without degrading predictive performance. In doing so, the model preserves important physiological signal diversity without needing population-wide feature averaging or scaling.

Compared to baseline LightGBM and unconstrained VAE architectures, our framework shows a measurable reduction in neutral-zone RMSE and direction-penalty asymmetry, reinforcing the hypothesis that physiology-informed modeling improves both accuracy and robustness. Overall, PINN-VAE offers a promising and interpretable path forward for occupant-aware control, particularly in systems where physiological realism and explainability are essential for deployment.



\section*{Data and Code Availability}
The code developed for the PINN-VAE model and the analysis presented in this paper is available on GitHub at [URL Currently Private] The final imputed dataset, generated using the best-performing model fold as described in Section~\ref{sec:PINN_VAE_Arch}, is available at request (currently sitting on private repository on github). The original ASHRAE Global Thermal Comfort Database II and Chinese Thermal Comfort Datasets are publicly available from their respective sources cited in the text.

\appendix
\section{Appendix}
\subsection{Acronyms and Glossary of Terms}

\printglossary[type=\acronymtype, title=Acronyms]
\printglossary[title=Glossary of Terms]

\subsection{Data Alignment Table}\label{sec:alignment}
To provide better context of how the data alignment is performed across different columns and their correspoding fields of interests, we have included the table we used to map accordingly as showon in Table~\ref{tab:table-map}.

\begin{table}[h!]
  \centering
  \resizebox{\textwidth}{!}{%
    \begin{tabular}{|l|l|l|l|}
      \hline
      \textbf{Unified Column} & \textbf{Chinese (Set 1) Column} & \textbf{ASHRAE (Set 2) Column} & \textbf{Notes} \\ \hline
      timestamp                      & date\_clean                     & timestamp            & $\surd$ \\ \hline
      contributor                    & a3.data contributor            & contributor          & $\surd$ \\ \hline
      season                         & a4.season                      & season               & Chinese has transition season. \\ \hline
      city                           & a5.city                        & city                 & $\surd$ \\ \hline
      climate\_zone                  & a6.climate zone                & climate              & Totally different categorization methods. \\ \hline
      building\_type                 & b1.building type               & building\_type       & Categories different. \\ \hline
      gender                         & c1.sex                         & gender               & $\surd$ \\ \hline
      age                            & c2.age                         & age                  & $\surd$ \\ \hline
      height\_cm                     & c3.height(cm)                & ht                   & $\surd$ \\ \hline
      weight\_kg                     & c4.weight(kg)                & wt                   & $\surd$ \\ \hline
      thermal\_sensation             & d1.tsv                         & thermal\_sensation   & $\surd$ \\ \hline
      thermal\_comfort               & d2.tcv                         & thermal\_comfort     & Different numerical evaluation criteria. \\ \hline 
      thermal\_acceptability         & d3.tav                         & thermal\_acceptability & Different numerical evaluation criteria. \\ \hline
      clothing\_insulation           & d5.clothing insulation (clo)   & clo                  & $\surd$ \\ \hline
      metabolic\_rate                & d6.metabolic rate (met)        & met                  & $\surd$ \\ \hline
      mean\_radiant\_temperature      & f2.mean radiant temperature ($\degree C$) & tr                 & Calculated MRT\\ \hline
      pmv\_ce                       & f4.pmv                         & pmv\_ce              & $\surd$ \\ \hline
      ppd\_ce                       & f5.ppd                         & ppd\_ce              & $\surd$ \\ \hline
      ta\_l                         & e1.indoor air temperature ($\degree C$)   & ta\_l                & air temperature @0.1 m \\ \hline
      ta                            & e1.indoor air temperature ($\degree C$).1 & ta                   & air temperature @0.6 m \\ \hline
      ta\_h                         & e1.indoor air temperature ($\degree C$).2 & ta\_h                & air temperature @1.1 m \\ \hline
      vel\_l                        & e3.indoor air velocity (m/s)    & vel\_l               & air velocity @0.1 m  (m/s, fpm) \\ \hline
      vel                           & e3.indoor air velocity (m/s).1  & vel                  & air velocity @0.6 m  (m/s, fpm) \\ \hline
      vel\_h                        & e3.indoor air velocity (m/s).2  & vel\_h               & air velocity @1.1 m  (m/s, fpm) \\ \hline
      rh                            & e2.indoor relative humidity (\%) & rh                 & relative humidity not available at different heights.\\ \hline
      tg\_l                        & e4.globe temperature ($\degree C$)        & tg\_l               & Globe temperature at 0.1 m \\ \hline
      tg                           & e4.globe temperature ($\degree C$).1      & tg                  & Globe temperature at 0.6 m \\ \hline
      tg\_h                        & e4.globe temperature ($\degree C$).2      & tg\_h               & Globe temperature at 1.1 m \\ \hline
      to                           & f1.operative temperature ($\degree C$)    & top                 & $\surd$ \\ \hline
      year                         & year                           & year                & $\surd$ \\ \hline
      country                      & country                        & country             & $\surd$ \\ \hline
      coolingsys                   & b4.building operation mode     & cooling\_type       & Inconsistent definition. \\ \hline
      latitude                     & latitude                       & lat                 & generated with geopy from city info.\\ \hline
      longitude                    & longitude                      & lon                 & generated with geopy from city info. \\ \hline
      ta\_out & g3.Monthly Mean Outdoor Temperature($\degree C$)    & ta\_out             & $\surd$ \\ \hline
      \end{tabular}}
  \caption{Database alignment mapping between ASHRAE II and Chinese Thermal Comfort Databases}
  \label{tab:table-map}
\end{table}

\subsection{Frozen PPI Weights}
While the Personalized Physiology Interface (PPI) header was introduced to leverage individual demographic data, its impact on overall performance (model \texttt{pvp\_pen}) yielded only marginal improvements compared to the standard PINN-VAE (\texttt{pv\_pen}), as seen in Table~\ref{tab:performance_updated} and Figure 6. Although the improvement was consistent, it was not deemed sufficiently notable within the scope of this preliminary exploration to warrant extensive discussion. Further investigation is needed to fully understand how to maximize the benefit of such personalization. Future work could explore enhancing the PPI by integrating dynamic data, such as real-time physiological measurements from wearable sensors, potentially unlocking more significant performance gains and adaptability.

%Add one more table and we're done.

Nevertheless, we were able to freeze the final PPI weights across our proposed header addition to the PINN-VAE framework, notably the PPI-ANALYTIC Model as outlined in Figure~\ref{fig:workflow}. As was discussed inside the results and discussion section of this paper, its improvement upon the PINN-VAE model was clear yet minimal. Hence this header architecture was not pursued any further beyond the following table with weights frozen.



\bibliographystyle{IEEEtran} % or another style as per journal requirements
\bibliography{cas-refs}
\end{document}
