\section{Appendix}
\subsection{Acronyms and Glossary of Terms}

\printglossary[type=\acronymtype, title=Acronyms]
\printglossary[title=Glossary of Terms]

\subsection{Data Alignment Table}\label{sec:alignment}
To provide better context of how the data alignment is performed across different columns and their correspoding fields of interests, we have included the table we used to map accordingly as showon in Table~\ref{tab:table-map}.

\begin{table}[h!]
  \centering
  \resizebox{\textwidth}{!}{%
    \begin{tabular}{|l|l|l|l|}
      \hline
      \textbf{Unified Column} & \textbf{Chinese (Set 1) Column} & \textbf{ASHRAE (Set 2) Column} & \textbf{Notes} \\ \hline
      timestamp                      & date\_clean                     & timestamp            & $\surd$ \\ \hline
      contributor                    & a3.data contributor            & contributor          & $\surd$ \\ \hline
      season                         & a4.season                      & season               & Chinese has transition season. \\ \hline
      city                           & a5.city                        & city                 & $\surd$ \\ \hline
      climate\_zone                  & a6.climate zone                & climate              & Totally different categorization methods. \\ \hline
      building\_type                 & b1.building type               & building\_type       & Categories different. \\ \hline
      gender                         & c1.sex                         & gender               & $\surd$ \\ \hline
      age                            & c2.age                         & age                  & $\surd$ \\ \hline
      height\_cm                     & c3.height(cm)                & ht                   & $\surd$ \\ \hline
      weight\_kg                     & c4.weight(kg)                & wt                   & $\surd$ \\ \hline
      thermal\_sensation             & d1.tsv                         & thermal\_sensation   & $\surd$ \\ \hline
      thermal\_comfort               & d2.tcv                         & thermal\_comfort     & Different numerical evaluation criteria. \\ \hline 
      thermal\_acceptability         & d3.tav                         & thermal\_acceptability & Different numerical evaluation criteria. \\ \hline
      clothing\_insulation           & d5.clothing insulation (clo)   & clo                  & $\surd$ \\ \hline
      metabolic\_rate                & d6.metabolic rate (met)        & met                  & $\surd$ \\ \hline
      mean\_radiant\_temperature      & f2.mean radiant temperature ($\degree C$) & tr                 & Calculated MRT\\ \hline
      pmv\_ce                       & f4.pmv                         & pmv\_ce              & $\surd$ \\ \hline
      ppd\_ce                       & f5.ppd                         & ppd\_ce              & $\surd$ \\ \hline
      ta\_l                         & e1.indoor air temperature ($\degree C$)   & ta\_l                & air temperature @0.1 m \\ \hline
      ta                            & e1.indoor air temperature ($\degree C$).1 & ta                   & air temperature @0.6 m \\ \hline
      ta\_h                         & e1.indoor air temperature ($\degree C$).2 & ta\_h                & air temperature @1.1 m \\ \hline
      vel\_l                        & e3.indoor air velocity (m/s)    & vel\_l               & air velocity @0.1 m  (m/s, fpm) \\ \hline
      vel                           & e3.indoor air velocity (m/s).1  & vel                  & air velocity @0.6 m  (m/s, fpm) \\ \hline
      vel\_h                        & e3.indoor air velocity (m/s).2  & vel\_h               & air velocity @1.1 m  (m/s, fpm) \\ \hline
      rh                            & e2.indoor relative humidity (\%) & rh                 & relative humidity not available at different heights.\\ \hline
      tg\_l                        & e4.globe temperature ($\degree C$)        & tg\_l               & Globe temperature at 0.1 m \\ \hline
      tg                           & e4.globe temperature ($\degree C$).1      & tg                  & Globe temperature at 0.6 m \\ \hline
      tg\_h                        & e4.globe temperature ($\degree C$).2      & tg\_h               & Globe temperature at 1.1 m \\ \hline
      to                           & f1.operative temperature ($\degree C$)    & top                 & $\surd$ \\ \hline
      year                         & year                           & year                & $\surd$ \\ \hline
      country                      & country                        & country             & $\surd$ \\ \hline
      coolingsys                   & b4.building operation mode     & cooling\_type       & Inconsistent definition. \\ \hline
      latitude                     & latitude                       & lat                 & generated with geopy from city info.\\ \hline
      longitude                    & longitude                      & lon                 & generated with geopy from city info. \\ \hline
      ta\_out & g3.Monthly Mean Outdoor Temperature($\degree C$)    & ta\_out             & $\surd$ \\ \hline
      \end{tabular}}
  \caption{Database alignment mapping between ASHRAE II and Chinese Thermal Comfort Databases}
  \label{tab:table-map}
\end{table}

\subsection{Frozen PPI Weights}
While the Personalized Physiology Interface (PPI) header was introduced to leverage individual demographic data, its impact on overall performance (model \texttt{pvp\_pen}) yielded only marginal improvements compared to the standard PINN-VAE (\texttt{pv\_pen}), as seen in Table~\ref{tab:performance_updated} and Figure 6. Although the improvement was consistent, it was not deemed sufficiently notable within the scope of this preliminary exploration to warrant extensive discussion. Further investigation is needed to fully understand how to maximize the benefit of such personalization. Future work could explore enhancing the PPI by integrating dynamic data, such as real-time physiological measurements from wearable sensors, potentially unlocking more significant performance gains and adaptability.

%Add one more table and we're done.

Nevertheless, we were able to freeze the final PPI weights across our proposed header addition to the PINN-VAE framework, notably the PPI-ANALYTIC Model as outlined in Figure~\ref{fig:workflow}. As was discussed inside the results and discussion section of this paper, its improvement upon the PINN-VAE model was clear yet minimal. Hence this header architecture was not pursued any further beyond the following table with weights frozen.

