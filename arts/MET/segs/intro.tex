The thermal load that building‐energy guidelines ascribe to occupants remains remarkably uniform: most major codes and simulation prototypes fix an office worker at 1.2\,met, equivalent to roughly 120\,W of metabolic heat as widely seen across building codes, design guidelines\citep{EN16798_2019, GB50189_2015, JISA4706_2007} and engineering references for building energy simualtion \citep{DOEPrototype2018, ECBC2017}.  This single‐value prescription persists even as the industry champions “occupant-centric” design and advanced comfort analytics.  Large post-occupancy surveys nevertheless continue to report chronic over-cooling complaints, disproportionately voiced by women and older adults \citep{Karjalainen2007, Schweiker2012, Kim2021}.  Such evidence suggests that the canonical 1.2\,met assumption may systematically overstate internal heat for sizeable portions of the population, driving lower supply-air temperatures, higher sensible and latent cooling loads, and gender-skewed discomfort.

The 1.2\,met default traces back to Fanger’s laboratory work, which converted a basal surface heat flux of 58.2\,W\,m$^{-2}$ into the unit \textit{met} and, by extension, into load tables used world-wide \citep{Fanger1970}.  Contemporary standards embed that lineage with little variation: the U.S. DOE prototype models for ASHRAE 90.1 set occupants at 120\,W; EN 16798-1 lists an office default of 118\,W; China’s GB 50189 prescribes 110\,W; Japanese and Indian codes lie in the same band. That is, even within different regulations towards the average wattage of heat generation from existing regulations, the EnergyPlus default at 120 watts per person is a generous over-estimation. Field studies have put real office workers heat generate rates to range from 45–110 W, with women disproportionately at the lower end \citep{Karjalainen2007, Kingma2015}. Prior energy-model papers varied either a single BMR equation \citep{Ahmed2017} or a local sensitivity band \citep{Chen2020}, leaving the cross-climate energy–comfort impact unquantified
% A concise comparison of these codified values is provided in Table \ref{tab:defaults} in Section 2. I don't think we need this sentence?

Physiological research, meanwhile, demonstrates a substantial variability in basal metabolic rate (BMR), and by association the resting metabolic rate, which represents the at-rest metabolic rates of average occupants.  Predictive equations such as Harris–Benedict, Cunningham, Henry, and Mifflin–St Jeor routinely yield BMR values of 60–80\,W for a large share of adult women and older adults, even after applying the conventional 10 \% lift from BMR to resting metabolic rate (RMR) \citep{HarrisBenedict1918, Cunningham1980, Henry2005, Mifflin1990}.  The gap between these empirically grounded values and the 120\,W default therefore ranges from 25 to 50 W per person—enough to bias predicted cooling loads and to justify lower operative setpoints that many occupants perceive as uncomfortably cold.

Despite decades of evidence for metabolic diversity, neither building codes nor mainstream simulation workflows have revisited the occupant heat-gain constant.  The resulting combination of inflated internal loads and one-size-fits-all comfort models risks unnecessary energy expenditure and unequal comfort outcomes.  This study addresses that overlooked lever by systematically quantifying (i) the energy penalty and (ii) the comfort inequity attributable to uniform metabolic-rate assumptions.

Two complementary update strategies are evaluated.  First, a set of composite scenarios recalculates per-person heat gains using established RMR equations scaled through Monte Carlo sampling of demographic distributions.  Second, a data-driven approach samples real occupant profiles from large thermal-comfort databases to generate stochastic metabolic-rate schedules.  Comparing these scenarios with the 120\,W baseline across a reference office building isolates the incremental cooling energy and predicted discomfort that current standards inadvertently lock in.  By quantifying these impacts, we aim to provide evidence-based guidance for revising occupant heat-gain inputs—thereby reducing avoidable cooling kilowatt-hours, carbon emissions, and persistent gendered discomfort in air-conditioned buildings. We find that replacing 120 W with a demographic-aware metabolic distribution lowers HVAC site energy by ⟨\%⟩ and halves gender-based comfort bias; Sections 2–4 detail methods, results and implications.
