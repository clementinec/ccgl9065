Thermal comfort modeling is critical for optimizing building design and occupant well-being, yet prevailing machine learning approaches often overlook the underlying physiological processes that drive human thermal sensation. To address this research gap, we propose a Physics-Informed Neural Network (PINN) that explicitly incorporates intermediate physiological outputs—namely, core temperature ($T_{cr}$) and skin temperature ($T_{sk}$)—into its predictive framework. By embedding biophysical constraints and energy balance principles into a custom loss function, the PINN enhances physical consistency and interpretability without sacrificing predictive performance. We benchmark the PINN against a vanilla neural network and a boosting regressor using aggregated global thermal comfort datasets. On held-out data, our best PINN variant (with Gagge-derived physiology and range-penalty constraints) achieved an RMSE of 1.115 °C, MAE of 0.843, and MAPE of 14.5\%, improving upon a vanilla neural network’s RMSE of 1.088 °C and MAPE of 20.5\%. Beyond accuracy gains, the PINN yields physiologically plausible core- and skin-temperature outputs for deeper model diagnostics. This integration of first-principles modeling with data-driven learning addresses inherent limitations in current methods, providing a robust, generalizable, and interpretable framework for thermal comfort prediction. The proposed approach has the potential to improve model reliability in diverse real-world applications, making it a promising tool for advancing the field of thermal comfort analysis.