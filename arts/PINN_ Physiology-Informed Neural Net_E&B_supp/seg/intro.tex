Modern-day sensing and control technologies have seen rapid developments over the past several decades, enabling better-than-ever capabilities for occupants to control their environments\cite{tartariniPythermalcomfortPythonPackage2020}. This has led to higher-than-ever demand from occupants to be more comfortable in their built environments. However, accurately predicting thermal comfort across diverse populations and building environments still remains a significant challenge\cite{liModifiedPredictedMean2025,wangRevisitingIndividualGroup2020c}, largely due to fundamental data quality limitations that compromise model reliability and generalizability.

Large-scale thermal comfort databases, while revolutionary for the field, suffer from inherent measurement inconsistencies, missing physiological data, and oversimplified assumptions that introduce systematic errors into predictive models. Traditional machine learning approaches\cite{luo_comparing_2020,al-sharifPredictingThermalPreferences2024,wangDimensionAnalysisSubjective2020b} treat thermal comfort prediction as a purely statistical problem, making them vulnerable to these data quality issues and prone to learning spurious correlations rather than true physiological relationships. Recent efforts to address these limitations through custom occupant profiles\cite{haghiradAdvancingPersonalThermal2024} or temporal modeling still fail to address the fundamental problem: without physiological constraints, models cannot distinguish between meaningful patterns and measurement artifacts.

To address these limitations, we propose a Physics-Informed Neural Network (PINN) framework that uses biophysical constraints as soft regularizers to overcome data quality issues while maintaining predictive performance. Our approach leverages an extensive, harmonized dataset comprising over 150,000 entries from the ASHRAE Global Thermal Comfort Database II\cite{foldvary_licina_development_2018} and Chinese thermal comfort datasets\cite{yang2023comparative}. Unlike conventional approaches, our PINN explicitly incorporates physiologically-derived inputs—core temperature ($T_{core}$), mean skin temperature ($T_{skin}$), and skin wettedness ($w$)—computed from established thermophysiological models (Gagge's two-node model\cite{gagge1971effective} and JOS-3\cite{takahashi2021thermoregulation}). By embedding these physiological principles through a custom loss function that enforces energy balance equations and physiological plausibility constraints, the network learns physically consistent relationships even when training data contains measurement artifacts or synthetic components.


Physiological signals and responses should, according to recent publications\cite{baePredictingIndoorThermal2017}, be considered in capturing individuals' responses to different thermal environment. Echoing the importance of bringing the occupants back to the loop, it is worth noticing that there has been a lot of analytical models that allows us to enhance and grow the combined ASHRAE-China dataset to demonstrate the potential of physiological states and consistency\cite{yangComparativeAnalysisIndoor2024}. Noting that there is, nonetheless, no explicit measurements of physiological conditions measured in either of these datasets, we want to address this challenge by proposing a Physiology-Informed Neural Network architecture that leverages known physiological states/boundaries and heat balance equations in predicting thermal comfort.

Our methodology involves first generating detailed physiological parameters for each entry in the comprehensive dataset using Gagge and JOS-3 models. These parameters, combined with environmental and contextual variables such as geographic coordinates and Köppen climate classifications, are then utilized as enriched input features for our deep learning models. Comparative analyses clearly demonstrate that our PINN approach achieves superior accuracy over traditional models, including $PMV_{CE}$ and stand-alone physiological models, highlighting its capability to capture nuanced occupant thermal comfort preferences.

We believe our solution not only incorporate physiological relationships and ensures physical consistency in predicting thermal comfort, but also heavily enhances the interpretability in contrast to traditional PINN. Especially when the predictions of physiological states can be validated against known biophysical behavior, which is something standard black-box models rarely offer. We also want to highlight that these physical constraints also will act as a form of regularization, reducing the risk of overfitting, effectively `anchor' our prediction to prevent it from learning spurious correlations that don't make physical sense, or poor data quality due to differences in experimental assumptions. The physiological realism we introduced into predictive modeling of thermal comfort allowed us to build a robust, scalable solution that bridges the gap between generalized comfort models and personalized occupant experiences. Future directions for this research include integrating real-time adaptive mechanisms and physiological sensing data to further enhance model responsiveness and applicability.