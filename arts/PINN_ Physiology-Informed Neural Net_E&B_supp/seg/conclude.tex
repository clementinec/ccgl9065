We present a Physiology-Informed Neural Network (PINN) framework in this paper that combines large-scale thermal comfort datasets with embedded physiological constraints to improve the accuracy and interpretability of thermal sensation prediction. By harmonizing the ASHRAE Global Thermal Comfort Database II and the Chinese Thermal Comfort Dataset, we construct a unified dataset enriched with derived physiological signals, addressing longstanding limitations in data heterogeneity and representational consistency.

The integration of physiological variables—$T_{core}$, $T_{skin}$ and $w$—and heat balance equations into the model's loss function enables predictions that are not only statistically robust but also physiologically plausible. Compared to standard models such as LightGBM or vanilla NN, the PINNs maintain competitive performance with the best-performed model variant achieving MAPE improvement of 29.27\% over vanilla NN. More importantly, PINNs offer an additional advantage of producing intermediate outputs grounded in biophysical theory. These outputs enhance the model’s transparency and open new opportunities for diagnostics, interpretability, and control relevance in thermal comfort modeling.

Importantly, our evaluation moves beyond conventional scalar metrics to assess performance in terms of categorical alignment with perceived comfort—capturing how well predictions reflect discrete thermal sensation levels and their practical significance. This offers a more occupant-aligned framework for assessing predictive models, particularly for applications where system response depends on directional correctness. Result-wise, the PINNs also demonstrated significant improvement over baseline in term of in-range-win-based metrics, with average IWR of 6 variants of 78.06\% compared to 58.9\% of baseline in the $TSV$ range of [-1,1].

The findings suggest that embedding physiological realism into data-driven frameworks may be essential for developing generalizable and trustworthy thermal comfort models, especially under noisy, sparse, or extrapolated conditions. The resulting architecture offers a promising foundation for advancing occupant-centric environmental control systems, where physiological plausibility can serve as both a constraint and a source of insight. Future work may build on this direction by incorporating real-time sensing, transient physiological data, and localized adaptation strategies to further refine comfort prediction and response.

Beyond improving predictive accuracy, the proposed PINN framework opens new possibilities for collaboration across disciplines. Researchers in machine learning, building simulation, and physiological sensing may find this architecture a useful foundation for integrating data-driven prediction with biophysical insight, particularly in developing adaptive comfort systems and personalized environmental control strategies.