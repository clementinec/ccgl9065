\subsection{Quality of Combined Dataset}
% Quantitative metrics (e.g., statistical summaries, visualizations of data distribution)
% Detailed discussion on the impact of data imputation on model performance

A machine learning model can only be as good as its underlying data. Yet all efforts involving big data ultimately comes back down to how effective the data cleaning pipeline was to create a robust and reusable model. We have dedicated extensive effort in this study to harness as much predictive power towards $TSV$ as we can, leveraging the existing datasets from both databases, yet still ran into a few major issues in reconciling these datasets that remains challenges unaddressed despite our best efforts. It is worth noting, that the scale actually used is the ASHRAE 7-point scale rather than the PMV scale, as noted by Schweiker et al.\cite{schweiker2017review}.

When we processed the data obtained in the two datasets, we encountered an issue of data quantity disparity. As an example of highly abstracted value, mean radiant temperature (MRT) are typically calculated from the air temperature (Ta), air velocity (Va) and global temperature (Tg) according to ISO 7726 (1998). However, the results of data analysis show that the combined dataset contained 74.84\% MRT records, but only 41.78\% Tg records. This raises a critical question: Given that MRT calculations depend on Tg, how were a significantly higher number of MRT values obtained compared to Tg? Furthermore, what was the source of the surplus MRT data, and what is their reliability?

After examining MRT raw data, we found that researchers rely on oversimplified assumptions rather than precise measurements or calculations, leading to MRT getting simplified in many wrong ways. One of the most used assumptions to simplify MRT is to equate MRT directly with air temperature (Ta) \cite{mouFieldStudyThermal2022,heidariComparativeAnalysisShortterm2002,schiavonDynamicPredictiveClothing2013}. This simplistic assumption neglects the radiative effects on thermal sensation, resulting in large deviation and inaccuracies in thermal assessments. Research indicates that assuming MRT equals Ta introduces an error of 7.793\% \cite{ozbeyComprehensiveComparisonAccuracy2022}. Moreover, MRT and Ta frequently diverge in sun-exposed spaces \cite{walikewitzDifferenceMeanRadiant2015}. And the relationship between MRT and Ta also varies by cooling system: in radiant cooling, MRT tends to be slightly cooler than Ta, while warmer in all-air cooling  \cite{daweComparisonMeanRadiant2020}. Most critically, Chaudhuri et al. \cite{chaudhuriAssumingMeanRadiant2016} demonstrated that even minor MRT-Ta discrepancies can distort $PMV$ calculations and thermal comfort evaluations, with larger differences producing proportionally greater inaccuracies. While some alternative assumptions are derived from statistical analyses, empirical research has demonstrated that these approaches exhibit significant inaccuracies \cite{ozbeyComprehensiveComparisonAccuracy2022}.

These oversimplified MRT values leads to inaccuracies on ML models \cite{luo_comparing_2020}. Inconsistencies in MRT calculation methods across studies contribute to dataset fragmentation, hindering the learning of robust patterns. The absence of standardized MRT measurements not only compromises model accuracy but also confounds interpretability in feature importance analyses, which may reflect data artifacts rather than true physiological relationships. With inherent regulating effect from physiology, PINNs can mitigate errors caused by MRT simplification and improve performance when combined with PMV.

\subsection{Implication of PINN Findings}

The integration of physiological constraints within the PINN architecture introduces not only interpretability but also a stabilizing effect on model training, particularly under conditions of data sparsity or uncertainty. By enforcing learned outputs to remain within established physiological bounds for $T_{core}$, $T_{skin}$ and $w$, the model resists overfitting to outlier patterns and noisy entries that are prevalent in combined large-scale datasets.

Notably, the comparative analysis between Gagge’s two-node model and the JOS-3 thermoregulation model revealed systematic differences in their estimated physiological signals. While both models are grounded in biophysical principles, the Gagge model yielded more centralized predictions for $T_{core}$ and $T_{skin}$, clustering tightly around canonical resting values (e.g., $T_{core}$ $\approx$ 36.8 \textdegree{}C), whereas JOS-3 produced a wider spread, especially in the estimation of $T_{skin}$. This divergence may suggest that the simpler structure of the Gagge model may function more effectively as a regularizing prior in noisy datasets, while JOS-3’s more detailed segmentation and responsiveness to individual demographic variables—particularly age, gender, and body mass—leading to greater variability across occupant profiles, thus requires more careful tuning to avoid propagating variance.

Moreover, constraining $T_{skin}$ within empirically grounded ranges had an observable secondary effect on the distribution of $T_{core}$ predictions, effectively narrowing their spread even without direct penalties applied—demonstrating a latent coupling between surface and core thermoregulatory states captured within the network. Also, the models with more strict constraints learned more physically plausible correlation between $TSV$ and intermediate outputs. These findings confirm that the learned network does not merely regress physiological variables independently, but internally adapts to biophysical coherence enforced through the loss function.

Taken together, the PINN’s performance—particularly when trained with range penalties and Gagge-derived targets—demonstrates that physiology-informed constraints act not only as soft bounds but as functional regularizers that guide learning towards generalizable patterns. This improves robustness under imbalanced or uncertain data, and enhances the reliability of intermediate outputs critical for downstream control logic or occupant-facing diagnostics. As such, this framework heavily stresses the utility of embedding first-principles physiological reasoning into the structure of predictive models for thermal comfort.

\subsection{PINN vs. Classic Machine-Learning Regressor}

Comparing the performance between the PINN and the boosting regressor, even on purely out-of-sample held out set that we engineered, we believe it is important to address what on surface value appears to be a deterioration of accuracy. Like other PINN models, our proposed architecture is allows the training of a deep neural network model to be trained while guided by real-world constraints through embedding physical laws. This will help it to generalize better to unseen conditions, which is especially valuable when we notice the noisy training data that we have at hand.

Our architecture also allows for the generation and storing of the intermediate outputs (e.g. $T_{skin}, T_{core}, w$ that fall within realistic physiological ranges and are confined to heat balance equations. This consistency with known physiological conditions is critical, especially in an environment that our last line of defense is general acceptability from the occupants. These intermediate variables also provides valuable insights into the model's decision-making process. Although the final prediction isn't dramatically more accurate, the ability to verify the network is indeed learning from plausible physiology can be very valuable for building trust in the model. This not only allows us to better relate the thermal comfort votes to the occupants' physiology, but also provide unique diagnostic value by examining how sensitive PINN is to specific input features, and whether changes in these features lead to physiological meaningful adjustments in the outputs. This could mean allowing for physiology-informed building control that not only optimizes for occupants' thermal comfort but also a specific range of acceptable $T_{core}$ or $T_{skin}$ values.

Moreover, the integration of physics-based constraints acts as an effective regularizer, reducing overfitting and improving the model’s robustness in the face of noisy or imbalanced data. This inherent constraint facilitates better generalization, particularly in extrapolation scenarios where data are sparse or perturbations occur. This is particularly helpful for leveraging the ASHRAE and Chinese datasets since many of the data entries that includes significant perturbations and extrapolations. The PINN, in this case, can generalize better to conditions that are not well represented in the training data since the equations and constraints inside the custom loss function with the built-in physics and physiological constraints. These constraints also will act as a regularizer, which potentially will reduce the overfitting to what we've established the combined data to be noisy and build in extra trust to the model in contrast to a boosting model which has lower computational cost and higher accuracy.

\subsection{Limitation and Future Work}
While the proposed PINN framework demonstrates notable improvements in physiological consistency and predictive robustness, several limitations remain that suggest clear avenues for further research.

First, the absence of direct physiological measurements in both the ASHRAE and Chinese datasets leads to total reliance on model-derived estimates from physiological models like Gagge and JOS-3. While these models provide reasonable approximations, their outputs inevitably reflect simplifying assumptions that may not generalize across all demographic or activity contexts. Future work could benefit from integrating empirical physiological datasets—such as those involving wearable sensors capturing skin temperature, heart rate, or sweat rate—collected in tandem with subjective thermal comfort votes. This would enable the PINN to not only validate its internal predictions but also adapt to time-varying and individualized thermoregulatory profiles, extending beyond the current steady-state modeling paradigm.

Second, the formulation of the heat balance constraint in this study followed standard expressions grounded in ASHRAE 55. While useful as a starting point, the generalization of heat exchange processes across different clothing ensembles, metabolic states, or ventilation configurations may warrant alternative formulations. Incorporating more advanced thermoregulation models or modifying boundary conditions within the heat balance term could improve constraint accuracy and model fidelity, especially under edge-case environmental scenarios.

Third, the current classification of thermal comfort responses remains coarse, implicitly relying on the seven-point $PMV$ scale and evaluating models through integer binning and range-based diagnostics. This limits the granularity of discomfort identification and may obscure more subtle patterns in occupant feedback. A potential future direction involves exploring semi-supervised or representation learning techniques to better characterize latent structure in the combined dataset. Such approaches could assist in identifying mislabeled or low-quality entries, guiding soft classification schemes that reflect probabilistic or fuzzy comfort boundaries rather than rigid class assignments.

Moreover, while the custom loss function improves generalizability, the balancing of multiple penalty terms introduces additional hyperparameter complexity. Albeit results from the current study appears to be justifying equal weights on all penalty terms, additional studies upon them may yield interesting results. Although grid search was used in this study, more scalable optimization strategies such as Bayesian tuning or gradient-based weighting adjustment could streamline training and improve convergence stability across datasets.

Finally, the PINN architecture assumes a static representation of the occupant-environment interaction, omitting temporal dynamics such as acclimatization, behavioral adaptation, or control feedback. Extending this framework to accommodate sequential or time-series modeling—potentially via physics-informed recurrent networks or attention-based architectures—may offer richer insight into transient comfort trajectories and system response.

Altogether, these limitations help us in highlighting the importance of ongoing development in both model expressiveness and data quality. As environmental control moves toward more individualized, adaptive, and physiologically-grounded thermal comfort modeling, frameworks like the one proposed here may serve as foundational scaffolds for integrating future sensing modalities, adaptive controls, and real-time predictive systems.

Through these findings, we are seeing the need for a paradigm shift in thermal comfort research—from black-box predictive accuracy to biophysically grounded modeling. As occupant-centric control strategies and wearable physiological sensing become more commonplace, leveraging frameworks such as PINN may become increasingly central to generalizable, interpretable, and equitable comfort modeling. These directions may particularly benefit researchers working at the intersection of physiological modeling, environmental control, and real-time sensing, where interpretability and constraint-informed learning are increasingly valuable.