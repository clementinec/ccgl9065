Under such premise, it is of great importance that we identify an appropriate proxy of air freshness alongside the most common iAQ parameters such as air temperature, relative humidity, and PM2.5 values. After examining past research recognizing carbon dioxide's importance for iAQ, we found that the concentration of carbon dioxide in the air is a good proxy for air freshness. This is because carbon dioxide is a byproduct of human respiration and is exhaled into the air. Therefore, the concentration of carbon dioxide in the air is a good indicator of the presence of humans in the room. Therefore, the concentration of carbon dioxide in the air is also a good indicator of the room's ventilation rate under the impact of existing occupants. This proposition was initially published in the 1980s to surrogate human bio-effluents and was subsequently introduced into ASHRAE Standards in 1981 \cite{ashrae1981}. Fanger et al. further supported this by pointing out that higher steady-state CO2 correlates to increased perceived stuffiness, highlighting that 1000 PPM can be considered an indoor air quality guideline. The importance of CO2 was subsequently celebrated as a single variable of indoor air quality but was later debunked by OSTI, stating that CO2 "does not provide a comprehensive indication of indoor air quality," a statement that ASHRAE supported further in 2019 and 2022 \cite{ashrae2022}. The measurement of CO2 was therefore primarily confined to the realm of demand-control ventilation, as engineers monitor its concentration to estimate actual occupancy and adjust ventilation rates accordingly. The COVID-19 pandemic changed the scope of this problem, as researchers collectively recognized CO2's strength, stating that "indoor CO2 measurements can be a useful tool for understanding building ventilation and iAQ" \cite{ashrae2022}. As a result, guidelines on setting the recommended indoor threshold for CO2 have been updated to 800 PPM in certain studies (SEETHEAIR).

