\section{Introduction}
Balancing occupant thermal comfort and energy efficiency in buildings remains a central challenge in building operation, as HVAC systems account for a significant portion of total building energy use and are responsible for over 40\% of energy consumption in commercial buildings \cite{Crawley2001}.  Traditional HVAC control strategies rely on fixed set‐point schedules or the Predicted Mean Vote (PMV) analytical model, which uses heat‐balance equations standardized in ISO 7730 to estimate thermal sensation \cite{Fanger1970}.  However, PMV cannot dynamically adapt to real occupant feedback, clothing changes, or activity levels, often leading to suboptimal comfort and increased energy use \cite{Schaudienst2017}. In response to these challenges, this study proposes an innovative approach—directly coupling data-driven thermal sensation predictions with energy simulation through a robust co-simulation framework. Unlike previous methods, which primarily rely on comfort proxies or indirect integration, our approach directly leverages occupant feedback from a large-scale database (148,148 occupant votes) to inform dynamic control.

Recent developments leverage large‐scale field data to create data‐driven and adaptive comfort models rooted in actual occupant votes.  The ASHRAE Global Thermal Comfort Database II, comprising over 109,000 human subject records from 30 countries, has enabled novel predictive comfort models that outperform PMV in real‐world scenarios \cite{Foldvary2018}.  Adaptive standards such as ASHRAE 55 and EN 15251 further incorporate environmental and personal factors, yet still lack integration with building energy simulations \cite{deDear2018}.  

Concurrently, machine learning techniques have proven effective for both energy consumption prediction and thermal sensation estimation.  Gradient Boosting methods like LightGBM offer fast, accurate energy‐use forecasts suitable for real‐time control \cite{Ke2017}, while recent physics‐informed neural network–variational autoencoder (PINN‐VAE) architectures leverage physical laws and latent representations to predict thermal sensation with high fidelity \cite{Chen2023}.  Interpretable ML frameworks have been reviewed in several studies, highlighting their potential to enhance both comfort and efficiency in building operations \cite{turn1search6}.  Simulation platforms such as Sinergym enable stochastic weather perturbations and RL‐based control development by interfacing Python agents with EnergyPlus \cite{perarnau2021sinergym, campoy2025sinergym}.  

Despite these advances, existing co‐simulation frameworks (e.g., Python EMS \cite{turn1search8}, FMU‐based coupling \cite{turn1search9}) rarely integrate ML-predicted comfort metrics directly into EnergyPlus control loops, limiting validation to proxies rather than actual occupant thermal sensation.  As a result, there remains a gap in leveraging occupant‐centric ML models for set‐point optimization within building energy simulations.  

To explicitly address this gap and clearly advance beyond existing state-of-the-art, this paper makes the following novel contributions: 
\begin{enumerate}
  \item We introduce a \textbf{novel co‐simulation framework}, combining PINN‐VAE (thermal sensation) and LightGBM (energy use) models explicitly grounded in the extensive ASHRAE Global Thermal Comfort Database II, enabling dynamic occupant-driven HVAC control.
  
  \item We propose an innovative, ML‐inspired \textbf{grid‐search control algorithm} that minimally adjusts HVAC set‐points based on real‐time predicted TSV values, applicable robustly to both classical PMV and advanced ML‐based predictors.
  
  \item We present the \textbf{unexpected and significant finding} that simple PMV‐based controllers, when systematically optimized via grid‐search, achieve energy savings comparable to sophisticated ML‐based controllers, challenging the common assumption that more complex models always yield superior outcomes.
  
  \item We validate of these approaches across four representative the U.S. Department of Energy (DOE) climate zones, demonstrating an unexpected convergence in performance between optimized classical and ML‐based controllers, thus offering a new and practical pathway for efficient and interpretable building control implementation.
\end{enumerate}


% Our findings challenge the assumption that sophisticated comfort prediction necessarily yields superior control performance, demonstrating that properly optimized classical methods can match state-of-the-art ML models while remaining interpretable and computationally efficient. The remainder of this paper is organized as follows.  Section \ref{sec:related_work} reviews relevant literature; Section \ref{sec:methodology} details our co‐simulation and control algorithms; Section \ref{sec:experimental_design} describes the experimental setup; Section \ref{sec:results} presents and discusses the findings; and Section \ref{sec:conclusions} concludes with implications and future work.

These results explicitly challenge the conventional assumption that sophisticated machine learning-driven comfort models inherently deliver superior performance. Instead, we demonstrate that carefully optimized classical models, like PMV, can deliver state-of-the-art energy efficiency and occupant comfort, fundamentally shifting the focus of future research towards practical control integration rather than merely predictive complexity.