\section{Discussions}
This study's primary contribution is a validated methodology for integrating data-driven comfort models with building control systems, enabling previously impossible comparisons between comfort prediction approaches under realistic deployment conditions. By establishing direct coupling between thermal sensation models and EnergyPlus control loops, our framework reveals that optimized classical PMV achieves comparable performance to sophisticated ML models—a counterintuitive finding that challenges prevailing assumptions about the value of model complexity in building control.

While these initial findings provide compelling evidence for rethinking comfort-based control strategies, they should be interpreted within the context of our validation approach. We deliberately selected a canonical EnergyPlus building model (5ZoneAutoDXVAV) and standardized occupant assumptions (fixed metabolic rates and clothing values) to ensure reproducible comparisons across control strategies. This controlled approach, while limiting immediate generalizability to all building types and occupant populations, was essential for isolating the effects of different comfort modeling approaches. The consistent performance patterns observed across four diverse DOE climate zones and under stochastic weather perturbations suggest our findings reflect fundamental characteristics of comfort-based control rather than artifacts of specific test conditions.

\subsection{The PMV Optimization Paradox: Why Naive Comfort Control Fails}

The stark contrast between the energy performance of naive PMV control (\texttt{pmv0}) and grid-search optimized PMV strategies (\texttt{pmv1-3}) reveals a fundamental paradox in building control: pursuing thermal comfort without considering energy implications can be counterproductive for both objectives. Our results demonstrate that \texttt{pmv0} consistently increases energy consumption by 4.3--23.5\% across all climates, while the optimized variants achieve substantial savings of 7--18.5\%. This counterintuitive finding challenges the conventional wisdom that comfort-focused control inherently improves occupant satisfaction.

The failure of naive PMV control stems from its myopic focus on instantaneous comfort optimization without consideration of building thermal dynamics or energy costs. When PMV predictions deviate from the neutral zone ($\pm 0.5$), the \texttt{pmv0} controller immediately adjusts setpoints by the maximum allowable increment (1$^\circ$C every 15 minutes), creating a reactive control behavior that fights against natural building thermal inertia. This aggressive pursuit of thermal neutrality leads to several pathological behaviors: (1) \textit{thermal oscillations}, where the controller overcorrects for temporary comfort deviations, causing the HVAC system to repeatedly overshoot target conditions; (2) \textit{energy waste during transients}, as the system expends maximum effort to counteract natural temperature fluctuations that occupants might not even perceive; and (3) \textit{conflict with building physics}, where the controller attempts to maintain impossible steady-state conditions in buildings with significant thermal mass and environmental variability.

\subsection{The Grid-Search Solution: Balancing Competing Objectives}

The success of grid-search optimization (strategies \texttt{pmv1-3}) lies in its ability to transform the comfort-energy relationship from adversarial to synergistic through systematic exploration of the control parameter space. Rather than making instantaneous reactive adjustments, grid-search evaluates multiple potential setpoint modifications simultaneously, selecting the adjustment that minimizes the distance to comfort boundaries while implicitly considering energy implications through its bounded search space. This approach effectively implements a form of \textit{predictive comfort control}, where the system anticipates the consequences of control actions before implementation, avoiding the energy-wasteful overcorrections that plague naive approaches.

The effectiveness of grid-search optimization becomes apparent when examining its operational characteristics compared to naive control. While \texttt{pmv0} exhibits binary decision-making (either maximum adjustment or no change), the grid-search variants explore a discrete set of potential adjustments $\Delta T \in \{-2, -1, 0, +1, +2\}$$^\circ$C and select the option that brings predicted comfort metrics closest to the target boundaries of $\pm 0.5$ PMV. This multi-option evaluation serves as an implicit energy regularization mechanism: smaller adjustments are preferred when they achieve similar comfort outcomes, naturally preventing the aggressive overcorrections that characterize naive control. Furthermore, the bounded search space prevents the controller from making extreme adjustments that could destabilize building thermal conditions, ensuring that comfort improvements are achieved through measured, systematic interventions rather than reactive responses to momentary comfort deviations.

\subsection{Climate-Dependent Performance and Physical Insights}

The magnitude of energy penalties associated with naive PMV control varies significantly across climate zones, providing insights into the underlying physical mechanisms driving these performance differences. Cold climates (Stockholm: +23.5\%, Helsinki: +21.4\%) exhibit the largest energy penalties under \texttt{pmv0}, while hot-dry climates (Davis-Monthan: +7.7\%) show more modest increases. This pattern reflects the asymmetric energy costs of heating versus cooling in different climates, where heating-dominated buildings experience greater energy penalties from setpoint hunting due to the typically higher energy intensity of space heating compared to cooling, particularly when considering the coefficient of performance differences between heat pumps/boilers and vapor compression cooling systems.

The consistent success of grid-search optimization across all climate zones—with savings ranging from 6.7--18.5\%—suggests that the fundamental problem with naive comfort control is not climate-specific but rather stems from the mismatch between instantaneous comfort optimization and building system dynamics. The grid-search approach succeeds because it implicitly recognizes that thermal comfort is better served by stable, predictable indoor conditions rather than aggressive pursuit of instantaneous comfort metrics. This finding has profound implications for building control philosophy: rather than viewing comfort and energy as competing objectives requiring explicit multi-objective optimization, grid-search demonstrates that properly constrained comfort control can naturally achieve energy efficiency through its systematic exploration of the control parameter space. The success of this approach across diverse climate conditions suggests that the underlying control principles are robust to environmental variability and building-specific characteristics.

\subsection{The Diminishing Returns of Advanced ML Models: Approaching Physical Performance Limits}

A striking finding from our results is that sophisticated machine learning models (LightGBM and PINN-VAE) demonstrate significantly smaller improvements from grid-search optimization compared to the analytical PMV model. While PMV-based control achieves dramatic energy savings of 15--18.5\% when optimized via grid-search, the ML models show only marginal improvements of 1--3\% over their non-optimized counterparts. This counterintuitive result challenges the prevailing assumption that more sophisticated predictive models necessarily translate to better control performance, and suggests that building control systems may be approaching fundamental physical performance limits that transcend the accuracy of individual comfort prediction models.

The modest gains observed in optimized ML models can be attributed to the fact that these models were already operating near optimal control regimes prior to grid-search enhancement. Unlike the analytical PMV model, which suffers from well-documented limitations in capturing individual comfort variations and adaptive behaviors, data-driven models like LightGBM and PINN-VAE inherently encode more nuanced comfort patterns learned from large-scale occupant feedback databases. Consequently, their baseline control strategies (\texttt{lightgbm} and \texttt{pv}) already avoid many of the pathological behaviors that plague naive PMV control, such as excessive setpoint hunting and overcorrection for transient comfort deviations. When grid-search optimization is subsequently applied to these already-refined control strategies, the potential for improvement is inherently limited because the models are operating closer to the theoretical performance frontier defined by building physics and HVAC system constraints.

\subsection{The Building-as-Bottleneck Hypothesis}

Our findings support a \textit{building-as-bottleneck hypothesis}, which posits that the performance ceiling in building control is determined not by the sophistication of comfort prediction models, but by the fundamental constraints of building thermal dynamics and HVAC system capabilities. The convergence of performance between optimized PMV and optimized ML models suggests that all control strategies are encountering the same underlying physical limitations: thermal mass response times, HVAC system capacity constraints, sensor measurement uncertainties, and the discrete nature of available control actions (1$^\circ$C setpoint adjustments every 15 minutes). Once a control strategy successfully navigates around the major inefficiencies caused by poor comfort prediction (as grid-search enables PMV to do), further improvements require overcoming these physical system constraints rather than enhancing prediction accuracy.

This hypothesis is supported by the consistent ranking patterns observed across diverse climate zones, where optimized control strategies cluster within narrow performance bands regardless of the underlying prediction model sophistication. The fact that a physics-informed variational autoencoder (PINN-VAE)—incorporating explicit physiological constraints and advanced neural architectures—performs only marginally better than optimized analytical PMV suggests that the control problem has shifted from one of comfort prediction accuracy to one of system-level optimization within physical constraints. The building and its HVAC system, rather than the comfort model, become the limiting factors in achieving further performance improvements.


\subsection{Limitations and Future Research}

A significant limitation of this study is that all seven control strategies primarily rely on thermal comfort constraints without explicitly incorporating energy consumption objectives in the control formulation, which may result in suboptimal performance. Traditional building model predictive control (MPC) research typically defines multi-objective cost functions that explicitly balance thermal comfort and energy use, with optimization performed over prediction horizons. 
Similarly, reinforcement learning approaches for building control employ reward functions that combine penalties for both comfort violations and excessive energy consumption. However, in this study, we intentionally adopted simpler, more practical HVAC control strategies readily available for conventional building management systems (BMS) to compare how different thermal comfort modeling techniques affect control performance. 
This methodological choice makes our findings applicable to a broader range of building contexts, as more sophisticated control strategies typically require additional data streams (weather forecasting, comprehensive sensing networks, or real-time occupancy detection) that may not be available in most existing buildings. 
Furthermore, this study evaluates control performance using a simplified building model that may not adequately represent the diverse building typologies, geometry, and occupancy patterns found in practice. Building construction characteristics also vary significantly across different climates, which may further limit the representativeness of our case study building and restrict the generalizability of our findings to more complex building environments. 

We also discovered that naive ML-based controllers (LightGBM, PV) frequently produced setpoints outside the [12$^\circ$C, 30$^\circ$C] actuator range, resulting in actuator saturation and misleading energy-comfort metrics; addressing such clipping effects is a critical avenue for future work. While this study evaluates a single building model, the robustness of our findings is strengthened by the stochastic weather perturbations applied throughout all simulations. Each control strategy was effectively tested across a distribution of weather conditions rather than a single deterministic year, providing implicit statistical validation of the performance differences observed.

The actuator saturation behavior revealed by our co-simulation framework exemplifies the type of implementation challenge that traditional comfort modeling research cannot address in isolation. While developing solutions to saturation is beyond the scope of this methodological contribution, our framework provides the essential infrastructure for future control-focused research to systematically evaluate remediation strategies. This includes exploring soft constraints, anti-windup schemes, or alternative actuator configurations that expand the feasible control space. By establishing direct coupling between comfort models and building simulation, our methodology enables researchers to identify and quantify such deployment challenges before real-world implementation, potentially saving significant time and resources in building control system development.

Our findings have profound implications for both practical building control deployment and future research directions. From a deployment perspective, the minimal performance advantage of complex ML models over optimized PMV suggests that the substantial computational, maintenance, and infrastructure costs associated with ML-based control systems may not be justified by their modest performance gains. Grid-search optimized PMV control offers a compelling value proposition: it achieves near-optimal energy performance using well-established analytical models that require no training data, avoid overfitting concerns, and remain interpretable to building operators. This is particularly relevant for retrofit applications and smaller buildings where the infrastructure costs of ML deployment may be prohibitive.

From a research perspective, our results suggest that future efforts in building control optimization should focus on addressing system-level constraints rather than further refinements to comfort prediction models. Priority areas might include: (1) \textit{enhanced actuation capabilities}, such as continuously variable setpoint control rather than discrete 1$^\circ$C adjustments; (2) \textit{multi-zone coordination strategies} that leverage spatial thermal diversity to improve overall system efficiency; (3) \textit{predictive control horizons} that anticipate weather and occupancy patterns beyond the current 15-minute control interval; and (4) \textit{integrated building-grid optimization} that considers utility pricing and demand response opportunities. The plateauing performance of comfort prediction improvements indicates that the next frontier in building control lies in these system-level enhancements rather than continued refinement of individual comfort models.

Furthermore, our findings raise important questions about the allocation of research resources in the building controls community. The substantial effort invested in developing sophisticated thermal comfort models—including physics-informed neural networks, ensemble learning approaches, and personalized comfort prediction—may yield diminishing returns if the underlying building systems cannot effectively utilize the enhanced prediction accuracy. This suggests a need to rebalance research priorities toward addressing the fundamental constraints that limit control system performance, potentially through novel HVAC architectures, advanced sensor networks, or integrated building-grid optimization strategies that expand the available control parameter space beyond traditional setpoint manipulation.