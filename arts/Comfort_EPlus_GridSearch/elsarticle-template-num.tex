%% 
%% Copyright 2007-2025 Elsevier Ltd
%% 
%% This file is part of the 'Elsarticle Bundle'.
%% ---------------------------------------------
%% 
%% It may be distributed under the conditions of the LaTeX Project Public
%% License, either version 1.3 of this license or (at your option) any
%% later version.  The latest version of this license is in
%%    http://www.latex-project.org/lppl.txt
%% and version 1.3 or later is part of all distributions of LaTeX
%% version 1999/12/01 or later.
%% 
%% The list of all files belonging to the 'Elsarticle Bundle' is
%% given in the file `manifest.txt'.
%% 
%% Template article for Elsevier's document class `elsarticle'
%% with numbered style bibliographic references
%% SP 2008/03/01
%% $Id: elsarticle-template-num.tex 272 2025-01-09 17:36:26Z rishi $
%%
\documentclass[preprint,12pt]{elsarticle}

%% Use the option review to obtain double line spacing
%% \documentclass[authoryear,preprint,review,12pt]{elsarticle}

%% Use the options 1p,twocolumn; 3p; 3p,twocolumn; 5p; or 5p,twocolumn
%% for a journal layout:
%% \documentclass[final,1p,times]{elsarticle}
%% \documentclass[final,1p,times,twocolumn]{elsarticle}
%% \documentclass[final,3p,times]{elsarticle}
%% \documentclass[final,3p,times,twocolumn]{elsarticle}
%% \documentclass[final,5p,times]{elsarticle}
%% \documentclass[final,5p,times,twocolumn]{elsarticle}

%% For including figures, graphicx.sty has been loaded in
%% elsarticle.cls. If you prefer to use the old commands
%% please give \usepackage{epsfig}

%% The amssymb package provides various useful mathematical symbols
\usepackage{amssymb}
% \usepackage[noend]{algpseudocode}
\usepackage[ruled,vlined]{algorithm2e}

%% The amsmath package provides various useful equation environments.
\usepackage{amsmath}
\usepackage[margin=1in]{geometry}


\usepackage{amsmath}
\usepackage{amsfonts}
\usepackage{amssymb}
\usepackage{graphicx}
\usepackage{booktabs} % For professional looking tables
\usepackage{longtable,makecell,tabularx,booktabs} % For tables that might span multiple pages
\usepackage{multirow}

% Add any other packages you commonly use

%% The amsthm package provides extended theorem environments
%% \usepackage{amsthm}

%% The lineno packages adds line numbers. Start line numbering with
%% \begin{linenumbers}, end it with \end{linenumbers}. Or switch it on
%% for the whole article with \linenumbers.
%% \usepackage{lineno}

\journal{Energy and Buildings}

\begin{document}

\begin{frontmatter}

%% Title, authors and addresses

%% use the tnoteref command within \title for footnotes;
%% use the tnotetext command for theassociated footnote;
%% use the fnref command within \author or \affiliation for footnotes;
%% use the fntext command for theassociated footnote;
%% use the corref command within \author for corresponding author footnotes;
%% use the cortext command for theassociated footnote;
%% use the ead command for the email address,
%% and the form \ead[url] for the home page:
%% \title{Title\tnoteref{label1}}
%% \tnotetext[label1]{}
%% \author{Name\corref{cor1}\fnref{label2}}
%% \ead{email address}
%% \ead[url]{home page}
%% \fntext[label2]{}
%% \cortext[cor1]{}
%% \affiliation{organization={},
%%             addressline={},
%%             city={},
%%             postcode={},
%%             state={},
%%             country={}}
%% \fntext[label3]{}

\title{A Co-Simulation Methodology for Integrating Data-Driven Thermal Sensation Models with Building Energy Control}

%% use optional labels to link authors explicitly to addresses:
%% \author[label1,label2]{}
%% \affiliation[label1]{organization={},
%%             addressline={},
%%             city={},
%%             postcode={},
%%             state={},
%%             country={}}
%%
%% \affiliation[label2]{organization={},
%%             addressline={},
%%             city={},
%%             postcode={},
%%             state={},
%%             country={}}

\author[hku]{Hongshan Guo} %% Author name
\author[hku]{Kanxuan He}
\author[nus]{Yue Lei}
% \author{hku}{Yu Chang}
%% Author affiliation
\affiliation[hku]{organization={Department of Architecture, Faculty of Architecture, The University of Hong Kong},%Department and Organization
            % addressline={}, 
            city={Hong Kong SAR},
            % postcode={}, 
            % state={},
            country={China}}

\affiliation[nus]{organization={Department of Built Environment, National University of Singapore},%Department and Organization
            % addressline={}, 
            % city={Hong Kong SAR},
            % postcode={}, 
            % state={},
            country={Singapore}}

%% Abstract
\begin{abstract}
This paper presents a novel co-simulation methodology that directly integrates data-driven thermal sensation models with EnergyPlus building control, addressing the gap between comfort prediction research and practical HVAC implementation. Our framework enables real-time coupling of any comfort predictor—from analytical PMV to deep learning models—with building control while handling actuator saturation and stochastic disturbances. 
We demonstrate the methodology by implementing seven control strategies using models trained on 148,148 occupant votes, including PMV, LightGBM, and a physics-informed neural network-variational autoencoder (PINN-VAE), revealing critical implementation challenges such as actuator saturation in ML-based controllers.
The framework reveals unexpected insights: grid-search optimized PMV achieves 15–18.5\% energy savings comparable to sophisticated ML models while requiring no training data and completing simulations 50x faster. All optimized strategies reduce uncomfortable hours from 25\% to below 3\% while saving energy. Furthermore, incorporating PINN-VAE's physiological predictions improves comfort by 14.85\% with minimal energy impact. These findings, enabled by our co-simulation methodology, demonstrate that control integration quality matters more than model sophistication—challenging assumptions about comfort-based building control and providing essential infrastructure for realistic control strategy evaluation.
\end{abstract}
%%Graphical abstract
\begin{graphicalabstract}
\centering
\includegraphics[width=.90\linewidth]{figs/gridworkflow_3.pdf}\\
\\
\includegraphics[width=.65\linewidth]{figs/savings_r.png}\\
\includegraphics[width=.95\linewidth]{figs/saving_end_pct.png}
%\includegraphics{grabs}
\end{graphicalabstract}

%%Research highlights
\begin{highlights}
\item Co-simulation of data-driven thermal sensation models (148,148 votes) with EnergyPlus
\item Grid-search PMV achieves 15-18.5\% energy savings, comparable to ML models
\item Runtime comparison: PMV/LightGBM <200s vs PINN-VAE 10-15min per year
\item Incorporating skin temperature improves comfort 14.85\% at 1.97\% energy cost
\item All optimized modes reduce unmet hours to <3\% while saving energy
\end{highlights}

%% Keywords
\begin{keyword}
building energy simulation \sep thermal sensation \sep machine learning \sep deep learning \sep model predictive control
\end{keyword}

\end{frontmatter}

%% Add \usepackage{lineno} before \begin{document} and uncomment 
%% following line to enable line numbers
%% \linenumbers

%% main text
%%

%% Use \section commands to start a section
The thermal load that building‐energy guidelines ascribe to occupants remains remarkably uniform: most major codes and simulation prototypes fix an office worker at 1.2\,met, equivalent to roughly 120\,W of metabolic heat as widely seen across building codes, design guidelines\citep{EN16798_2019, GB50189_2015, JISA4706_2007} and engineering references for building energy simualtion \citep{DOEPrototype2018, ECBC2017}.  This single‐value prescription persists even as the industry champions “occupant-centric” design and advanced comfort analytics.  Large post-occupancy surveys nevertheless continue to report chronic over-cooling complaints, disproportionately voiced by women and older adults \citep{Karjalainen2007, Schweiker2012, Kim2021}.  Such evidence suggests that the canonical 1.2\,met assumption may systematically overstate internal heat for sizeable portions of the population, driving lower supply-air temperatures, higher sensible and latent cooling loads, and gender-skewed discomfort.

The 1.2\,met default traces back to Fanger’s laboratory work, which converted a basal surface heat flux of 58.2\,W\,m$^{-2}$ into the unit \textit{met} and, by extension, into load tables used world-wide \citep{Fanger1970}.  Contemporary standards embed that lineage with little variation: the U.S. DOE prototype models for ASHRAE 90.1 set occupants at 120\,W; EN 16798-1 lists an office default of 118\,W; China’s GB 50189 prescribes 110\,W; Japanese and Indian codes lie in the same band. That is, even within different regulations towards the average wattage of heat generation from existing regulations, the EnergyPlus default at 120 watts per person is a generous over-estimation. Field studies have put real office workers heat generate rates to range from 45–110 W, with women disproportionately at the lower end \citep{Karjalainen2007, Kingma2015}. Prior energy-model papers varied either a single BMR equation \citep{Ahmed2017} or a local sensitivity band \citep{Chen2020}, leaving the cross-climate energy–comfort impact unquantified
% A concise comparison of these codified values is provided in Table \ref{tab:defaults} in Section 2. I don't think we need this sentence?

Physiological research, meanwhile, demonstrates a substantial variability in basal metabolic rate (BMR), and by association the resting metabolic rate, which represents the at-rest metabolic rates of average occupants.  Predictive equations such as Harris–Benedict, Cunningham, Henry, and Mifflin–St Jeor routinely yield BMR values of 60–80\,W for a large share of adult women and older adults, even after applying the conventional 10 \% lift from BMR to resting metabolic rate (RMR) \citep{HarrisBenedict1918, Cunningham1980, Henry2005, Mifflin1990}.  The gap between these empirically grounded values and the 120\,W default therefore ranges from 25 to 50 W per person—enough to bias predicted cooling loads and to justify lower operative setpoints that many occupants perceive as uncomfortably cold.

Despite decades of evidence for metabolic diversity, neither building codes nor mainstream simulation workflows have revisited the occupant heat-gain constant.  The resulting combination of inflated internal loads and one-size-fits-all comfort models risks unnecessary energy expenditure and unequal comfort outcomes.  This study addresses that overlooked lever by systematically quantifying (i) the energy penalty and (ii) the comfort inequity attributable to uniform metabolic-rate assumptions.

Two complementary update strategies are evaluated.  First, a set of composite scenarios recalculates per-person heat gains using established RMR equations scaled through Monte Carlo sampling of demographic distributions.  Second, a data-driven approach samples real occupant profiles from large thermal-comfort databases to generate stochastic metabolic-rate schedules.  Comparing these scenarios with the 120\,W baseline across a reference office building isolates the incremental cooling energy and predicted discomfort that current standards inadvertently lock in.  By quantifying these impacts, we aim to provide evidence-based guidance for revising occupant heat-gain inputs—thereby reducing avoidable cooling kilowatt-hours, carbon emissions, and persistent gendered discomfort in air-conditioned buildings. We find that replacing 120 W with a demographic-aware metabolic distribution lowers HVAC site energy by ⟨\%⟩ and halves gender-based comfort bias; Sections 2–4 detail methods, results and implications.

\section{Literature Review}

\subsection{Analytical PMV and Adaptive Comfort Models}

Fanger’s Predicted Mean Vote (PMV) model, codified in ASHRAE Standard~55, remains the foundational analytical approach for estimating average occupant thermal sensation based on heat‐balance equations and empirical metabolic and clothing parameters \cite{ASHRAE2020}. However, PMV assumes quasi–steady‐state conditions and representative activity levels, leading to inaccuracies under transient indoor environments such as variable HVAC cycles and dynamic occupancy patterns \cite{Run2025Transient}. To address computational challenges in large‐scale or transient populations, Sirhan and Golan proposed an efficient piecewise linear regression method for PMV calculation that significantly reduces computation time while maintaining acceptable accuracy in public settings \cite{Sirhan2021EfficientPMV}.

Adaptive comfort models extend the PMV paradigm by correlating occupant sensation with outdoor running‐mean temperatures and behavioral adaptation, improving relevance in naturally ventilated and free‐running buildings \cite{Yao2022Adaptive}. A recent review in \emph{Energies} analyzed office‐building field studies across diverse climates, confirming that adaptive models better capture occupant preferences while highlighting the need for climate‐specific comfort equations \cite{MDPI2023Adaptive}.

\subsection{Data‐Driven Thermal Comfort Modeling}

The release of ASHRAE Global Thermal Comfort Database II has catalyzed data‐driven comfort models. Raissi \emph{et al.} embedded transient heat‐transfer equations within a physics‐informed variational autoencoder (PINN‐VAE), achieving 15–20\% RMSE improvements over PMV benchmarks \cite{Raissi2022VAE}. Chen \emph{et al.} introduced a control‐oriented PhysCon PINN architecture that leverages building thermal‐mass parameters to enhance demand‐response predictions and energy management \cite{Chen2024PINN}. Boutahri and Tilioua demonstrated that ensemble ML models (SVM, ANN, RF, XGBoost) can reduce thermal sensation vote prediction errors by up to 20\% compared to analytical PMV metrics \cite{Boutahri2024}.

\subsection{Machine Learning for Energy Consumption Prediction}

Gradient‐boosting decision trees have emerged as a leading technique for building energy forecasting. Zhou \emph{et al.} showed that LightGBM outperforms XGBoost in predicting operational carbon emissions by capturing nonlinear interactions via SHAPley values, enabling more accurate downstream control strategies \cite{Zhou2024LightGBM}.

\subsection{Co‐Simulation and Control Optimization in HVAC Systems}

Co‐simulation frameworks coupling EnergyPlus with CONTAM or the Building Controls Virtual Test Bed allow integrated evaluation of HVAC strategies \cite{alonso2022using}. A taxonomic review highlighted the Functional Mock‐up Interface (FMI) as the most prominent co‐simulation standard for buildings and smart energy systems \cite{Alfalouji2023CoSim}.

Bayesian optimization has been applied to HVAC setpoint tuning, achieving near-optimal performance with minimal model evaluations \cite{Lin2023BayesOpt}. Evolutionary algorithms, including genetic algorithms and particle‐swarm optimization, have been used for multi‐objective HVAC control, balancing energy use and comfort across representative scenarios \cite{EC32024Evolutionary,MultiObj2024}. Reinforcement learning offers model‐free strategies that adapt to stochastic occupancy and weather; recent reviews synthesize best practices and safety constraints for RL‐based HVAC control \cite{RLReview2025}. However, many prior studies do not explicitly consider actuator-saturation (i.e., when commanded setpoints exceed the hard bounds of 12$^\circ$C or 30$^\circ$C and are clipped), which can lead to controllers becoming “stuck” at the boundary and masking true comfort-versus-energy trade-offs.
\section{Methodology}
\subsection{Overall Research Framework}
\label{sec:overall_framework}

Our workflow integrates thermal sensation modeling with building simulation in a unified co‐simulation loop (Figure~\ref{fig:workflow}) based on Sinergym virtual testbed \cite{campoy2025sinergym}. First, at each 15-minute timestep, Sinergym queries EnergyPlus for current zone states (air temperature, humidity, etc.) and weather inputs. Second, a chosen comfort predictor—PMV, LightGBM, or PINN-VAE—estimates zone‐level thermal sensation votes (TSVs). Third, the control module computes new heating ($T_{h}$) and cooling ($T_{c}$) setpoints based on the predicted TSV (or a $\Delta T$ grid search around an interior reset). Finally, these setpoints are sent back to EnergyPlus via Sinergym’s actuator API, and the simulation advances. This cycle repeats for all timesteps across each selected climate, enabling direct comparison of energy and comfort outcomes under different control strategies.

\begin{figure}[h!]
    \centering
    \includegraphics[width=0.95\linewidth]{figs/gridworkflow_3.pdf}
    \caption{Overall Workflow of Current Study}
    \label{fig:workflow}
\end{figure}

\subsection{Simulation Setup and Testbed Building}%Confirm we have enough on systems...?
% \subsubsection{Building Model Description and Justification}
This study adopts the \texttt{5ZoneAutoDXVAV} building model distributed with EnergyPlus \citep{energyplus}, which is commonly used for building control algorithm benchmarking \cite{gao2021deep, an2023clue, kadamala2024enhancing}. The model represents a single-story, 463 m$^2$ office building with five conditioned thermal zones and one return plenum, reflecting a prototypical open-plan office layout.

The building model supports zone-level temperature setpoint adjustment through Sinergym's actuator API, with the Energy Management System (EMS) framework enabling real-time control integration \cite{campoy2025sinergym}. We selected this model because it provides architectural symmetry and zoning simplicity that enable lightweight simulation environment and systematic algorithm testing, making it well-suited for comparative evaluation of different control strategies. While using a single building model limits generalizability to other building types and HVAC systems, this approach isolates performance differences attributable specifically to thermal comfort models and their climate interactions, rather than confounding variables from diverse building characteristics.


% \subsection{Envelope and Construction Details}
% The envelope construction is defined via abstracted \texttt{Construction} objects such as \texttt{WALL-1} and \texttt{ROOF-1}, though material layer specifications are absent in this JSON version. No window surface areas or window material properties are defined, and as such, the window-to-wall ratio (WWR) cannot be computed directly from this model. Nevertheless, the model's simplicity ensures consistent thermal response behavior and avoids geometric ambiguities \citep{drgovna2020all}.

% Each conditioned zone includes:
% \begin{itemize}
%   \item A \texttt{People} object representing occupancy density,
%   \item A \texttt{Lights} object simulating fixed lighting load, and
%   \item An \texttt{ElectricEquipment} object representing plug loads.
% \end{itemize}
% All internal gains are scheduled using \texttt{Schedule:Compact}, typically emulating business-hour profiles consistent with standard office operations. On top of that, the model includes:
% \begin{itemize}
%   \item A single \texttt{AirLoopHVAC} system with an outdoor air subsystem and return plenum (\texttt{PLENUM-1}),
%   \item Zone-level conditioning using VAV terminal units with DX cooling coils, and
%   \item Thermostat setpoint schedules that are designed to be overridden using external control via EMS actuators.
% \end{itemize}

% This setup supports cooling setpoint adjustment at the zone level, as exposed by Sinergym’s actuator API. The EMS framework facilitates real-time control from reinforcement learning agents without manual editing of EnergyPlus schedules or control logic.

% We selected this model for its:
% \begin{itemize}
%   \item \textbf{Representativeness}: Reflects the spatial and control complexity of a typical commercial office building;
%   \item \textbf{Simplicity}: Abstract geometry enables efficient simulation, particularly valuable for reinforcement learning experiments requiring thousands of rollouts;
%   \item \textbf{Modularity}: Provides a multi-zone testbed suitable for testing both centralized and decentralized control policies;
%   \item \textbf{Compatibility}: Fully integrated into the Sinergym ecosystem with predefined action and observation mappings \citep{perarnau2021sinergym};
%   \item \textbf{Reproducibility}: Used widely in prior RL-for-HVAC studies, including benchmarking efforts in \citep{gao2021deep}.
% \end{itemize}

% We understand the selection of a single IDF for this study may come with its own limitation, that the results we obtained will be specific to this building type and HVAC system. However, we believe a single, well-characterized IDF can help us isolate the performance differences that can be attributable solely to the model-enabled control strategies and their interactions with different climates, rather than variations in building systems/designs. %Reiterate this again in conclusions/discussions


\subsection{Data Source for Thermal Comfort Modeling} \label{sec:data_foundation}

As we're showing in Figure~\ref{fig:workflow}, the data source of the current study is a combination of both the ASHRAE Global Thermal Comfort Database II and the Chinese Thermal Comfort Datasets combined. This combined dataset represents a form of \textit{collective intelligence}, aggregated as 148,148 distinct pre-existing thermal sensation steady-state votes, encompassing both field and laboratory experiments from ASHRAE Thermal Comfort Database II and Chinese Thermal Comfort Datasets. This amalgamation results in a substantial corpus of 148,148 individual data points, capturing a wide diversity of environmental conditions, building typologies, geographical locations, and demographic profiles. 

The aggregation process involved careful harmonization of thermal sensation votes, typically reported on the 7-point ASHRAE scale, and standardization of input features to ensure consistency across the varied source studies. Further details on the construction, composition, and validation of this aggregated dataset can be found in [Dataset Paper Citation, if applicable, or a brief description of your harmonization process if not published separately]. This large-scale, diverse data foundation aims to foster the development of more generalized and robust thermal sensation models compared to those trained on smaller, single-source experimental data. Using this dataset, we were able to train both a machine learning (LightGBM) and deep learning (PINN-VAE, or physiological-informed neural net with variable autoencodeer) models for predicting thermal sensation are developed upon a comprehensive, aggregated dataset.

\subsection{Thermal Sensation Vote Predictors}\label{sec:comfort_models}
Aside from the analytical thermal sensation predictor of the population's mean thermal sensation, i.e. Predicted Mean Vote (PMV) \cite{Fanger1970}, we made explicit decisions on choosing the appropriate thermal sensation vote predictor. The selection process was guided by the dual objectives of robustness and innovation. We chose Light Gradient Boosting Machine (LightGBM) due to its consistently strong predictive performance with tabular datasets, as extensively validated in previous research\cite{Ke2017}. LightGBM offers computational efficiency, interpretability, and robust generalization capabilities, making it highly suitable for real-time predictive control tasks where reliability and speed are critical.

Concurrently, we introduce a novel Physics-Informed Neural Network combined with a Variational Autoencoder (PINN-VAE), specifically developed by the authors for thermal sensation prediction (publication forthcoming). This innovative model integrates physiological realism into thermal comfort modeling, explicitly leveraging Gagge's two-node model (Gagge et al., 1966) to analytically inform intermediate physiological outputs such as skin temperature ($T_{skin}$) and core body temperature ($T_{core}$). Unlike conventional predictive methods, the PINN-VAE architecture not only predicts thermal sensation with high fidelity through its deep neural network’s multi-head capability but simultaneously generates robust physiological data, significantly enhancing its utility in physiology-informed indoor environmental controls. This combination of predictive accuracy and physiological realism positions PINN-VAE as a potential transformative approach in building thermal environment management.

\subsubsection{Analytical Baseline: Predicted Mean Vote (PMV)}
\label{sec:pmv_model}
The Predicted Mean Vote (PMV) model, as developed by Fanger and standardized in ISO 7730 \citeplaceholder{ISO7730} and ASHRAE Standard 55 \citeplaceholder{ASHRAE55}, serves as an analytical baseline. PMV predicts the mean thermal sensation vote of a large group of people based on six key parameters: air temperature ($T_a$), mean radiant temperature ($T_r$), relative humidity ($RH$), air velocity ($v_{air}$), metabolic rate (Met), and clothing insulation ($I_{cl}$). In this study, when PMV-based control is active, these input parameters are derived from EnergyPlus outputs as observation of the environmental variables, coupled with predefined assumptions for occupant metabolic rate and clothing insulation, with more details documented in Section \ref{sec:control_strategies}.

We adopted standardized assumptions for occupant metabolic rates and clothing insulation based on established industry guidelines and widely accepted thermal comfort research. Specifically, a metabolic rate (MET) of 1.1 (sedentary office work) was used uniformly across simulations, consistent with recommendations by ASHRAE Standard 55\cite{ASHRAE2020}. Clothing insulation values (clo) were set to 0.65 during summer and 1.0 in winter, reflecting typical seasonal clothing variations commonly observed in office environments \cite{ASHRAE2020}.

These standardized assumptions facilitate reproducibility and enable direct comparability with established comfort models, such as PMV, while simultaneously aligning with typical office conditions documented in thermal comfort literature. Although using fixed values simplifies simulation procedures, we acknowledge potential limitations arising from the natural variability of individual occupant behavior and adaptive clothing choices in real-world scenarios.

\subsubsection{Machine Learning Benchmark: LightGBM}\label{sec:lightgbm_model}
A Light Gradient Boosting Machine (LightGBM) model is utilized as a high-performance machine learning benchmark for thermal sensation prediction. LightGBM is a tree-based gradient boosting framework known for its efficiency and accuracy \citeplaceholder{LightGBM\_Paper}. The model is trained on the aggregated dataset described in Section \ref{sec:data_foundation} to predict Thermal Sensation Vote (TSV) on the 7-point ASHRAE scale. Inputs to the LightGBM model include key environmental parameters such as [$T_a, T_r, RH, v_{air}$, clo, MET, gender, height, weight, etc.]. The model was trained using k-fold cross-validation, with hyperparameters optimized for predictive accuracy (details of training and validation are beyond the scope of this section but can be found in Appendix).

\subsubsection{Deep Learning Model: Physics-Informed Neural Network - Variational Autoencoder (PINN-VAE)}
\label{sec:pinn_vae_model}
The core novel model investigated in this study is a Physics-Informed Neural Network combined with a Variational Autoencoder (PINN-VAE), designed specifically for thermal sensation prediction and co-simulation of key physiological variables. This model architecture, detailed in a pending publication at BAE (will cite when publication is finalized), aims to overcome limitations of purely data-driven approaches by integrating domain knowledge such as physiological constraints on skin/core temperature, heat balance whilst improving the quality of input data that are missing.

In particular, the VAE component learns a robust latent representation of the input data, beneficial for handling the complexities and potential imperfections inherent in large-scale aggregated datasets. The PINN component incorporates physiological realism by embedding constraints derived from principles of human thermoregulation [e.g., based on a simplified Gagge's 2-node model \citeplaceholder{Gagge1966} or similar -- specify principles used, e.g., heat balance equations for core and skin compartments]. These physics-informed constraints guide the model to produce physiologically plausible predictions for mean skin temperature ($T_{skin}$) and core body temperature ($T_{core}$), alongside the primary output of TSV.

Inputs to the PINN-VAE are similar to the LightGBM model: [$T_a, T_r, RH, v_{air}$]. Its outputs for this study are the predicted TSV (on the 7-point ASHRAE scale), predicted $T_{skin}$, and predicted $T_{core}$. The key advantages of this PINN-VAE approach include its potential for improved generalization due to the physics-based regularization, enhanced interpretability through the prediction of intermediate physiological states, and the unique ability to leverage $T_{skin}$ as part of advanced control strategies. Training and validation details are provided in the aforementioned paper.

\section{Control Framework Implementation and Testing}
To demonstrate our co-simulation methodology's capabilities and assess its effectiveness, we implement seven distinct control strategies spanning analytical models to sophisticated deep learning approaches. This implementation serves multiple purposes: (1) confirming the framework successfully accommodates diverse model types and control logics, (2) testing the robustness of the actuator saturation handling and stochastic weather integration, and (3) generating initial insights into the relative performance of different comfort modeling approaches when deployed in actual control loops.

We emphasize that this section presents a systematic testing of our methodology rather than validation against measured building data—such empirical validation represents important future work once the framework's capabilities are established.

Our testing protocol comprises three complementary investigations:
\begin{itemize}
    \item PMV Grid-Search Optimization Study (Section 5.2): We first examine whether systematic optimization can enhance classical PMV control performance. Using 10 diverse climate locations, we compare naive PMV control against three grid-search variants to establish the potential of optimization for analytical models.
    \item Comprehensive Control Strategy Comparison (Section 5.3): We then apply our complete framework to compare seven control strategies—including both optimized and non-optimized variants of PMV, LightGBM, and PINN-VAE models—across four representative DOE climate zones. This investigation tests the framework's ability to fairly evaluate fundamentally different modeling approaches.
    \item Physiological Variable Integration (Section 5.4): Finally, we demonstrate the framework's extensibility by incorporating PINN-VAE's unique physiological predictions (skin temperature) into control logic, showcasing how our methodology enables novel control strategies beyond traditional comfort metrics.
\end{itemize}

\subsection{Control Strategies Tested}
\label{sec:control_strategies}
Across all three investigations, control strategies share common implementation characteristics. Despite their different internal logic, all controls implement a bang–bang adjustment of the heating and cooling set-points to be directly comparable against one another, with a maximum step of 1$^\circ$C per timestep (every 15 minutes). All modes share the following pre-processing steps:

\begin{itemize}
  \item \textbf{Occupancy check:} The space is considered unoccupied if it is a weekend day or outside the hours of 06:00–22:00. In unoccupied periods, the allowable comfort range is relaxed by \(\pm\SI{2}{\degreeCelsius}\).
  \item \textbf{Seasonal set‐points:} Two baseline comfort ranges, \(\left[T_\mathrm{low}^\mathrm{sum},T_\mathrm{high}^\mathrm{sum}\right]\) for summer (June 1–September 30) and \(\left[T_\mathrm{low}^\mathrm{win},T_\mathrm{high}^\mathrm{win}\right]\) for the remainder of the year, are defined a priori.  
  \item \textbf{Environmental inputs:} At each timestep, the current air temperature \(T_a\), relative humidity RH, and mean radiant temperature \(T_r\) are read from the observation vector.  
\end{itemize}

\subsubsection{Baseline Control (reference)}
\label{sec:reference_control}
The reference controller adjusts set-points based solely on the air temperature relative to the comfort range. Let
\[
(T_\mathrm{low},\,T_\mathrm{high}) = 
\begin{cases}
(T_\mathrm{low}^\mathrm{sum},\,T_\mathrm{high}^\mathrm{sum}), & \text{summer},\\
(T_\mathrm{low}^\mathrm{win},\,T_\mathrm{high}^\mathrm{win}), & \text{winter},
\end{cases}
\]
expanded by \(\pm2\)\,$\degree$C if unoccupied. Denote the current heating and cooling set‐points by \(T_\mathrm{htg}\) and \(T_\mathrm{clg}\). Then:
\begin{equation}
\begin{aligned}
\text{if } T_a &< T_\mathrm{low}: & T_\mathrm{htg}&\leftarrow T_\mathrm{htg}+1,\quad T_\mathrm{clg}\leftarrow T_\mathrm{clg}+1,\\
\text{if } T_a &> T_\mathrm{high}: & T_\mathrm{htg}&\leftarrow T_\mathrm{htg}-1,\quad T_\mathrm{clg}\leftarrow T_\mathrm{clg}-1,\\
\text{otherwise:} & & T_\mathrm{htg},\,T_\mathrm{clg}&\text{ unchanged.}
\end{aligned}
\end{equation}

\subsubsection{PMV-Based Setpoint Control (pmv)}
\label{sec:pmv_control}
The PMV controller uses the predicted Predicted Mean Vote (PMV) from the \texttt{pmv\_ppd\_iso} function, computed over the current conditioned air state. We define
\[
\mathrm{PMV} = f_\mathrm{ISO}\bigl(T_a,\,T_r,\,\mathrm{RH},\,\mathrm{met}=1.1,\,\mathrm{clo}\bigr),
\]
where \(\mathrm{clo}=0.65\) in summer and \(1.0\) in winter (adjusted for unoccupied periods as above). Denoting the mean PMV by \(\overline{\mathrm{PMV}}\), the set-point adjustment is
\begin{equation}
\begin{aligned}
\text{if } \overline{\mathrm{PMV}} &< -0.5: & T_\mathrm{htg}&\leftarrow T_\mathrm{htg}+1,\quad T_\mathrm{clg}\leftarrow T_\mathrm{clg}+1,\\
\text{if } \overline{\mathrm{PMV}} &> +0.5: & T_\mathrm{htg}&\leftarrow T_\mathrm{htg}-1,\quad T_\mathrm{clg}\leftarrow T_\mathrm{clg}-1,\\
\quad \quad & \text{otherwise:} & & T_\mathrm{htg},\,T_\mathrm{clg} \text{ unchanged.}
\end{aligned}
\end{equation}

Both controllers’ outputs \((T_\mathrm{htg},\,T_\mathrm{clg})\) are clipped to the environment’s allowable action space before being sent to EnergyPlus. These two modes serve as benchmarks against which we compare our ML­-enhanced and grid-search optimized strategies.

\subsubsection{LightGBM- and PINN-VAE- Based Control (lightgbm \& pv)}
\label{sec:lightgbm_control}

The two ML/DL-based controllers use our pre‐trained Gradient Boosting model (lightgbm) and PINN-VAE model (pv) respectively to predict the proxy occupants’ Thermal Sensation Vote (TSV) and applies a simple bang–bang adjustment of the HVAC set‐points. At each timestep, the following procedure is executed:

\begin{enumerate}
  \item \textbf{Feature assembly.} Construct a feature vector (or matrix) \(\mathbf{x}\) containing the current indoor air temperature \(T_a\), relative humidity RH, mean radiant temperature \(T_r\), clothing insulation (\(\mathrm{clo}\)), etc., exactly as in Section~\ref{sec:control_strategies}.
  \item \textbf{TSV prediction.}  
    \[
      \widetilde{\mathrm{TSV}} = \mathrm{ModelWrapper.predict}(\mathbf{x}),
    \]
    where \(\widetilde{\mathrm{TSV}}\in\mathbb{R}^n\) (one prediction per zone or sample).  
  \item \textbf{Aggregate comfort metric.} Compute the median sensation vote:
    \[
      \mathrm{TSV}_{\mathrm{med}} = \mathrm{median}\bigl(\widetilde{\mathrm{TSV}}\bigr).
    \]
  \item \textbf{Bang–bang set‐point update.} Let \(T_{\mathrm{htg}}\) and \(T_{\mathrm{clg}}\) be the current heating and cooling set‐points. Then
    \[
    \begin{cases}
      T_{\mathrm{htg}} \leftarrow T_{\mathrm{htg}} + 1,\quad
      T_{\mathrm{clg}} \leftarrow T_{\mathrm{clg}} + 1, 
      & \text{if } \mathrm{TSV}_{\mathrm{med}} < -0.5,\\[6pt]
      T_{\mathrm{htg}} \leftarrow T_{\mathrm{htg}} - 1,\quad
      T_{\mathrm{clg}} \leftarrow T_{\mathrm{clg}} - 1, 
      & \text{if } \mathrm{TSV}_{\mathrm{med}} > +0.5,\\[6pt]
      \text{unchanged,} & \text{otherwise.}
    \end{cases}
    \]
  \item \textbf{Clipping.} Finally, clip \((T_{\mathrm{htg}},T_{\mathrm{clg}})\) to the environment’s allowable action‐space.
\end{enumerate}

\noindent\emph{Computational considerations:}  
Because the wrapper instantiates a LightGBM model and meta‐information only once per episode (caching column order, category mappings, etc.), the per‐timestep cost is dominated by a single vectorized predict call, \(\mathcal{O}(n)\) in the number of samples. Memory overhead is negligible for typical zone counts (\(n<10\)).


\subsubsection{Boundary-Optimized Control (pmv-o, lightgbm-o \& pv-o)}
\label{sec:opt_boundary_pseudocode}

All boundary modes perform the same grid-search over a set of temperature offsets \(\Delta T\) to drive the predicted comfort metric as close as possible to the comfort boundaries \(\pm0.5\). The PMV or TSV values are predicted with different prediction models, i.e., PMV, LightGBM and PINN-VAE (abbreviated hereinafter as pmv-o, lightgbm-0 and pv-o). We summarize the procedure in Algorithm~\ref{alg:boundary_opt}.

\begin{algorithm}[ht]
\caption{Boundary‐Optimized Set‐Point Adjustment}\label{alg:boundary_opt}
\KwIn{Current features $\mathbf{x}$, set‐points $(T_\text{htg},T_\text{clg})$, predictor $\mathsf{pred}(\cdot)$}
\KwData{Shifts $\mathcal{D}=\{-2,\dots,2\}$}
Build batch $\mathbf{X}\leftarrow[\mathbf{x}+\delta_1;\dots;\mathbf{x}+\delta_m]$\;
$\mathbf{y}\leftarrow\mathsf{pred}(\mathbf{X})$\;
Reshape into $\mathbf{Y}\in\mathbb{R}^{m\times n}$\;
\For{$i\leftarrow1$ \KwTo $m$}{
  $s_i^-\leftarrow|\mathrm{med}(\mathbf{Y}_i)+0.5|$\;
  $s_i^+\leftarrow|\mathrm{med}(\mathbf{Y}_i)-0.5|$\;
}
$\delta^-\leftarrow\arg\min_i s_i^-$; $\delta^+\leftarrow\arg\min_i s_i^+$\;
$\delta^*\leftarrow\arg\min_{\delta\in\{\delta^-,\delta^+\}}|\delta|$\;
$T_\text{htg}\mathbin{+{\mkern-8mu}=}\delta^*$; $T_\text{clg}\mathbin{+{\mkern-8mu}=}\delta^*$\;
Clip to action space\;
\end{algorithm}

Because we reset any extreme setpoints to (19$^\circ$C, 27$^\circ$C) before evaluating ∆T, none of the candidate shifts (\pm2$^\circ$C, \pm1$^\circ$C, \pm0.5$^\circ$C) ever crosses the hard bounds [12$^\circ$C, 23.25$^\circ$C]×[23.25$^\circ$C, 30$^\circ$C], thereby eliminating actuator saturation in the grid-search process.
\noindent\emph{Practical considerations:}  
The single batch predict call reduces overhead compared to looping over shifts. Total complexity remains \(\mathcal{O}(mn)\) model evaluations, with memory to hold \(m\!n\) feature rows. In our experiments (\(m=9,n<10\)), this executes in under 150\,ms per timestep, making it suitable for real‐time or near‐real‐time HVAC control loops.

\subsubsection{Physiological Variables Incorporated Boundary-Optimized Control (pv-tsk-strick \& pv-tsk-loose)}
\label{sec:pv_tsk_control}
The PINN-VAE model, with its intrinsically physiological-constrained model design, is capable of predicting physiological variables, including skin temperature ($T_{skin}$), apart from TSV value. Built upon the aforementioned boundary-optimized PINN-VAE control strategy (\texttt{pv-o}), we further incorporate the predicted $T_{skin}$ into control strategy to better leverage the capacities of the advanced model. Specifically, at each action point, the agent evaluates whether both the predicted TSV and $T_{skin}$ falls within the target range while maintaining the same grid search strategy to find the minimum setpoint deviation that meet the both criterion (\texttt{pv-tsk-strict}). Alternatively, as a more relaxed control strategy, the comfort requirement is considered to be met if either the TSV or $T_{skin}$ falls within its respective comfort range (\texttt{pv-tsk-loose}). The $T_{skin}$ target range is set at a conservative comfort range $32.8\,^\circ\mathrm{C} \leq T_{skin} \leq 33.8\,^\circ\mathrm{C}$ as reported by Weiwei et al. \cite{liuUseMeanSkin2015}. Since pv-o itself never generates clipped setpoints, these pv-tsk variants likewise avoid any actuator saturation.

\subsection{Inputs and Outputs}
\paragraph{Weather Files used}\label{sec:weather_noise}
We employed twelve EPW-format weather files as the nominal dataset for year-long EnergyPlus simulations\footnote{Typical EPW fields and structure are defined in the EnergyPlus Weather File Data Dictionary\cite{EPW_Data_Dictionary}}.  
From these, four representative files—selected according to DOE climate zone classifications (e.g., 2B: hot‐dry; 4A: mixed‐humid; 5C: cool‐marine; 7: cold)—were used to benchmark the control strategies outlined in Section~\ref{sec:control_strategies}.  

An additional eight EPW files were reserved for extended validation of PMV-based set‐point optimization. To account for meteorological variability and reinforce the robustness of our findings, we enabled the stochastic weather module in the Sinergym package, which injects Gaussian perturbations (zero mean, $\sigma$ = 2.5 $\degree$C) into the dry-bulb and radiant temperature series for each timestep\cite{Sinergym_Stochastic}.


This stochastic weather module is critical for establishing the statistical robustness of our findings. Rather than evaluating control performance under a single deterministic weather year—which could lead to overfitting to specific weather patterns—each simulation experiences unique weather realizations. The Gaussian perturbations ($\sigma$ = 2.5°C) applied to dry-bulb and radiant temperatures at every timestep ensure that our control strategies are tested against a distribution of possible weather conditions, not just historical averages. This approach inherently provides statistical validation by demonstrating that the performance rankings and energy savings remain consistent despite weather variability, strengthening confidence in the generalizability of our results.
\begin{itemize}
  \item \textbf{EPW format:} Comma-delimited with hourly records for 8,760 hours, containing standard fields such as Year, Month, Day, Hour, Dry-Bulb Temperature, and Sky Infrared Radiation\cite{EPW_Format}.
  \item \textbf{Climate selection:} Four climates (2B, 4A, 5C, 7) chosen to span hot-dry, mixed-humid, cool-marine, and cold conditions.
  \item \textbf{Stochastic noise:} Gaussian noise ($\mu$=0, $\sigma$=2.5 $\degree$C) added to temperature fields via Sinergym’s stochastic environments\cite{Sinergym_Stochastic}, enhancing sensitivity analysis and result stability.
\end{itemize}

\paragraph{Performance Metrics (Overview)}
To compare the different control strategies (reference, PMV, LightGBM, PINN-VAE and optimized variants), we will evaluate the outputs from the EnergyPlus simulation across:  
\begin{itemize}
  \item Annual energy consumption by end use types (and percent savings relative to the reference), on top of energy usage intensitivies.  
  \item Fraction of occupied hours with TSV within the $\pm$0.5 comfort/thermal neutral band.  
  \item Average simulation runtime per annual run.  
\end{itemize}
The results of the comparison of these metrics across different runs are reported in the results section of this paper and further discussed in corresponding discussion sections.%Add more?
The precise mathematical definitions and any additional robustness or combined‐FoM (Combined Figure of Merit) metrics are given in Section~\ref{sec:performance_metrics}.%? I guess perhaps add some more to appendix?


\subsection{Actuator Limits and Saturation Considerations}
\label{sec:actuator_saturation_method}

In \texttt{5ZoneAutoDXVAV} example file, the default heating and cooling setpoints are hard‐limited beyond the EnergyPlus allowable range to
\[
12\,^\circ\mathrm{C} \;\leq T_{h} \;\leq 23.25\,^\circ\mathrm{C} 
\quad\text{and}\quad
23.25\,^\circ\mathrm{C} \;\leq T_{c} \;\leq 30\,^\circ\mathrm{C}.
\]
Whenever a controller requests a setpoint outside these bounds, Sinergym automatically clips it to the nearest limit—a phenomenon we refer to as \emph{actuator saturation}. If left unaddressed, any logic that naively drives $T_{h}<12\,^\circ\mathrm{C}$ or $T_{c}>30\,^\circ\mathrm{C}$ will simply “stick” at the boundary, masking the true comfort–energy trade‐off.

In our initial “pure” implementations (PMV inversion, LightGBM inversion, PV inversion), we observed that repeated attempts to push setpoints beyond $12\,^\circ\mathrm{C}$ or $30\,^\circ\mathrm{C}$ yielded clipped controls that never returned to an interior value. To mitigate this, we introduced a boundary‐optimized grid‐search protocol as follows:  
\begin{enumerate}
  \item \textbf{Interior Reset.} Whenever the previous setpoint pair $(T_{h},T_{c})$ falls outside $[13\,^\circ\mathrm{C},29\,^\circ\mathrm{C}]$, we immediately reset to $(19\,^\circ\mathrm{C},\,27\,^\circ\mathrm{C})$.  
  \item \textbf{$\Delta T$ Grid Search.} From that interior point, we evaluate seven candidate offsets
    \[
      \Delta T \;\in\; \{-2.0,\,-1.0,\,-0.5,\,0.0,\,+0.5,\,+1.0,\,+2.0\}\,\text{°C}
    \]
    simultaneously. Because $(19\,^\circ\mathrm{C} \pm 2\,^\circ\mathrm{C})$ and $(27\,^\circ\mathrm{C} \pm 2\,^\circ\mathrm{C})$ lie strictly within the allowable $[12\,^\circ\mathrm{C},\,30\,^\circ\mathrm{C}]$ band, none of these candidates are ever clipped.  
  \item \textbf{Final Gap Enforcement.} After selecting the best $\Delta T$ (based on $\bigl|\lvert \text{comfort}\rvert - 0.5\bigr|$ plus an energy penalty), we reimpose a $2\,^\circ\mathrm{C}$ heating–cooling gap and then perform one last clamp to guarantee feasibility.  
\end{enumerate}

By acknowledging Sinergym’s actuator saturation up front and embedding this reset + $\Delta T$ grid logic in the methodology, we ensure that none of our “optimized” setpoints are artifactual products of clipping. This approach eliminates “sticking” at $12\,^\circ\mathrm{C}$ or $30\,^\circ\mathrm{C}$ and preserves a meaningful comfort gradient throughout the simulation.
\section{Results and Discussions}
\subsection{Missingness MAR Validity}

The validity of applying imputation techniques like VAE hinges on the nature of the missing data. We formally tested the hypothesis that missingness in key personal variables (\texttt{height\_cm}, \texttt{age}, \texttt{gender}, \texttt{weight\_kg}) was dependent on other observed variables, thereby providing evidence against the MCAR assumption. Univariate tests (Chi-squared, t-tests) and multivariate logistic regressions were employed. The results, originally presented in detail in Table~\ref{tab:MAR_Summary}, consistently showed statistically significant associations ($p < 0.05$) between missingness and observed predictors. The logistic regression models achieved high predictability for missingness in height (AUC = 0.969), age (AUC = 0.984), and weight (AUC = 0.976). While the AUC for gender was lower, significant univariate associations remained. Table~\ref{tab:MAR_Summary} provides a condensed overview of these findings.

Based on domain knowledge of the data collection process, where data represent experimental records aggregated from various contributors, missingness often occurred systematically within specific experimental groups (defined by characteristics such as contributor, location, or building type, aligning with the statistical associations found). This typically reflected a determination by the data contributor at the time of the experiment that recording a specific variable was not required or relevant for that particular experimental condition or protocol. While the possibility that missingness could, in some cases, be related to the unobserved value itself (MNAR) cannot be entirely excluded without more granular information on the exact reason for each missing entry, the observed systematic patterns linked to recorded experimental factors lend stronger support to the MAR assumption. Therefore, we proceeded under the MAR assumption for VAE imputation, acknowledging this as a standard, necessary assumption for applying such imputation techniques in complex observational datasets\cite{Rubin1976}.

\begin{table}[htbp]
    \centering
    \caption{Simplified Summary of MAR Validation Results}
    \label{tab:MAR_Summary}
    \begin{tabular}{l l l}
        \toprule
        Variable & Key Evidence Finding & Overall Assessment (vs. MCAR) \\
        \midrule
        Height (\texttt{height\_cm}) & Multiple Sig. Assoc.; High AUC (0.969) & Evidence against MCAR \\
        Age (\texttt{age}) & Multiple Sig. Assoc.; High AUC (0.984) & Evidence against MCAR \\
        Gender (\texttt{gender}) & Multiple Sig. Assoc.; Low/Mod AUC (0.673) & Evidence against MCAR \\
        Weight (\texttt{weight\_kg}) & Multiple Sig. Assoc.; High AUC (0.976) & Evidence against MCAR \\
        \bottomrule
    \end{tabular}
    \vspace{1ex} % Add some space below the table
    \footnotesize % Smaller font for the note
    Note: Summarizes statistical tests (Chi-squared, t-tests, logistic regression AUC) assessing dependency of missingness on observed variables. Significant findings support MAR/MNAR over MCAR.
\end{table}

These results strongly suggest that the missing data mechanism is consistent with MAR or potentially Missing Not At Random (MNAR). Given the data collection context (aggregation from various experiments), systematic missingness linked to specific protocols or contributor choices aligns well with the MAR assumption. Therefore, we proceeded with VAE-based imputation under the MAR assumption, acknowledging it as a standard and necessary premise for applying such techniques to complex observational datasets \cite{Rubin1976}. Therefore, despite the potential for MNAR in some fields, the dominant missingness patterns are consistent with MAR, justifying the use of VAE-based imputation.


\subsection{BMR Ablation Results}
First and foremost, examining the aggregated out-of-sample lightgbm models performance on the BMRs (calculated only when inputs are available), we were able to calculate their respective relative RMSE, MAE and MAPE as according to Figure~\ref{fig:lightgbm-bmrs}. Across the three heat maps, the top rows are where the overall data availability is better, and towards the bottom the data availability becomes worse. The BMR models yielded consistent improvements across the board with data fill above 80\% when evaluating against RMSe and MAE, and starts to see a variation of performance, where Livingston–Kohlstadt appears to have the best performance across the three models, which is to be expected since it is a much newer calculation approach compared to the other two. It is also worth noticing that when the level of data fill falls below 50\%, all but the Livingston-Kohlstadt variant started to consistently show deterioration of performance, which is to be anticipated since the model's performance improvement now requires more inputs from BMR fields than before. %This is strangely put, revise later.
\begin{figure}[h!]
    \centering
    \includegraphics[width=0.3\linewidth]{fig/lgbrmse.png}
    \includegraphics[width=0.3\linewidth]{fig/lgbmae.png}
    \includegraphics[width=0.3\linewidth]{fig/lgbmape.png}
    \caption{Relative RMSE, MAE and MAPE of lightgbm models trained with different BMR equations}
    \label{fig:lightgbm-bmrs}
\end{figure}

% This translates to ? \% of overall RMSE improvement towards the lightgbm model simply by engineering BMR terms - which is inherently an analytical-equation-driven feature engineering exercise. Yet by working with this one variable only we believe we have demonstrated how adding additional analytically-described features that are truly transformed existing inputs can have the effect of improving the performance of thermal sensation predictor. 

The inclusion of calculated Basal Metabolic Rate (BMR) as an input feature demonstrated tangible benefits for the LightGBM benchmark model. As highlighted previously, the addition of BMR features, particularly leveraging the Livingston-Kohlstadt formulation \cite{LivingstonKohlstadt2005}, yielded an overall performance improvement of approximately 8\% in predictive accuracy (based on RMSE/MAE metrics, see Highlights) compared to the identical LightGBM model trained without BMR inputs. This underscores the value of incorporating physiology-based, analytically derived features, even when based on inputs (age, height, weight) that themselves suffer from missingness, for improving thermal sensation prediction. The Livingston-Kohlstadt variant generally provided the most consistent benefits, particularly in data subsets with higher completeness (fill > 80\%), although performance gains diminished or reversed in highly incomplete data (fill < 50\%), likely due to the unreliability of BMR inputs in those cases.


\subsection{Performance Evaluation}
\subsubsection{Thermal Sensation Prediction: Overall Performance Evaluation}
The performance of the baseline model (\texttt{pmv\_ce}), the benchmark (lightgbm), and the sequentially developed models (vae, \texttt{pv\_pen}, \texttt{pvp\_pen}, ultra) were evaluated using RMSE, MAE, MAPE, SMAPE, and $R^2$ metrics. The results are summarized in Table \ref{tab:performance_updated}.

\begin{table}[h!] % You can adjust the placement specifier [htbp] as needed
\centering
\caption{Overall Model Performance Comparison} % Add your desired caption
\label{tab:performance_updated} % Add a label for referencing
\begin{tabular}{lcccccc}
\toprule
Metric/Model & pmv\_ce & lightgbm & vae     & pv\_pen & pvp\_pen & ultra  \\
\midrule
RMSE   & 1.310   & 0.952    & 1.205   & 0.994   & 0.991    & 0.984  \\
MAE    & 1.001   & 0.711    & 0.880   & 0.745   & 0.745    & 0.739  \\
MAPE   & 101.7\% & 76.3\%   & 99.6\%  & 80.9\%  & 80.6\%   & 79.8\% \\
SMAPE  & 154.2\% & 145.8\%  & 185.3\% & 153.7\% & 152.5\%  & 149.8\%\\
$R^2$     & -0.185  & 0.376    & 0.000   & 0.320   & 0.324    & 0.334  \\
\bottomrule
\end{tabular}
\end{table}
The \texttt{pmv\_ce} baseline exhibited the poorest performance across all metrics, notably yielding a negative $R^2$ value (-0.185). The lightgbm benchmark, which handled missing data internally, achieved the lowest RMSE (0.952) and MAE (0.711), and the highest $R^2$ (0.376). The standard \gls{vaeplain} model performed poorly, with high errors (e.g., SMAPE 185.3\%) and an $R^2$ of 0.000. LightGBM's superior overall RMSE/MAE metrics might stem from its capacity to capture highly complex, non-linear interactions within the full feature set, potentially including latent patterns derived implicitly from the missing data itself. While effective for raw prediction, this contrasts with the PINN-VAE approach, where the VAE structure and explicit physiological constraints may introduce beneficial regularization and interpretability at the cost of potentially smoothing over some intricate data patterns.

Introducing physics-informed constraints (\texttt{pv\_pen}) significantly improved upon the vae, reducing RMSE from 1.205 to 0.994 and increasing $R^2$ from 0.000 to 0.320. Further incorporating personalization (\texttt{pvp\_pen}) resulted in marginal improvements over \texttt{pv\_pen}, with RMSE decreasing to 0.991 and $R^2$ increasing to 0.324. The \gls{ultra} model, using imputed data with LightGBM, performed slightly worse than the lightgbm benchmark that handled missingness implicitly (RMSE 0.984 vs 0.952; $R^2$ 0.334 vs 0.376).

The results indicate that the standard PMV-based model (\texttt{pmv\_ce}) is insufficient for accurately predicting thermal sensation in the extent of dataset this large. The LightGBM benchmark (lightgbm) demonstrated strong predictive power, achieving the best overall metrics by effectively leveraging the dataset, including missing values.

However, while lightgbm's performance is high, its internal mechanisms for handling missingness operate essentially as a "black box". The model may learn predictive patterns from the presence of missing data itself, but this does not necessarily translate into actionable or interpretable insights. For instance, superior performance relying on missingness patterns does not yield practical recommendations like "ensure this data point remains missing for better comfort prediction". This inherent limitation in interpretability regarding missing data motivates the development of models that aim for a more structured or causal understanding, even if benchmark metrics are slightly lower.

A detailed feature importance analysis (e.g., using SHAPley values or permutation importance) for the LightGBM model was not conducted as part of this study. LightGBM primarily served as a high-performance benchmark, establishing a reference point for predictive accuracy attainable by a state-of-the-art gradient boosting model that handles missing data implicitly. The focus of our analysis was on comparing the performance and characteristics of the explicitly constrained PINN-VAE approaches against this benchmark, rather than performing an in-depth feature attribution analysis of the benchmark model itself.

\begin{figure}[h!]
    \centering
    \includegraphics[width=0.5\linewidth]{fig/rmse_all_7.png}
    \caption{Alternative Model RMSEs across thermal sensation buckets}
    \label{fig:rmse-7bucket}
\end{figure}

The poor performance of the standard VAE (vae) highlights the challenges of applying generative models directly without domain constraints. Attempting to impute the missing values during the 5-fold epochs, the trained VAE ended up predicting all predictions as positive values, leading to not only a $R^2$ score at 0, but also an unusable model. In contrast, the substantial improvement observed as in Figure~\ref{fig:rmse-7bucket} with the PINN-VAE (\texttt{pv\_pen}) the significant benefit of integrating physics-informed constraints when performing parallel training of model alongside missing data imputation. This step aligns the model more closely with underlying physical principles and brought its performance much closer to the lightgbm benchmark, particularly in explained variance ($R^2$).

The addition of personal physiological indices (\texttt{pvp\_pen}) provided further, albeit smaller, performance gains, validating the hypothesis that personalization can enhance prediction accuracy within this physically-constrained framework. Comparing ultra and lightgbm, the results suggest that for this particular dataset and imputation strategy, allowing LightGBM to handle missing data internally was more advantageous than the explicit imputation method employed.

Overall, there is a potential trade-off: the lightgbm benchmark offers peak predictive accuracy by leveraging all patterns including missingness, but with limited interpretability regarding why missingness might be predictive. The PINN-VAE approaches (\texttt{pv\_pen}, \texttt{pvp\_pen}), while achieving slightly lower metrics in this instance, represent a promising pathway for integrating domain knowledge and personalization, potentially offering more interpretable and structurally sound predictions for thermal sensation.
% \begin{figure}[h!]
%     \centering
%     \includegraphics[width=0.49\linewidth]{fig/mods_rmse_hm.png}
%     \includegraphics[width=0.49\linewidth]{fig/mods_mae_hm.png}
%     \caption{RMSE and MAE of alternative models tested vs Baseline (PMV\_CE) and Benchmark (lightgbm)}
%     \label{fig:enter-label}
% \end{figure}
To beteter visualize how each model fared within each of the thermal sensation bins, we create a boxen plot across various thermal sensation bins as shown in Figure~\ref{fig:residual-boxen} to compare their respective residuals against each other.
\begin{figure}[h!]
    \centering
    \includegraphics[width=0.55\linewidth]{fig/res_boxen.png}
    \caption{Residuals from models measured}
    \label{fig:residual-boxen}
\end{figure}

Qualitatively, the distribution of residuals ocross different models can be simply compared against one another where lightgbm, despite its accuracy tends to result in a much wider range of prediction resituals, which is adequately modulated by our proposed PINN-framework across the two model alternatives. 

\subsubsection{Performance by Fill}
% In the meantime, we also evaluated the performance metrics of the models across different data availaibility `rungs', specifically the percentages of fields that are not null across all raw inputs, divided into 10\% buckets. This leads to Figure~\ref{fig:10perc_metrics}. Similar to the results reported with respect to thermal sensation bins, switching of VAE leads to RMSE and MAE increase across all levels of not-null values. Across the various percentage fill buckets reported in Figure~\ref{fig:10perc_metrics}, this translates to an average 31.5\% and 28.2\% of RMSE and MAE increase from lightgbm as the benchmark model. By incorporating the PINN framework, the RMSE and MAE increase dropped to 4.7\% and 5.1\% respectively, which ultimately dropped to 3.4\% and 3.9\% for the final lightGBM model leveraging the imputed dataset.%Expand a little...? 

Analyzing model performance across different levels of data completeness, stratified by the percentage of non-null input features per record (Figure~\ref{fig:10perc_metrics}), reveals expected trends. As illustrated by the heatmaps for RMSE and MAE, the predictive accuracy of all models generally degrades as the data fill percentage decreases. For instance, the PINN-VAE models (pv\_pen, pvp\_pen) show increased error metrics in the 40-60\% fill range compared to the >90\% fill range. This observed degradation is likely attributable to the fundamentally reduced information content available for prediction in records with higher missingness. Key variables, including the personal data required for reliable BMR calculations (\texttt{age}, \texttt{height}, \texttt{weight}) and potentially other influential environmental or contextual factors, are more frequently absent in these lower-fill data rows. This inherent lack of crucial input information naturally limits the predictive capability achievable by any modeling approach. Nonetheless, the PINN-VAE models consistently outperformed the unconstrained VAE across all fill levels and maintained performance closer to the LightGBM benchmark, particularly at higher fill percentages.


\begin{figure}[h!]
    \centering
    \includegraphics[width=0.99\linewidth]{fig/PerfMetrics_10.png}
    \caption{Root mean squared error (left), mean absolute error (middle) and mean absolute percentage error across all models evaluated}
    \label{fig:10perc_metrics}
\end{figure}
\subsubsection{Around Thermal Neutral Zone}%Does this belong to discussion...?

We had already seen that both PINN-based models were noticeably more precise than LightGBM inside the neutral zone, but a single RMSE does not show why. 
%%%%???
The results we got are as Table~\ref{tab:dir_penalty}. 

\begin{table}[t]
\centering
\caption{Direction-penalty metric  
$\Delta_{\text{dir}}=\text{MAE}_{\text{wrong sign}}-\text{MAE}_{\text{right sign}}$  
(lower is better; a positive value means the model’s error grows once it crosses to the wrong side of neutral).
Bootstrapped 95\,\% confidence intervals are based on 1\,000 resamples of the
out-of-sample rows produced by the original 5-fold CV.}
\label{tab:dir_penalty}
\begin{tabular}{lcc}
\toprule
\textbf{Model} & $\boldsymbol{\Delta_{\text{dir}}}$ & 95\,\% CI \\ 
\hline
PMV\textsubscript{CE}            & 0.1236 & [\,0.1130,\;0.1333\,] \\
LightGBM              & 0.0459 & [\,0.0386,\;0.0530\,] \\
VAE                   & $-0.1565$ & [\,-0.1657,\;-0.1470\,] \\
PINN–VAE (\texttt{pv\_pen})      & $-0.0036$ & [\,$-0.0112$,\;0.0035\,] \\
PINN–VAE–PPI (\texttt{pvp\_pen}) & $-0.0181$ & [\,$-0.0250$,\;$-0.0108$\,] \\
Ultra LightGBM (\texttt{ultra})  & 0.0342 & [\,0.0273,\;0.0409\,] \\

\bottomrule
\end{tabular}\label{tb:neutral-pen}
\end{table}

According to Table~\ref{tab:dir_penalty}, LightGBM and PMV\textsubscript{CE} incur clear asymmetric costs ($\approx$0.05 and 0.12 votes, respectively), meaning their errors inflate sharply when they cross neutral.
The physiology-guided networks behave differently: the plain PINN-VAE shows no significant asymmetry (CI straddles zero), and the PINN-VAE–PPI variant actually drives the wrong-sign error \emph{below} the right-sign error by $\approx$0.02 votes. Inspecting results from the table we noticed the performance of PINN-VAE was effective in a few interesting ways:

\paragraph{PINN variants level the playing field.}  
  With the physiology constraint—and particularly when the PPI gate is used—errors remain tightly bounded even if the network picks the wrong side of neutral.  We acknowledge the negative $\Delta$ may sound counter-intuitive, but it simply means the model’s over- and under-shoots have similar magnitudes.

\paragraph{LightGBM and PMV are asymmetric.}  
  Their right-sign mistakes are modest, but once they cross the neutral point the error grows by 0.03–0.12 votes, a pattern that matches occupant complaints about “getting the sign wrong.”  For control applications a small symmetric error is preferable: the HVAC system can compensate with a fixed offset.  Large asymmetric errors, by contrast, cause abrupt, perception-opposite responses.

To put the table into a better perspective, we also visually broke down the performance change with Figure~\ref{fig:neu_zone}. The improvement of RMSE of VAE observed in the thermal neutral zone in ~\ref{fig:rmse-7bucket} gets borken down clearly with 

\begin{figure}[h!]
    \centering
    \includegraphics[width=0.95\linewidth]{fig/w5_dir.png}
    \caption{Neutral Zone Performance Lift Analysis: Zone-specific accuracy, MAE of correct/wrong direction and their respective differences}
    \label{fig:neu_zone}
\end{figure}

This analysis using the direction-penalty statistic, $\Delta_{dir} = \text{MAE}_{\text{wrong sign}} - \text{MAE}_{\text{right sign}}$, reveals distinct behaviors near the thermal neutral zone ($|TSV| \le 0.5$). As shown in Table~\ref{tb:neutral-pen}, baseline models like $PMV_{CE}$ and LightGBM exhibit significantly positive $\Delta_{dir}$ values, indicating that their prediction errors substantially increase when the predicted sensation incorrectly crosses the neutral threshold (i.e., predicting warm when the true sensation is cool, or vice-versa). In contrast, the physiology-guided networks demonstrate markedly different behavior. The PINN-VAE (pv\_pen) shows no significant asymmetry ($\Delta_{dir} \approx -0.004$, with a 95\% CI spanning zero), while the PINN-VAE-PPI (pvp\_pen) achieves a slightly negative $\Delta_{dir}$ ($\approx -0.018$). This indicates that, for the PINN-based models, the magnitude of prediction error remains tightly bounded and symmetric, regardless of whether the prediction falls on the correct or incorrect side of neutral. This behavior, attributed to the physiological constraints preventing unrealistic predictions, is highly desirable for control applications where large, asymmetric errors can lead to inappropriate system responses\cite{Jain2018}. The PINN framework effectively removes the sharp performance cliff observed in traditional models when they misjudge the direction of thermal sensation relative to neutral.

Thus, adding the physiological loss—and, when available, the personal-profile embedding—not only improves overall RMSE but also removes the sharp performance cliff that traditional models hit when they guess the wrong side of comfort.

\subsection{Imputed Features from VAE Architecture}
A key capability of the VAE framework is data imputation. We qualitatively assessed the imputed distributions for variables with high initial missingness, namely \texttt{age}, \texttt{height}, and \texttt{weight} (Figure~\ref{fig:impu_three}), as well as the intermediate physiological variable $\hat{T}_{skin}$ (Figure~\ref{fig:skin_inputs}). Visual inspection reveals that the PINN-VAE produces more plausible distributions compared to an unconstrained VAE, avoiding unrealistic values (e.g., age < 0, height < 100 cm) due to the implicit guidance from the learned data structure and explicit physiological constraints.

While Table~\ref{tab:performance_updated} shows that LightGBM and PINN-VAE models achieve superior performance metrics, it is particularly notable that the unconstrained VAE performs the worst across all evaluation criteria—despite having access to the same data and architecture depth. This underperformance confirms that imputation alone, without physiological regularization, is insufficient for meaningful TSV prediction. The poor performance of VAE, despite its flexibility, highlights the importance of constraining intermediate outputs through biophysical priors. The improvement in PINN-VAE and further gains in PINN-VAE–PPI validate the necessity of combining imputation and physiological structure.


\begin{figure}[h!]
    \centering
    \includegraphics[width=0.5\linewidth]{fig/imp_age.png}
    \includegraphics[width=0.5\linewidth]{fig/imp_height_cm.png}
    \includegraphics[width=0.5\linewidth]{fig/imp_weight_kg.png}
    \caption{Imputed age, height and weight field values histogram between simple VAE and PINN-VAE}
    \label{fig:impu_three}
\end{figure}
Quantitative evaluation of imputation accuracy (e.g., calculating RMSE or MAE between imputed values and ground-truth values on a held-out test set) was not performed in this study. The primary variables requiring the most imputation (\texttt{age}, \texttt{height}, \texttt{weight}) originally suffered from very high levels of missing data (up to 70\%), leaving insufficient reliable ground-truth data within the dataset to conduct a robust quantitative validation of imputation performance for these specific fields. Furthermore, optimizing imputation accuracy itself, for instance through comparisons involving artificial data masking, was considered secondary to the main research objective of evaluating the overall PINN-VAE framework for constrained prediction in the presence of missing data. A dedicated study focusing on quantitatively optimizing and validating imputation strategies for these specific variables could be a valuable direction for future work.

Similarly, a quantitative comparison quantifying the frequency of physiological constraint violations in imputed intermediate variables ($\hat{T}_{core}, \hat{T}_{skin}, \hat{w}$) between the constrained PINN-VAE and an unconstrained VAE was not conducted. This is primarily due to the lack of comprehensive ground-truth physiological measurements in the original aggregated dataset; the 'input' physiological values used for comparison were themselves derived from Gagge's model \cite{gagge1986standard}. The analysis therefore focused on the qualitative demonstration, via imputed distributions (Figure~\ref{fig:impu_three}), that the physics-informed constraints effectively guide the imputation process towards generating more physiologically realistic values compared to an unconstrained approach.

As was outlined in the methodology, we employed a soft constraint on the imputation of the age, height and weight columns. Our imputed values for the three fields becomes much more reasonable in terms of coverage of values in contrast to age below 0 years old and height below 100 centimeters, and weight below 0 kg etc. Visually examining the VAE against the PINN-VAE results for these fields where null values are commonly observed, PINN-VAE does appear to be providing a much more reasonable imputation quality. %Expand further?

Beyond qualitatively comparing the results of imputed data of the combined dataset, we also compared the intermediate outputs of $T_{skin}$ as outlined in methodology to assess how PINN-VAE framework affects the resulting intermediate outputs of our proposed framework. Recognizing that $T_{skin}$ were initially calculated with analytical equations developed by Gagge et al. following their built-in assumptions, it is important to point out that its original inputs may need to be further confined. We also want to point out that, while direct quantitative validation of imputation accuracy for variables like age or height was precluded by the high degree of initial missingness, the qualitative plausibility demonstrated (Figure~\ref{fig:skin_inputs}) and the improved downstream predictive performance compared to an unconstrained VAE support the effectiveness of the constrained imputation within the overall framework's objectives.
\begin{figure}[h!]
    \centering
    \includegraphics[width=0.45\linewidth]{fig/skinimput.png}
    \caption{Imputed Skin temperature with different skin temperature penalty settings ranging from 30.5$\degree$ C to 35.5 $\degree$ C}
    \label{fig:skin_inputs}
\end{figure}


Figure~\ref{fig:skin_inputs} further illustrates the model's thermoregulatory response by showing the distribution of the imputed intermediate variable $\hat{T}_{\text{skin}}$ under progressively narrower soft constraints. The initial physiological constraint applied a range of $[30, 36]^{\circ}$C. As depicted in the subsequent panels of Figure 11, when this range was incrementally narrowed (from $[30.5, 35.5]^{\circ}$C down to $[33.0, 34.0]^{\circ}$C), the resulting distribution of imputed $\hat{T}_{\text{skin}}$ shifted and became more concentrated. This demonstrates a unique, non-linear relationship where tightening the soft constraint on skin temperature directly influences the model's imputed physiological state. While a similar analysis for $\hat{T}_{\text{core}}$ is not presented, its predicted values remained tightly clustered within physiologically plausible bounds (constrained between $[36, 38]^{\circ}$C), reflecting the body's stricter core temperature regulation. Nonetheless, the observed modulation of $\hat{T}_{\text{skin}}$ distribution based on its constraints highlights the effective physiological regulation capability embedded within the proposed PINN-VAE framework, where adjustments in one physiological parameter influence the overall predicted state.
\section{Limitations and Future Work}

\subsection{Limitations}

\paragraph{Scope of Physiological Modeling:} The current implementation relies on Gagge’s two-node model to constrain physiological variables (core and skin temperature, skin wettedness). While widely used, this model simplifies thermal regulation and may not fully capture inter-individual variability or responses under extreme thermal conditions, physical exertion, or transient exposures.

\paragraph{Soft Constraints and Lack of Hard Guarantees:} The physiological constraints are enforced via soft penalties, which guide but do not guarantee adherence to biophysical bounds. In rare cases, especially under high imputation uncertainty or data sparsity, the model may still yield borderline physiological outputs that would not be biologically plausible under strict energy balance.

\paragraph{Dataset Bias and Underrepresented Populations:} Despite combining two large-scale datasets, certain groups—such as older adults, children, or individuals in tropical and arid climates—remain underrepresented. As a result, predictions for these populations may not generalize without additional targeted data.

\paragraph{Reliance on MAR Assumption:} While statistical tests and data context support the Missing At Random (MAR) assumption, any unobserved dependencies influencing missingness (i.e., potential MNAR characteristics) could theoretically affect imputation accuracy. However, VAE-based methods are generally considered robust under MAR, which we established as the most plausible scenario.

\paragraph{Operational Feasibility for Real-Time Applications:} While model interpretability and prediction quality are improved, the added complexity of latent-variable inference and physiology-based losses increases training and inference time. This may limit near-term deployment in embedded or low-latency building control systems without further optimization.


\subsection{Future Work}
\noindent We therefore believe there are at least the following possible directions of future work for PINN-VAE and its variants for thermal sensation:
\paragraph{Joint Optimization of Comfort and Physiology:} Given that PINN-VAE outputs not only thermal sensation but also skin and core temperatures, future work could explore control algorithms that co-optimize predicted thermal comfort and physiological deviation from neutral. This opens new opportunities for occupant-aware HVAC strategies that balance subjective sensation with biophysical safety margins.

\paragraph{Integration of Wearable and Streaming Data:} Future deployments may leverage wearable sensors or IoT devices to dynamically update personal physiological inputs (e.g., real-time skin temperature or heart rate), enabling the PPI interface to adapt to intra-occupant variation or temporal changes such as illness, stress, or activity level.

\paragraph{Architectural Simplification and Model Compression:} To support deployment in building systems with limited computational capacity, model compression techniques such as pruning, quantization, or teacher-student distillation could be applied to PINN-VAE. These techniques would aim to preserve interpretability while reducing inference time and memory usage.

\paragraph{Alternative Physiological and Comfort Models:} The current architecture could be extended to accommodate other biothermal models (e.g., Stolwijk\cite{Stolwijk1971} or multi-segment models like JOS3\cite{Takahashi2002JOS}) or adaptive comfort formulations that better reflect behavioral and cultural variability. Comparative evaluation would help assess trade-offs in complexity and predictive value.

\paragraph{Field Validation in Operational Settings:} Beyond retrospective evaluation, a key next step is validating PINN-VAE in real-world building environments, comparing predictions to occupant feedback and observed control behavior. This will test not only accuracy but also acceptance and integration into control workflows.

\section{Conclusions}
This study introduced a physiology-informed neural framework (PINN-VAE) that jointly addresses two core challenges in thermal comfort modeling: (i) imputing missing values in large observational datasets and (ii) improving predictive performance while preserving physiological interpretability. By embedding soft physiological constraints into a variational autoencoder pipeline and leveraging both tabular and personal-profile inputs, our model successfully balances the flexibility of data-driven imputation with the structure of thermoregulatory realism.

We demonstrate that PINN-VAE achieves comparable or superior performance to traditional state-of-the-art models, particularly near the thermal neutral zone, where directional prediction errors are often most critical for occupant-centric applications. The inclusion of intermediate physiological outputs—core and skin temperatures—not only constrains unrealistic predictions but also offers interpretability advantages unavailable in tree-based or purely statistical models.

Furthermore, the availability of intermediate physiological outputs ($\hat{T}_{\text{core}}, \hat{T}_{\text{skin}}, \hat{w}$) offers enhanced interpretability; beyond predicting sensation, monitoring these variables could enable systems to infer underlying physiological states, potentially allowing for more proactive or health-aware environmental control strategies that anticipate discomfort or thermal stress.

Beyond thermal comfort, the PINN-VAE architecture offers a generalizable framework for combining latent-variable learning with physics-informed constraints. Its use could extend to domains such as biomechanics, energy metabolism modeling, or any setting where tabular physiological data are incomplete yet governed by known physical relationships.

The Personalized Physiology Interface (PPI) further strengthens this framework by enabling the use of raw demographic variables (e.g., age, height, gender, weight) without degrading predictive performance. In doing so, the model preserves important physiological signal diversity without needing population-wide feature averaging or scaling.

Compared to baseline LightGBM and unconstrained VAE architectures, our framework shows a measurable reduction in neutral-zone RMSE and direction-penalty asymmetry, reinforcing the hypothesis that physiology-informed modeling improves both accuracy and robustness. Overall, PINN-VAE offers a promising and interpretable path forward for occupant-aware control, particularly in systems where physiological realism and explainability are essential for deployment.



\section*{Data and Code Availability}
The code developed for the PINN-VAE model and the analysis presented in this paper is available on GitHub at [URL Currently Private] The final imputed dataset, generated using the best-performing model fold as described in Section~\ref{sec:PINN_VAE_Arch}, is available at request (currently sitting on private repository on github). The original ASHRAE Global Thermal Comfort Database II and Chinese Thermal Comfort Datasets are publicly available from their respective sources cited in the text.
\section{Conclusions}
This study presents the first comprehensive co-simulation framework directly integrating data-driven thermal sensation models with EnergyPlus, overcoming historical reliance on PMV as a proxy for occupant comfort. By leveraging a dataset of 148,148 thermal sensation votes from the ASHRAE Global Thermal Comfort Database II and Chinese datasets, we enabled real-time predictions of occupant thermal sensation during building simulation—a capability previously challenging to achieve.

Our evaluation of seven control strategies across four diverse climate zones yielded several critical insights. First, optimized PMV-based control via grid-search algorithm significantly outperformed naive comfort-driven control, achieving 15–18.5\% energy savings compared to a 4.3–23.5\% energy penalty with non-optimized PMV control. Remarkably, optimized PMV achieved comparable energy efficiency to advanced machine learning models (LightGBM and PINN-VAE) while requiring substantially lower computational resources. Among ML models, LightGBM exhibited robust predictive performance but at tenfold runtime relative to optimized PMV. Meanwhile, the PINN-VAE model, although novel in incorporating physiological realism through predictions of skin temperatures ($T_{skin}$), showed only marginal improvements in energy performance while requiring prohibitively long runtimes (10–15 minutes per year simulation), potentially limiting its direct real-time applicability.

The ability of PINN-VAE to predict physiological variables nonetheless introduced valuable opportunities for enhancing occupant comfort. By explicitly incorporating predicted skin temperature into HVAC control logic, we achieved a notable 14.85\% improvement in comfort hours at a modest 1.97\% energy increase, demonstrating the practical potential of physiologically-informed control strategies. The incorporation of stochastic weather perturbations in all simulations ensures these results reflect expected performance under real-world weather variability, not idealized conditions.

It's also important to point out our co-simulation methodology revealed that naive ML-based controllers frequently could encounter actuator saturation—a critical deployment challenge invisible to traditional comfort prediction studies—thereby validating the necessity of integrated simulation frameworks for practical control development.

Critically, our re-analysis showed that naive LightGBM and PV controllers appeared to outperform PMV only because they were frequently clipped to 12$^\circ$C or 30$^\circ$C (actuator saturation), not because they improved comfort.
This finding underscores that any future ML-based control strategy must explicitly account for actuator bounds when optimizing setpoints; otherwise, apparent energy savings may simply reflect saturation at the extremes.

The core implication of our findings explicitly challenges the assumption that increasing model complexity inherently yields superior HVAC performance. Instead, carefully optimized classical models such as PMV provide near-equivalent energy and comfort performance at significantly reduced computational and infrastructural costs, thus offering a highly practical pathway for broad deployment. The demonstrated feasibility and performance of our co-simulation framework underscore that future research should prioritize addressing critical operational constraints—such as continuous setpoint adjustments, multi-zone coordination, and adaptive predictive horizons—rather than solely refining comfort prediction accuracy.

For the building controls community, this study reinforces that integrated, occupant-centric co-simulation frameworks are both viable and valuable. By shifting research and development efforts towards practical optimization strategies and advanced physiological sensing integration, we can significantly enhance HVAC operational efficacy. Future work should extend this co-simulation methodology across additional building types, HVAC systems, and occupant demographics, ensuring broader generalizability and practical relevance. Ultimately, addressing fundamental physical and operational constraints is crucial for fully realizing the potential benefits of advanced comfort models in real-world building control applications.

\newpage
\section*{Nomenclature}
\begin{table}[h!]
\centering
\resizebox{\textwidth}{!}{
\begin{tabular}{ll}
\toprule
\textbf{Symbol / Acronym} & \textbf{Description} \\
\midrule
\textbf{PMV}          & Predicted Mean Vote – analytical thermal sensation index (ISO 7730) \\
\textbf{TSV}          & Thermal Sensation Vote – occupant-reported thermal comfort level \\
\textbf{PINN-VAE}     & Physics-Informed Neural Network with Variational Autoencoder \\
\textbf{LightGBM}     & Light Gradient Boosting Machine – tree-based ML model for tabular data \\
\textbf{EUI}          & Energy Use Intensity – building energy consumption per unit floor area (kWh/$m^2$) \\
$T_{skin}$            & Mean skin temperature ($^\circ$C) – predicted or modeled via biophysical models \\
$T_{core}$            & Core body temperature ($^\circ$C) – estimated using Gagge's two-node model \\
\textbf{clo}          & Clothing insulation level (1 clo $\approx$ 0.155 $m^2$·K/W) \\
\textbf{met}          & Metabolic rate (1 met $\approx$ 58.2 W/$m^2$) \\
\textbf{HVAC}         & Heating, Ventilation, and Air Conditioning system \\
\textbf{ASHRAE}       & American Society of Heating, Refrigerating and Air-Conditioning Engineers \\
\textbf{ISO 7730}     & International Standard defining PMV/PPD indices \\
\textbf{DOE}          & U.S. Department of Energy – defines standard climates/building archetypes \\
\textbf{Sinergym}     & Python–EnergyPlus co-simulation and RL environment \\
\textbf{EnergyPlus}   & Whole-building energy simulation engine \\
\textbf{EMS}          & Energy Management System – rule-based control scripting in EnergyPlus \\
\textbf{PV-o}         & PINN-VAE optimized control mode \\
\textbf{PMV-o}        & PMV control mode optimized via grid-search \\
\textbf{LB-o}         & LightGBM control mode optimized via grid-search \\
\textbf{RBC}          & Rule-Based Controller – baseline on/off control scheme \\
\textbf{Comfort hours} & Simulation hours where TSV is within comfort range ($-0.5 \leq$ TSV $\leq +0.5$) \\
\textbf{Runtime}      & Time to execute one year of simulation (seconds) \\
\bottomrule
\end{tabular}
}
\caption{List of symbols and acronyms used throughout the paper.}
\end{table}



% %% The Appendices part is started with the command \appendix;
% %% appendix sections are then done as normal sections
% \appendix
% \section{Example Appendix Section}
% \label{app1}

% Appendix text.


\bibliographystyle{elsarticle-num}
\bibliography{cas-refs}

\end{document}

\endinput
%%
%% End of file `elsarticle-template-num.tex'.
